% ----------------------------------------------------------
% Subseção Expansão lógica
% ----------------------------------------------------------
\subsection{Expansão lógica}
A lógica primordial (negação de si) cria expansões lógicas infinitas. Uma expansão lógica é análoga a um universo. O primeiro momento lógico é o início de uma dessas expansões, porém existem infinitas possibilidades de negação do primeiro momento lógico, o que revela a possibilidade das infinitas expansões binomiais, pois o SER é ilógico e imutável, portanto pode ser negado infinitamente. A negação do SER não transforma SER em NÃO SER, pois este é imutável (o NÃO SER é apenas o outro lado do SER). 
\begin{figure}[H]
\caption{Momentos lógicos iniciais}
\label{fig:third_logical_moment}
\centering
\includegraphics[scale=.85]{sections/images/third_logical_moment.jpg}
\floatfoot{Exemplo dos três primeiros momentos de uma expansão.}%\footnotemark}
\end{figure}
%\footnotetext{Fonte: note}

Com base na Figura \ref{fig:third_logical_moment} pode-se extrair as seguintes observações em relação ao primeiro, segundo e terceiro momentos lógicos:
\begin{description}
   \item[Primeiro momento lógico] A negação da lógica primordial a si, a subdivide em duas unidades, que somadas são o todo ilógico. Apesar dessas partes terem proporções diferentes, elas exprimem as mesmas quantidades de pontos ou possibilidades de mudança, uma vez que são representações da lógica primordial, que \textit{ad infinitum}. A parte fracionada em azul representa a proporção da negação lógica em relação à sua unidade.
   \item[Segundo momento lógico] É gerado pela negação das duas sub-lógicas primordiais, fracionadas no primeiro momento lógico. Na impossibilidade dessas frações lógicas do primeiro momento lógico continuar negando a si, faria com que elas fossem incapazes de negar suas unidades que formam o todo, ou seja, seriam incapazes de negar suas duas unidades e por consequência o todo que é formado precisamente por elas, o que faria da lógica apenas ilógica (SER), uma unicidade. As partes fracionadas em azul representam a proporção da negação lógica em relação às suas respectivas unidades.
   \item[Terceiro momento lógico] Decorre da negação do segundo momento lógico, assim como o segundo momento lógico decorre da negação do primeiro e assim por diante.
\end{description}

A cada negação ou subnegação da lógica primordial, seus novos valores são influenciados pelos valores adjacentes do momento lógico anterior. Na figura \ref{fig:imposition_of_binomial_expansion}, a lógica primordial nega a si gerando o primeiro momento lógico com o valor [0,2].  No segundo momento lógico, suas subdivisões estão contidas no limite imposto pelo valor do primeiro momento lógico. Os pontos do terceiro momento lógico, por exemplo, sofrem as imposições dos valores do segundo momento lógico que por sua vez sofrem a imposição do primeiro. Os valores de momentos lógicos descendentes sofrem imposições acumulativas dos valores dos momentos lógicos anteriores. À imposição de um valor em seus dois valores imediatamente descendentes denominou-se sincronismo lógico. Isso é o que pode ser visto no triângulo de pascal. Esse sincronismo irá levar à frequências de amostras cada vez maiores em intervalos cada vez menores, que serão vistos na próxima seção do Teorema central do limite.

\begin{figure}[H]
\caption{Imposição da expansão lógica}
\label{fig:imposition_of_binomial_expansion}
\centering
\includegraphics[scale=.85]{sections/images/imposition_of_binomial_expansion.jpg}
\floatfoot{Imposição acumulativa aos momentos lógicos descendentes.}%\footnotemark}
\end{figure}
%\footnotetext{Fonte: note}

No triângulo de pascal, Figura \ref{fig:pascal_triangle}, cada número é os dois números acima mais próximos somados. Esse número representa quantos diferentes possíveis caminhos levam até ele. Por exemplo, o número [4], na Figura \ref{fig:pascal_triangle}, representa os quatro diferentes caminhos que levam até ele. Os coeficientes binômias encontrados no triangulo de Pascal representam apenas as quantidades de imposições sofridas por cada valor de um momento lógico. Um outro aspecto interessante do triângulo de pascal é a sequência de Fibonacci, Figura \ref{fig:pascal_triangle_fibonacci} \cite{mathisfun_pascal_triangle}.  

\begin{figure}[H]
\centering
	\begin{subfigure}[H]{0.47\linewidth}
	\centering
	\includegraphics[width=.55\linewidth]{sections/images/pascal_triangle.jpg}
	\caption{}
	\label{fig:pascal_triangle}
	\end{subfigure}
\hfill
	\begin{subfigure}[H]{0.47\linewidth}
	\centering
	\includegraphics[width=.9\linewidth]{sections/images/pascal_triangle_fibonacci.jpg}
	\caption{}
	\label{fig:pascal_triangle_fibonacci}
	\end{subfigure}%
\caption{Características do triângulo de Pascal}

\floatfoot{Fonte: MathsIsFun, 2019.\protect\footnotemark}
\end{figure}
\footnotetext{\url{www.mathsisfun.com/pascals-triangle.html}}

O NÃO SER da lógica primordial é análogo a uma constante abstrata, ou seja, suas infinitas negações e subnegações transcendem o tempo. Todas essas infinitas negações acontecem no tempo zero. A incapacidade da lógica negar a si por um intervalo entre suas negações faria a lógica SER ilógica nesse intervalo, por menor que este seja, o que quebraria a premissa primordial da lógica, NÃO SER. Em outras palavras, é como se fosse “todos vezes todos”, ou seja, não é preciso esperar o negação do primeiro momento lógico, pois todos as subnegações do segundo momento lógico são possíveis para todos as negações do primeiro momento lógico e assim por diante. A lógica é como um algoritmo composto de apenas uma constante auto executada, uma sequência simultânea. É a consciência que conduz a experiência do tempo, não pela criação da sequência de mudanças que é simultânea, mas sim pela ordem dessa sequência, que nada mais é que do que a observação da ordem das mudanças de cada momento lógico.

Algumas respostas podem ajudar a esclarecer o que é essa sequência simultânea:
\begin{description}
   \item[Todas as negações acontecem simultaneamente?] Sim, infinitas negações na ausência de tempo, ou tempo zero.
   \item[Como ou porque essa simultaneidade acontece?] Acontecem em uma sequência de negações da lógica a si mesma, no tempo zero, onde em nenhum momento a lógica converte-se em SER, garantindo assim a premissa primordial da constante lógica, NÃO SER.
   \item[O que é uma sequência simultânea?] É a negação da lógica a si (uma sequência ou ordem) no tempo zero, ou seja, em nenhum momento a lógica passa a SER durante as infinitas negações (simultaneidade). Sequência simultânea é o sinônimo da constante lógica NÃO SER.
\end{description}