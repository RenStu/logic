\begin{table}[H]
\caption{Probabilidade da distribuição binomial}
\floatfoot{Tabela gerada pelo algoritmo BinomialDistribuion\_PROB com a distribuição binomial de 100 a 10000. \footnotemark}
\label{tab:10000_all}
\resizebox{\textwidth}{!}{%
\begin{tabular}{cccc
>{\columncolor[HTML]{8D3CE1}}c 
>{\columncolor[HTML]{5754D6}}c 
>{\columncolor[HTML]{8FFFFB}}c c}
\hline
\cellcolor[HTML]{95ABEC} & \cellcolor[HTML]{95ABEC} & \multicolumn{2}{c}{\cellcolor[HTML]{95ABEC}} & \cellcolor[HTML]{95ABEC} & \cellcolor[HTML]{95ABEC} & \cellcolor[HTML]{95ABEC} & \cellcolor[HTML]{95ABEC} \\
\cellcolor[HTML]{95ABEC} & \cellcolor[HTML]{95ABEC} & \multicolumn{2}{c}{\cellcolor[HTML]{95ABEC}} & \cellcolor[HTML]{95ABEC} & \cellcolor[HTML]{95ABEC} & \cellcolor[HTML]{95ABEC} & \cellcolor[HTML]{95ABEC} \\
\multirow{-3}{*}{\cellcolor[HTML]{95ABEC}\textbf{Meta}} & \multirow{-3}{*}{\cellcolor[HTML]{95ABEC}\textbf{\begin{tabular}[c]{@{}c@{}}Soma do \\ Range\end{tabular}}} & \multicolumn{2}{c}{\multirow{-3}{*}{\cellcolor[HTML]{95ABEC}\textbf{Range}}} & \multirow{-3}{*}{\cellcolor[HTML]{95ABEC}\textbf{\begin{tabular}[c]{@{}c@{}}Total de \\ Amostras\end{tabular}}} & \multirow{-3}{*}{\cellcolor[HTML]{95ABEC}\textbf{\begin{tabular}[c]{@{}c@{}}Amostras \\ do Range\end{tabular}}} & \multirow{-3}{*}{\cellcolor[HTML]{95ABEC}\textbf{\begin{tabular}[c]{@{}c@{}}\% das \\ Amostras \\ do Range\end{tabular}}} & \multirow{-3}{*}{\cellcolor[HTML]{95ABEC}\textbf{\begin{tabular}[c]{@{}c@{}}Range de $\approx$ 28\% \\ das Amostras \\ do Range\end{tabular}}} \\ \hline
99,99\% & 99,994\% & 31 & 70 & \textbf{101} & \textbf{39} & \textbf{38\%} & 72,87\% \\ \hline
\cellcolor[HTML]{C0C0C0}99,99\% & \cellcolor[HTML]{C0C0C0}99,992\% & \cellcolor[HTML]{C0C0C0}73 & \cellcolor[HTML]{C0C0C0}128 & \textbf{201} & \textbf{55} & \textbf{27\%} & \cellcolor[HTML]{C0C0C0}{\color[HTML]{000000} 71,11\%} \\ \hline
99,99\% & 99,991\% & 117 & 184 & \textbf{301} & \textbf{67} & \textbf{22\%} & 72,73\% \\ \hline
\cellcolor[HTML]{C0C0C0}99,99\% & \cellcolor[HTML]{C0C0C0}99,990\% & \cellcolor[HTML]{C0C0C0}162 & \cellcolor[HTML]{C0C0C0}239 & \textbf{401} & \textbf{77} & \textbf{19\%} & \cellcolor[HTML]{C0C0C0}70,62\% \\ \hline
99,99\% & 99,991\% & 207 & 294 & \textbf{501} & \textbf{87} & \textbf{17\%} & 73,64\% \\ \hline
\cellcolor[HTML]{C0C0C0}99,99\% & \cellcolor[HTML]{C0C0C0}99,991\% & \cellcolor[HTML]{C0C0C0}253 & \cellcolor[HTML]{C0C0C0}348 & \textbf{601} & \textbf{95} & \textbf{15\%} & \cellcolor[HTML]{C0C0C0}72,96\% \\ \hline
99,99\% & 99,991\% & 299 & 402 & \textbf{701} & \textbf{103} & \textbf{14\%} & 72,69\% \\ \hline
\cellcolor[HTML]{C0C0C0}99,99\% & \cellcolor[HTML]{C0C0C0}99,990\% & \cellcolor[HTML]{C0C0C0}346 & \cellcolor[HTML]{C0C0C0}455 & \textbf{801} & \textbf{109} & \textbf{13\%} & \cellcolor[HTML]{C0C0C0}72,69\% \\ \hline
99,99\% & 99,991\% & 392 & 509 & \textbf{901} & \textbf{117} & \textbf{12\%} & 72,86\% \\ \hline
\cellcolor[HTML]{C0C0C0}99,99\% & \cellcolor[HTML]{C0C0C0}99,991\% & \cellcolor[HTML]{C0C0C0}439 & \cellcolor[HTML]{C0C0C0}562 & \textbf{1001} & \textbf{123} & \textbf{12\%} & \cellcolor[HTML]{C0C0C0}73,16\% \\ \hline
99,99\% & 99,991\% & 486 & 615 & \textbf{1101} & \textbf{129} & \textbf{11\%} & 73,54\% \\ \hline
\cellcolor[HTML]{C0C0C0}99,99\% & \cellcolor[HTML]{C0C0C0}99,991\% & \cellcolor[HTML]{C0C0C0}533 & \cellcolor[HTML]{C0C0C0}668 & \textbf{1201} & \textbf{135} & \textbf{11\%} & \cellcolor[HTML]{C0C0C0}71,45\% \\ \hline
99,99\% & 99,991\% & 580 & 721 & \textbf{1301} & \textbf{141} & \textbf{10\%} & 72,06\% \\ \hline
\cellcolor[HTML]{C0C0C0}99,99\% & \cellcolor[HTML]{C0C0C0}99,990\% & \cellcolor[HTML]{C0C0C0}628 & \cellcolor[HTML]{C0C0C0}773 & \textbf{1401} & \textbf{145} & \textbf{10\%} & \cellcolor[HTML]{C0C0C0}72,68\% \\ \hline
99,99\% & 99,991\% & 675 & 826 & \textbf{1501} & \textbf{151} & \textbf{10\%} & 73,31\% \\ \hline
\cellcolor[HTML]{C0C0C0}99,99\% & \cellcolor[HTML]{C0C0C0}99,990\% & \cellcolor[HTML]{C0C0C0}723 & \cellcolor[HTML]{C0C0C0}878 & \textbf{1601} & \textbf{155} & \textbf{9\%} & \cellcolor[HTML]{C0C0C0}71,76\% \\ \hline
99,99\% & 99,991\% & 770 & 931 & \textbf{1701} & \textbf{161} & \textbf{9\%} & 72,49\% \\ \hline
\cellcolor[HTML]{C0C0C0}99,99\% & \cellcolor[HTML]{C0C0C0}99,990\% & \cellcolor[HTML]{C0C0C0}818 & \cellcolor[HTML]{C0C0C0}983 & \textbf{1801} & \textbf{165} & \textbf{9\%} & \cellcolor[HTML]{C0C0C0}73,20\% \\ \hline
99,99\% & 99,990\% & 866 & 1035 & \textbf{1901} & \textbf{169} & \textbf{8\%} & 71,90\% \\ \hline
\cellcolor[HTML]{C0C0C0}99,99\% & \cellcolor[HTML]{C0C0C0}99,990\% & \cellcolor[HTML]{C0C0C0}914 & \cellcolor[HTML]{C0C0C0}1087 & \textbf{2001} & \textbf{173} & \textbf{8\%} & \cellcolor[HTML]{C0C0C0}72,67\% \\ \hline
99,99\% & 99,990\% & 1394 & 1607 & \textbf{3001} & \textbf{213} & \textbf{7\%} & 71,86\% \\ \hline
\cellcolor[HTML]{C0C0C0}99,99\% & \cellcolor[HTML]{C0C0C0}99,991\% & \cellcolor[HTML]{C0C0C0}1877 & \cellcolor[HTML]{C0C0C0}2124 & \textbf{4001} & \textbf{247} & \textbf{6\%} & \cellcolor[HTML]{C0C0C0}72,47\% \\ \hline
99,99\% & 99,990\% & 2363 & 2638 & \textbf{5001} & \textbf{275} & \textbf{5\%} & 72,38\% \\ \hline
\cellcolor[HTML]{C0C0C0}99,99\% & \cellcolor[HTML]{C0C0C0}99,990\% & \cellcolor[HTML]{C0C0C0}2850 & \cellcolor[HTML]{C0C0C0}3151 & \textbf{6001} & \textbf{301} & \textbf{5\%} & \cellcolor[HTML]{C0C0C0}72,75\% \\ \hline
99,99\% & 99,990\% & 3338 & 3663 & \textbf{7001} & \textbf{325} & \textbf{4\%} & 72,32\% \\ \hline
\cellcolor[HTML]{C0C0C0}99,99\% & \cellcolor[HTML]{C0C0C0}99,990\% & \cellcolor[HTML]{C0C0C0}3827 & \cellcolor[HTML]{C0C0C0}4174 & \textbf{8001} & \textbf{347} & \textbf{4\%} & \cellcolor[HTML]{C0C0C0}72,18\% \\ \hline
99,99\% & 99,990\% & 4316 & 4685 & \textbf{9001} & \textbf{369} & \textbf{4\%} & 72,23\% \\ \hline
\cellcolor[HTML]{C0C0C0}99,99\% & \cellcolor[HTML]{C0C0C0}99,990\% & \cellcolor[HTML]{C0C0C0}4806 & \cellcolor[HTML]{C0C0C0}5195 & \textbf{10001} & \textbf{389} & \textbf{3\%} & \cellcolor[HTML]{C0C0C0}72,42\% \\ \hline
\end{tabular}%
}
\end{table}
\footnotetext{O Apêndice \ref{app:algoritmos} é dedicado a clarificar o algoritmo BinomialDistribuion\_PROB e validar o fórmula da probabilidade binomial geral usada por ele.}
\vspace{-8mm}
\begin{description}
   \item[Meta] Porcentagem das amostras observadas;
   \item[Soma do Range] Porcentagem que o \textbf{"Range"} atingiu a \textbf{"Meta"}, da mediana para as bordas, descentralizado;
   \item[Range] Range de amostras onde a \textbf{"Meta"} foi atingida do \textbf{"Total de Amostras"};
   \item[Total de Amostras] Exibe o range total avaliado, no caso da primeira linha da tabela o valor 101 corresponde às possibilidades de 0 a 100;
   \item[Amostras do Range] Quantidade de amostras do \textbf{"Range"};
   \item[Porcentagem das Amostras do Range] Porcentagem que o \textbf{"Range} representa do \textbf{"Total de Amostras"};
   \item[Range de $\approx$ 28\% das Amostras do Range] Esse range é  subconjunto do \textbf{"Range"}, formado a partir da mediana somando 14\% a direita e a esquerda, totalizando 28\%. Esses 28\% correspondem a aproximadamente 72\% das \textbf{"Amostras do Range"} e está por sua vez correspondem a 99,99\% da população total. O restante, que representam 72\% do tamanho do \textbf{"Range"}, correspondem a aproximadamente 28\% das amostras. Isso condiz com o Princípio de Pareto também conhecido como a regra do 80/20 e que também pode ser 70/30 ou 90/10, por exemplo \cite{administradores_principio_pareto}.
\end{description}
\bigbreak

