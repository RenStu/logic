% ----------------------------------------------------------
% Introdução
% ----------------------------------------------------------
\section*{Introduction}
\addcontentsline{toc}{section}{Introduction}

The reasoning of this text arose as an answer to the most essential question that philosophy can formulate and that science and philosophy have not been able to fully answer so far, which is: whether there is something instead of nothing, or why is there something instead of nothing? 
This question was first asked by the philosopher Gottfried Wilhelm Leibniz in a letter in 1697 and is often described as the greatest philosophical question \cite{leibniz_origin_of_things}.

The answer to that question comes from the answer of what logic is. In exploring what logic [is] and what it [IS NOT], it gave rise to a theory about the origin of everything, all things. Logic in its essence refers to nothingness, what [IS NOT], that is, it denies itself (negates itself). The self-negation of primordial logic can be abstracted recursively (negates itself, infinitely) into three axioms that are the basis of the central theorem of this theory.

The logic [NOT BEING] is in conformity with NOTHING, because if on the one hand the logic [IS NOT], on the other hand [is] its opposite, that is, illogical and unchanging. In this duality, one has existence grounded by a logic that [IS NOT], while [is] illogical, unchanging and non-existent. 

\noindent The text is organized according to the following hierarchy:
%	{\scriptsize
	\begin{enumerate}[label*=\arabic*.]
	   \item Logic
	   \begin{enumerate}[label*=\arabic*.]
		   \item Logical Expansion
		   \item Central Limit Theorem
		   \item Consciousness
			   \begin{enumerate}[label*=\arabic*.]
				   \item Infinite
				   \item Waves 
				   \begin{enumerate}[label*=\arabic*.]
				   		\item Wavelength and amplitude
				   		\item Entanglement
				   		\item Leap
				   \end{enumerate}  
				   \item Time
				   \item Space
				   \begin{enumerate}[label*=\arabic*.]
				   		\item Spiral
				   \end{enumerate} 
				   \item Fundamental forces
				   \item Dark matter and dark energy
				   \item Antimatter
				   \item Black Hole
				   \item Observer and life
				   \begin{enumerate}[label*=\arabic*.]
				   		\item Senses
				   \end{enumerate}  
			   \end{enumerate}   
	   \end{enumerate}
	\end{enumerate}
%	}

Initially, it is defined what logic [is] and especially what [IS NOT], so its consonance with nothingness is presented. It is then described how this primordial logic, the essence of any logic, develops through its logical expansion. Then it is observed that the samples combined at each stage of this expansion characterize the fundamentals of the central limit theorem, generating new logics (logical waves and sub-waves). These are the logical aspects responsible for saying what is the fundamental nature of reality, knowledge, and existence. 

\bigbreak