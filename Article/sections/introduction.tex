% ----------------------------------------------------------
% Introdução
% ----------------------------------------------------------
\section*{Introdução}
\addcontentsline{toc}{section}{Introdução}

O raciocínio deste texto surgiu como resposta à pergunta mais essencial que a filosofia pode formular e que a ciência até então não foi capaz de responder plenamente, que é: se existe algo ao invés de nada ou porque existe algo ao invés de nada? 
Essa pergunta foi feita pela primeira vez pelo filosofo Gottfried Wilhelm Leibniz em uma carta de 1697 e é frequentemente descrita como a maior questão filosófica \cite{ leibnizbrasil_origem_das_coisas}.

A resposta a essa pergunta vem da resposta do que é a lógica. Ao explorar o que a lógica é e o que ela NÃO É, deu origem a uma teoria a respeito da origem de tudo, de todas as coisas. A lógica em sua essência remete ao nada, que NÃO É, ou seja, nega a si (nega ser). A autonegação da lógica (negação de si) pôde ser abstraída recursivamente (nega ser, infinitamente) em três axiomas que são a base do teorema núcleo dessa teoria.

A lógica NÃO SER é consonante com o NADA, pois se por um lado a lógica \underline{NÃO É}, por outro \underline{É} seu contrário, ou seja, ilógica e imutável. Nessa dualidade, tem-se a existência fundamentada pela lógica que \underline{NÃO É}, enquanto \underline{É} ilógica, imutável e inexistente. 

\noindent O texto está disposto na seguinte hierarquia:
%	{\scriptsize
	\begin{enumerate}[label*=\arabic*.]
	   \item Lógica
	   \begin{enumerate}[label*=\arabic*.]
		   \item Expansão lógica
		   \item Teorema central do limite
		   \item Consciência
			   \begin{enumerate}[label*=\arabic*.]
				   \item Infinito
				   \item Ondas 
				   \begin{enumerate}[label*=\arabic*.]
				   		\item Comprimento e amplitude
				   		\item Entrelaçamento
				   		\item Salto
				   \end{enumerate}  
				   \item Tempo
				   \item Espaço
				   \begin{enumerate}[label*=\arabic*.]
				   		\item Espiral
				   \end{enumerate} 
				   \item Forças fundamentais
				   \item Matéria escura e energia escura
				   \item Antimatéria
				   \item Buraco negro
				   \item O observador e a vida
			   \end{enumerate}   
	   \end{enumerate}
	\end{enumerate}
%	}

Inicialmente é definido o que é a lógica e principalmente o que ela NÃO É, assim é apresentado sua consonância ao nada. Depois é descrito como essa lógica primordial, a essência de qualquer lógica, se desenvolve por meio de sua expansão lógica. Em seguida é observado que as amostras combinadas em cada passo dessa expansão caracterizam os fundamentos do teorema central do limite, gerando novas lógicas (ondas e sub-ondas lógicas). Esses são os aspectos lógicos responsáveis em dizer qual é a natureza fundamental da realidade, do conhecimento e da existência. 

\bigbreak