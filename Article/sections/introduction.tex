% ----------------------------------------------------------
% Introdução
% ----------------------------------------------------------
\section*{Introdução}
\addcontentsline{toc}{section}{Introdução}

O raciocínio deste texto surgiu como resposta a pergunta mais essencial que a filosofia pode formular e que a ciência até então não foi capaz de responder plenamente, que é se existe algo ao invés de nada ou porque existe algo ao invés de nada? 
Essa pergunta foi feita pela primeira vez pelo filosofo Gottfried Wilhelm Leibniz em uma carta de 1697 e é frequentemente descrita como a maior questão filosófica \cite{ leibnizbrasil_origem_das_coisas}.

A resposta a essa pergunta vem da resposta do que é a lógica. Ao explorar o que a lógica é e o que ela NÃO É, deu origem a uma teoria a respeito da origem de tudo, de todas as coisas. A lógica em sua essência remete ao nada, que NÃO É, ou seja, nega a si (nega ser). A negação de si gera expansões binomiais, no qual suas amostras combinadas em cada passo dessa expansão se aproximam da distribuição normal e se aproximam do centro dessa distribuição infinitamente, o que configura o teorema central do limite. A compreensão destes dois conceitos matemáticos, expansão binomial e teorema central do limite, são essenciais para entendimento dessa teoria.

Os passos da expansão binomial, originados da negação lógica a si, regidos pela probabilidade descrita no teorema central do limite compreendem a consciência e tornam visíveis o que é e o porquê de seus aspectos mais perceptíveis: infinito, tempo, espaço, gravidade, matéria escura, energia escura e buraco negro. Ao responder a pergunta essencial deste estudo também é possível responder as principais questões da ciência, o que é e o porquê é a consciência e seus aspectos, pois são provenientes de um mesmo aspecto lógico.

A lógica NÃO SER é consonante com o NADA, pois se a lógica NÃO É ela também É seu contrário, ou seja, ilógica e imutável. Nessa dualidade, tem-se a existência fundamentada pela lógica que "nega a si", enquanto, por outro lado É ilógica, imutável e inexistente. O texto está disposto na seguinte hierarquia:
\begin{enumerate}[label*=\arabic*.]
   \item Lógica
   \begin{enumerate}[label*=\arabic*.]
   \item Expansão binomial
   \item Teorema central do limite
   \item Consciência
   \begin{enumerate}[label*=\arabic*.]
   \item Infinito
   \item Tempo
   \item Espaço
   \item Gravidade
   \item Matéria escura e energia escura
   \item Buraco negro
   \end{enumerate}   
%   \item Observações
   \end{enumerate}
\end{enumerate}
\bigbreak
Primeiro é definido o que é a lógica e principalmente o que ela NÃO É, assim é apresentado sua consonância ao nada. Depois é descrito como essa lógica primordial, essência de qualquer lógica, se desenvolve gerando novas lógicas por meio de sua expansão binomial. Em seguida é observado que as amostras combinadas em cada passo dessa expansão são regidas pela probabilidade descrita no teorema central do limite o qual da origem ao que a consciência. A negação da nada a si gera as expansões binomiais que são regidas pela probabilidade descrita no teorema central do limite (consciência) é o aspecto lógico responsável em dizer o porquê e o que é o infinito, o tempo, o espaço, a gravidade, a matéria escura, a energia escura e o buraco negro, inicialmente. 