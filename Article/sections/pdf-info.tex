% ---
% Configurações do pacote backref
% Usado sem a opção hyperpageref de backref
\renewcommand{\backrefpagesname}{Citado na(s) página(s):~}
% Texto padrão antes do número das páginas
\renewcommand{\backref}{}
% Define os textos da citação
\renewcommand*{\backrefalt}[4]{
	\ifcase #1 %
		Nenhuma citação no texto.%
	\or
		Citado na página #2.%
	\else 
		Citado #1 vezes nas páginas #2.%
	\fi}%
% ---

% --- Informações de dados para CAPA e FOLHA DE ROSTO ---
\titulo{Lógica, Apenas Lógica}
\tituloestrangeiro{Logic, Logic Only}

\autor{
Renan Aparecido Stuchi\thanks{E-mail: \href{malito:ren.stuchi@gmail.com}{ren.stuchi@gmail.com} | GitHub: private repo \url{https://github.com/RenStu/logic} } }

\local{Brasil}
\data{2020, v-1.1.1}
% ---

% ---
% Configurações de aparência do PDF final

% alterando o aspecto da cor azul
\definecolor{blue}{RGB}{41,5,195}

% informações do PDF
\makeatletter
\hypersetup{
     	%pagebackref=true,
		pdftitle={\@title}, 
		pdfauthor={\@author},
    	pdfsubject={Lógica (negação de si) - A natureza fundamental da realidade, do conhecimento e da existência.},
	    pdfcreator={LaTeX with abnTeX2},
		pdfkeywords={lógica. }{nada. }{tudo. }{binômio. }{expansão lógica. }{teorema central do limite. }{consciência. }{infinito. }{ondas. }{tempo. }{espaço. }{forças fundamentais. }{força gravitacional. }{força eletromagnética. }{força nuclear forte. }{força nuclear fraca. }{matéria escura. }{energia escura. }{antimatéria. }{buraco negro.},
%		pdfkeywords={lógica. }{nada. }{tudo. }{infinito. },
		colorlinks=true,       		% false: boxed links; true: colored links
    	linkcolor=blue,          	% color of internal links
    	citecolor=blue,        		% color of links to bibliography
    	filecolor=magenta,      	% color of file links
		urlcolor=blue,
		bookmarksdepth=4
}

\makeatother