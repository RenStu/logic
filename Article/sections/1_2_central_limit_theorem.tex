% ----------------------------------------------------------
% Subseção Teorema central do limite
% ----------------------------------------------------------
\subsection{Teorema central do limite}

O primeiro momento lógico divide e gera duas sub-lógicas que negam a si mesmas gerando novas subdivisões ou sub-lógicas que estão presentes no segundo momento lógico. Essa divisão e subdivisões são frações de uma lógica.  Assim as sub-lógicas de um momento lógico subnegam o SER, porém unificadas ou somadas elas negam o SER. Assim, a soma dessas frações é a representação de uma unidade lógica. Os momentos lógicos são suas sub-lógicas que representam uma unidade lógica, como são as subunidades de espaço ou tempo (metros e centímetros ou minutos e segundos).  Ao dividir uma laranja em duas partes exatas e entregar uma parte para cada pessoa, elas terão uma fração da laranja, mas se entregarmos as duas partes à mesma pessoa ela terá uma laranja em sua totalidade, não duas laranjas, ela terá duas frações da laranja que representa uma laranja.

\begin{figure}[H]
\caption{Momentos lógicos subdivididos}
\label{fig:second_logical_moment}
\centering
\includegraphics[scale=.85]{sections/images/second_logical_moment.jpg}
\floatfoot{Exemplo dos dois primeiros momentos de uma expansão.}%\footnotemark}
\end{figure}
%\footnotetext{Fonte: note}

E na Figura \ref{fig:logical_units} pode ser observada a representação do primeiro e segundo momentos lógicos como unidades lógicas. As unidades lógicas do segundo momento lógico em diante são obtidas pela soma de suas sub-lógicas. Assim, o segundo momento lógico, por exemplo, é a soma das subunidades [0,1+ 0,32] mostras na Figura \ref{fig:second_logical_moment}. No primeiro momento lógico a lógica em sua unidade nega a si, ou seja, nega SER. Nos demais momentos a união das sub-lógicas ou subdivisões representam essa unidade lógica que nega SER ilógica.

\begin{figure}[H]
\caption{Momentos lógicos unificados}
\label{fig:logical_units}
\centering
\includegraphics[scale=.85]{sections/images/logical_units.jpg}
\floatfoot{Exemplo dos dois primeiros momentos unificados de uma expansão.}%\footnotemark}
\end{figure}
%\footnotetext{Fonte: note}

O Teorema Central do Limite afirma que a distribuição amostral de uma população se aproxima de uma distribuição normal à medida que as quantidades das amostras aumentam, independente da forma da distribuição da população. Esse fato é especialmente verdadeiro para a quantidade de amostras acima de 30. Um simples teste que demonstra esse fato é o simples lançamento de dados não viciados. Quanto maior for o número de lançamento do dado, maior a probabilidade de o gráfico parecer com o gráfico da distribuição normal \cite{statisticshowto_teorema_central_limite}.

A Figura \ref{fig:statisticsbyjim_central_limit_theorem} ilustra o fundamento do Teorema central do limite quanto ao fato da aproximação do gráfico ao gráfico da distribuição normal e da aproximação da distribuição da população à mediana à medida que as amostras aumentam. No gráfico são distribuídas 500.000 amostras randomicamente em cada range amostral de ([5-vermelho], [20-azul] e [40-verde]), a cor cinza mostra os valores distorcidos da população \cite{statisticsbyjim_central_limite_theorem_explainded}.

\begin{figure}[H]
\caption{Aproximação do gráfico à distribuição normal e aproximação da distribuição da população à mediana}
\label{fig:statisticsbyjim_central_limit_theorem}
\centering
\includegraphics[scale=1]{sections/images/statisticsbyjim_central_limit_theorem.jpg}
\floatfoot{500.000 amostras distribuídas randomicamente em cada range amostral de ([5-vermelho], [20-azul] e [40-verde] \cite{statisticsbyjim_central_limite_theorem_explainded}\protect\footnotemark.}
\end{figure}
\footnotetext{\url{statisticsbyjim.com/basics/central-limit-theorem}}

Quanto maior o número de subunidades lógicas, quanto mais elas ocorrem, maior será o número de amostras dessa população. E acima de 30 amostras, facilmente alcançado pela lógica primordial que tende ao infinito, a tendência da população é se aproximar da distribuição normal e da média populacional infinitamente. A Figura \ref{fig:logical_subunits} abaixo ilustra a representação de uma amostra no gráfico de expansão binomial usado neste estudo.

\begin{figure}[H]
\caption{Amostra de uma população}
\label{fig:logical_subunits}
\centering
\includegraphics[scale=1]{sections/images/logical_subunits.jpg}
\floatfoot{Representação de uma amostra no gráfico de expansão binomial.}%\footnotemark}
\end{figure}
%\footnotetext{Fonte: note}

É importante notar, conforme Figura \ref{fig:trend_chart_of_normal_distribution}, que o equilíbrio probabilístico das variações nas faixas a direita e esquerda da mediana, causadas pela distribuição dos momentos lógicos com suas amostras unificadas, podem ilustrar a doutrina dos contrários de Heráclito de Éfeso \cite{brasilescola_heraclito}.

\begin{figure}[H]
\caption{Equilíbrio probabilístico das amostras contrárias em relação à mediana}
\label{fig:trend_chart_of_normal_distribution}
\centering
\includegraphics[scale=1.1]{sections/images/trend_chart_of_normal_distribution.jpg}
\floatfoot{1000 momentos lógicos em 500500 amostras distribuídas randomicamente em um range amostral de 1 e 9.}%\footnotemark}
\end{figure}
%\footnotetext{Fonte: note}

Na Tabela \ref{tab:10000_all} está a probabilidade da distribuição binomial entre 100 a 10000, consonante à amostras unificadas. Sua construção se deu com a fórmula da probabilidade binomial geral, que representa uma distribuição uniforme, por meio do algoritmo BinomialDistribuion\_PROB clarificado no Apêndice \ref{app:algoritmos} \cite{mathisfun_binomial_distribution}.
\begin{align*}
f(k;n,p) &= \binom{n}{k} p^k(1 - p)^{n-k}
\end{align*}
A distribuição binomial se comporta como o lançamento de moedas (cara ou coroa), mas poderia ser utilizado nesse estudo outras distribuições discretas, como o lançamento de dados não viciados, e as observações deste estudo continuariam as mesmas, pois o Teorema central do limite é independente da forma da distribuição da população \cite{statisticsbyjim_central_limite_theorem_explainded}.
\begin{table}[H]
\caption{Probabilidade da distribuição binomial}
\floatfoot{Tabela gerada pelo algoritmo BinomialDistribuion\_PROB com a distribuição binomial de 100 a 10000. \footnotemark}
\label{tab:10000_all}
\resizebox{\textwidth}{!}{%
\begin{tabular}{cccc
>{\columncolor[HTML]{8D3CE1}}c 
>{\columncolor[HTML]{5754D6}}c 
>{\columncolor[HTML]{8FFFFB}}c c}
\hline
\cellcolor[HTML]{95ABEC} & \cellcolor[HTML]{95ABEC} & \multicolumn{2}{c}{\cellcolor[HTML]{95ABEC}} & \cellcolor[HTML]{95ABEC} & \cellcolor[HTML]{95ABEC} & \cellcolor[HTML]{95ABEC} & \cellcolor[HTML]{95ABEC} \\
\cellcolor[HTML]{95ABEC} & \cellcolor[HTML]{95ABEC} & \multicolumn{2}{c}{\cellcolor[HTML]{95ABEC}} & \cellcolor[HTML]{95ABEC} & \cellcolor[HTML]{95ABEC} & \cellcolor[HTML]{95ABEC} & \cellcolor[HTML]{95ABEC} \\
\multirow{-3}{*}{\cellcolor[HTML]{95ABEC}\textbf{Meta}} & \multirow{-3}{*}{\cellcolor[HTML]{95ABEC}\textbf{\begin{tabular}[c]{@{}c@{}}Soma do \\ Range\end{tabular}}} & \multicolumn{2}{c}{\multirow{-3}{*}{\cellcolor[HTML]{95ABEC}\textbf{Range}}} & \multirow{-3}{*}{\cellcolor[HTML]{95ABEC}\textbf{\begin{tabular}[c]{@{}c@{}}Total de \\ Amostras\end{tabular}}} & \multirow{-3}{*}{\cellcolor[HTML]{95ABEC}\textbf{\begin{tabular}[c]{@{}c@{}}Amostras \\ do Range\end{tabular}}} & \multirow{-3}{*}{\cellcolor[HTML]{95ABEC}\textbf{\begin{tabular}[c]{@{}c@{}}\% das \\ Amostras \\ do Range\end{tabular}}} & \multirow{-3}{*}{\cellcolor[HTML]{95ABEC}\textbf{\begin{tabular}[c]{@{}c@{}}Range de +/- \\ 14\% (28\%) da \\ Faixa Central\end{tabular}}} \\ \hline
99,99\% & 99,994\% & 31 & 70 & \textbf{101} & \textbf{39} & \textbf{38\%} & 72,87\% \\ \hline
\cellcolor[HTML]{C0C0C0}99,99\% & \cellcolor[HTML]{C0C0C0}99,992\% & \cellcolor[HTML]{C0C0C0}73 & \cellcolor[HTML]{C0C0C0}128 & \textbf{201} & \textbf{55} & \textbf{27\%} & \cellcolor[HTML]{C0C0C0}{\color[HTML]{000000} 71,11\%} \\ \hline
99,99\% & 99,991\% & 117 & 184 & \textbf{301} & \textbf{67} & \textbf{22\%} & 72,73\% \\ \hline
\cellcolor[HTML]{C0C0C0}99,99\% & \cellcolor[HTML]{C0C0C0}99,990\% & \cellcolor[HTML]{C0C0C0}162 & \cellcolor[HTML]{C0C0C0}239 & \textbf{401} & \textbf{77} & \textbf{19\%} & \cellcolor[HTML]{C0C0C0}70,62\% \\ \hline
99,99\% & 99,991\% & 207 & 294 & \textbf{501} & \textbf{87} & \textbf{17\%} & 73,64\% \\ \hline
\cellcolor[HTML]{C0C0C0}99,99\% & \cellcolor[HTML]{C0C0C0}99,991\% & \cellcolor[HTML]{C0C0C0}253 & \cellcolor[HTML]{C0C0C0}348 & \textbf{601} & \textbf{95} & \textbf{15\%} & \cellcolor[HTML]{C0C0C0}72,96\% \\ \hline
99,99\% & 99,991\% & 299 & 402 & \textbf{701} & \textbf{103} & \textbf{14\%} & 72,69\% \\ \hline
\cellcolor[HTML]{C0C0C0}99,99\% & \cellcolor[HTML]{C0C0C0}99,990\% & \cellcolor[HTML]{C0C0C0}346 & \cellcolor[HTML]{C0C0C0}455 & \textbf{801} & \textbf{109} & \textbf{13\%} & \cellcolor[HTML]{C0C0C0}72,69\% \\ \hline
99,99\% & 99,991\% & 392 & 509 & \textbf{901} & \textbf{117} & \textbf{12\%} & 72,86\% \\ \hline
\cellcolor[HTML]{C0C0C0}99,99\% & \cellcolor[HTML]{C0C0C0}99,991\% & \cellcolor[HTML]{C0C0C0}439 & \cellcolor[HTML]{C0C0C0}562 & \textbf{1001} & \textbf{123} & \textbf{12\%} & \cellcolor[HTML]{C0C0C0}73,16\% \\ \hline
99,99\% & 99,991\% & 486 & 615 & \textbf{1101} & \textbf{129} & \textbf{11\%} & 73,54\% \\ \hline
\cellcolor[HTML]{C0C0C0}99,99\% & \cellcolor[HTML]{C0C0C0}99,991\% & \cellcolor[HTML]{C0C0C0}533 & \cellcolor[HTML]{C0C0C0}668 & \textbf{1201} & \textbf{135} & \textbf{11\%} & \cellcolor[HTML]{C0C0C0}71,45\% \\ \hline
99,99\% & 99,991\% & 580 & 721 & \textbf{1301} & \textbf{141} & \textbf{10\%} & 72,06\% \\ \hline
\cellcolor[HTML]{C0C0C0}99,99\% & \cellcolor[HTML]{C0C0C0}99,990\% & \cellcolor[HTML]{C0C0C0}628 & \cellcolor[HTML]{C0C0C0}773 & \textbf{1401} & \textbf{145} & \textbf{10\%} & \cellcolor[HTML]{C0C0C0}72,68\% \\ \hline
99,99\% & 99,991\% & 675 & 826 & \textbf{1501} & \textbf{151} & \textbf{10\%} & 73,31\% \\ \hline
\cellcolor[HTML]{C0C0C0}99,99\% & \cellcolor[HTML]{C0C0C0}99,990\% & \cellcolor[HTML]{C0C0C0}723 & \cellcolor[HTML]{C0C0C0}878 & \textbf{1601} & \textbf{155} & \textbf{9\%} & \cellcolor[HTML]{C0C0C0}71,76\% \\ \hline
99,99\% & 99,991\% & 770 & 931 & \textbf{1701} & \textbf{161} & \textbf{9\%} & 72,49\% \\ \hline
\cellcolor[HTML]{C0C0C0}99,99\% & \cellcolor[HTML]{C0C0C0}99,990\% & \cellcolor[HTML]{C0C0C0}818 & \cellcolor[HTML]{C0C0C0}983 & \textbf{1801} & \textbf{165} & \textbf{9\%} & \cellcolor[HTML]{C0C0C0}73,20\% \\ \hline
99,99\% & 99,990\% & 866 & 1035 & \textbf{1901} & \textbf{169} & \textbf{8\%} & 71,90\% \\ \hline
\cellcolor[HTML]{C0C0C0}99,99\% & \cellcolor[HTML]{C0C0C0}99,990\% & \cellcolor[HTML]{C0C0C0}914 & \cellcolor[HTML]{C0C0C0}1087 & \textbf{2001} & \textbf{173} & \textbf{8\%} & \cellcolor[HTML]{C0C0C0}72,67\% \\ \hline
99,99\% & 99,990\% & 1394 & 1607 & \textbf{3001} & \textbf{213} & \textbf{7\%} & 71,86\% \\ \hline
\cellcolor[HTML]{C0C0C0}99,99\% & \cellcolor[HTML]{C0C0C0}99,991\% & \cellcolor[HTML]{C0C0C0}1877 & \cellcolor[HTML]{C0C0C0}2124 & \textbf{4001} & \textbf{247} & \textbf{6\%} & \cellcolor[HTML]{C0C0C0}72,47\% \\ \hline
99,99\% & 99,990\% & 2363 & 2638 & \textbf{5001} & \textbf{275} & \textbf{5\%} & 72,38\% \\ \hline
\cellcolor[HTML]{C0C0C0}99,99\% & \cellcolor[HTML]{C0C0C0}99,990\% & \cellcolor[HTML]{C0C0C0}2850 & \cellcolor[HTML]{C0C0C0}3151 & \textbf{6001} & \textbf{301} & \textbf{5\%} & \cellcolor[HTML]{C0C0C0}72,75\% \\ \hline
99,99\% & 99,990\% & 3338 & 3663 & \textbf{7001} & \textbf{325} & \textbf{4\%} & 72,32\% \\ \hline
\cellcolor[HTML]{C0C0C0}99,99\% & \cellcolor[HTML]{C0C0C0}99,990\% & \cellcolor[HTML]{C0C0C0}3827 & \cellcolor[HTML]{C0C0C0}4174 & \textbf{8001} & \textbf{347} & \textbf{4\%} & \cellcolor[HTML]{C0C0C0}72,18\% \\ \hline
99,99\% & 99,990\% & 4316 & 4685 & \textbf{9001} & \textbf{369} & \textbf{4\%} & 72,23\% \\ \hline
\cellcolor[HTML]{C0C0C0}99,99\% & \cellcolor[HTML]{C0C0C0}99,990\% & \cellcolor[HTML]{C0C0C0}4806 & \cellcolor[HTML]{C0C0C0}5195 & \textbf{10001} & \textbf{389} & \textbf{3\%} & \cellcolor[HTML]{C0C0C0}72,42\% \\ \hline
\end{tabular}%
}
\end{table}
\footnotetext{O Apêndice \ref{app:algoritmos} é dedicado a clarificar o algoritmo BinomialDistribuion\_PROB e validar o fórmula da probabilidade binomial geral usada por ele.}
\vspace{-8mm}
\begin{description}
   \item[Meta] Porcentagem das amostras observadas;
   \item[Soma do Range] Porcentagem que o \textbf{"Range"} atingiu a \textbf{"Meta"}, da mediana para as bordas, descentralizado;Porcentagem que o Range atingiu a Meta, da mediana para as bordas, descentralizado;
   \item[Range] Range de amostras onde a \textbf{"Meta"} foi atingida do \textbf{"Total de Amostras"};
   \item[Total de Amostras] Exibe o range total avaliado, no caso da primeira linha da tabela o valor 101 corresponde às possibilidades de 0 a 100, como se fossem lançadas 100 moedas (distribuição binomial) e somassem suas faces voltadas para cima, podendo ser 0 para as caras e 1 para as coroas. Essa soma é uma combinação de possibilidades não uma permutação, ou seja, na permutação [0 1] é uma possibilidade diferente de [1 0], na combinação essa é uma possiblidade, porém com duas probabilidades de ocorrência;
   \item[Amostras do Range] Quantidade de amostras do \textbf{"Range"} do \textbf{"Total de Amostras"};
   \item[Porcentagem das Amostras do Range] Porcentagem que o \textbf{"Range} representa do \textbf{"Total de Amostras"};
   \item[Range de +/- 14\% (28\%) da Mediana] Esse range é  subconjunto do \textbf{"Range"}, formado a partir da mediana somando 14\% a direita e a esquerda, totalizando 28\%. Esses 28\% correspondem a aproximadamente 72\% das amostras da população do Range, que correspondem a 99,99\% da população total. O restante, que representam 72\% \textbf{"Range"}, correspondem a aproximadamente 28\% das amostras. Isso condiz com o Princípio de Pareto também conhecido como a regra do 80/20, que também pode ser 70/30 ou 90/10, por exemplo \cite{administradores_principio_pareto}.
\end{description}
\bigbreak


Pode-se observar que a medida as amostras aumentam, a porcentagem ocupada por 99,99\% das amostras tende a diminuir \textbf{"\% das Amostras do Range"}, ainda que a quantidade dessas amostras que representam essa porcentagem tenda a aumentar \textbf{"Amostras do Range"}.

A coluna de "Amostras do Range", setas azuis no gráfico da Figura \ref{fig:total_comparison_chart_with_99_range} vão no sentido ao centro do gráfico, ou seja, apesar de aumentar a quantidade de amostras onde o range das 99,99\% das probabilidades se encontram, a proporção que essas amostras assumem no "Total de Amostras" diminuem. As setas em roxo do gráfico representam a distribuição da coluna "Total de Amostras" da Tabela \ref{tab:10000_all}. Conforme os momentos lógicos aumentam mais próximos da mediana os 99,99\% de suas amostras estarão e mais irrelevantes se tornam os intervalos lógicos mais afastados do centro, os que não fazem mais parte dos 99,99\%.

\begin{figure}[H]
\caption{Comparação do total de amostrar com 99,99\% }
\label{fig:total_comparison_chart_with_99_range}
\centering
\includegraphics[scale=1]{sections/images/total_comparison_chart_with_99_range.jpg}
\floatfoot{As setas em roxo representam o total das amostras e as em azul os 99,99\% \footnotemark.}
\end{figure}
\footnotetext{O gráfico da Figura \ref{fig:total_comparison_chart_with_99_range} representa as 20 primeiras linhas da Tabela \ref{tab:10000_all}, pois sofrem incrementos iguais, de 100 amostras, em cada linha. A linha 21 em diante sofrem incremento de 1000 amostras a cada linha.}

No endereço \url{https://www.mathsisfun.com/data/quincunx.html} existe uma ferramenta chamada Quincunx ou Galton Board que exemplifica dinamicamente o que as figuras acima mostram. Uma explicação sobre o funcionamento dessa ferramenta pode ser vista em \url{https://www.mathsisfun.com/data/quincunx-explained.html}. 