% ----------------------------------------------------------
% ELEMENTOS PRÉ-TEXTUAIS
% ----------------------------------------------------------
% página de titulo
\setlength{\droptitle}{-50pt}
\maketitle
%\vspace{-10mm}
% resumo em português
%\begin{resumoumacoluna}
%	\vspace{-2mm}
%	Neste artigo pretende-se introduzir uma teoria a respeito da origem de tudo. O objetivo inicial é responder se existe algo ao invés de nada. Essa pergunta vem incomodando a filosofia e a ciência até os dias de hoje. A resposta a essa pergunta está na compreensão de que a lógica em sua essência remete ao nada (NÃO É - NEGA A SI - NEGA SER). A negação de si, essência lógica, gera expansões lógicas que caracterizam os fundamentos do teorema central do limite. Os passos da expansão lógica regidos pela probabilidade descrita no teorema central do limite correspondem à consciência, a maior onda lógica de uma população e seus aspectos: infinito, ondas, tempo, espaço, forças fundamentais, matéria escura, energia escura, antimatéria e buraco negro. Em outras palavras, a infinita negação de si da lógica gera expansões lógicas que probabilisticamente irão formar ondas lógicas e suas sub-ondas, estabelecendo qual é a natureza fundamental da realidade, do conhecimento e da existência. As expansões lógicas acontecem na ausência de tempo, o que define a essência lógica como uma infinita recursão generalizada, uma constante, análogo aos infinitos números ou pontos que compõem o intervalo de uma reta qualquer.
%	% \vspace{\onelineskip} 
% 	
% 	\noindent
%	\textbf{Palavras-chaves}: lógica. nada. tudo. expansão lógica. teorema central do limite. consciência. infinito. ondas. tempo. espaço. forças fundamentais. matéria escura. energia escura. antimatéria. buraco negro. observador e a vida.
%\end{resumoumacoluna}

% resumo em inglês
\vspace{-5mm}
\renewcommand{\resumoname}{Abstract}
\begin{resumoumacoluna}
 \begin{otherlanguage*}{english}
	\vspace{-2mm}
	This article aims to introduce the theory about the origin of everything. The initial goal is to answer if there is something instead of nothing. This question has been bothering philosophy and science to this day. The answer to this question lies in the understanding that logic in its essence refers to nothing (NOT TO BE - SELF-NEGATION - DENIES BEING). Self-negation, the logical essence, generates logical expansions that characterize the foundations of the central limit theorem. The steps of the logical expansion governed by the probability described in the central limit theorem corresponds to consciousness, the largest logical wave of a population and its aspects: infinity, waves, time, space, fundamental forces, dark matter, dark energy, antimatter and black hole. In other words, the infinite self-negation of logic generates logical expansions that will probabilistically form logical waves and their subwaves, establishing what is the fundamental nature of reality, knowledge and existence. Logical expansions happen in the absence of time, which defines the logical essence as an infinite generalized recursion, a constant, analogous to the infinite numbers or points that make up the interval of any given line.
	\vspace{\onelineskip}  
	
	\noindent
	\textbf{Keywords}: logic. nothing. everything. logical expansion. central limit theorem. consciousness. infinite. waves. time. space. fundamental forces. dark matter. dark energy. antimatter. black hole. observer and life.
 \end{otherlanguage*}  
\end{resumoumacoluna}
