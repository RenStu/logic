% ----------------------------------------------------------
% ELEMENTOS PRÉ-TEXTUAIS
% ----------------------------------------------------------
% página de titulo
\vspace{-15mm}
\maketitle
\vspace{-8mm}
% resumo em português
\begin{resumoumacoluna}
\vspace{-2mm}
	Neste artigo pretende-se introduzir uma teoria a respeito da origem de tudo. O objetivo inicial é responder se existe algo ao invés de nada. Essa pergunta vem incomodando a filosofia e a ciência até os dias de hoje. A resposta a essa pergunta está na compreensão de que a lógica em sua essência remete ao nada, que NÃO É, ou seja, nega a si mesmo (nega ser). A negação de si gera expansões binomiais, no qual suas amostras combinadas em cada passo dessa expansão se aproximam da distribuição normal e se aproximam do centro dessa distribuição infinitamente, o que configura o teorema central do limite. Os passos da expansão binomial regidos pela probabilidade descrita no teorema central do limite compreendem a consciência e tornam visíveis o que é e o porquê de seus aspectos mais perceptíveis: infinito, tempo, espaço, gravidade, matéria escura, energia escura, antimatéria e buraco negro. Como essência da expansão binomial, do teorema central do limite e da consciência e seus aspectos tem-se a lógica em sua dualidade, que por um lado NÃO É e por outro É ilógica, imutável e inexistente, uma vez que a existência está em tudo aquilo que NÃO É. 
 \vspace{\onelineskip} 
 \noindent
 
 \textbf{Palavras-chaves}: lógica. nada. tudo. binômio. expansão binomial. teorema central do limite. consciência. infinito. ondas. tempo. espaço. forças fundamentais. gravidade. matéria escura. energia escura. antimatéria. buraco negro.
\end{resumoumacoluna}

% resumo em inglês
\renewcommand{\resumoname}{Abstract}
\begin{resumoumacoluna}
 \begin{otherlanguage*}{english}
\vspace{-2mm}
	This article aims to introduce the theory about the origin of everything. The initial goal is to answer if there is something instead of nothing. This question has been bothering philosophy and science to this day. The answer to this question lies in understanding that logic in its essence refers to nothing, which IS NOT, that is, denies itself (denies being). Self-denial generates binomial expansions, in which their combined samples at each step of this expansion approach the normal distribution and approach the center of this distribution infinitely, which configures the central limit theorem. The steps of binomial expansion governed by the probability described in the central limit theorem comprise consciousness and make visible what it is and why its most noticeable aspects are: infinity, time, space, gravity, dark matter, dark energy, antimatter and black hole. The essence of binomial expansion, the central limit theorem and consciousness and its aspects is logic in its duality, which on the one hand IS NOT and on the other hand IS illogical, unchanging and non-existent, since existence is in all that IS NOT.
	\vspace{\onelineskip} 
	\noindent
	
	\textbf{Keywords}: logic. nothing. all. binomial. binomial expansion. central limit theorem. consciousness. infinite. waves. time. space. fundamental forces. gravity. dark matter. dark energy. antimatter. black hole.
 \end{otherlanguage*}  
\end{resumoumacoluna}

%\begin{center}\smaller
%\textbf{Identificação e disponibilidade}: elemento opcional. Pode ser indicado 
%o endereço eletrônico, DOI, suportes e outras informações relativas ao acesso.
%\end{center}