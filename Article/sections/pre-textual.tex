% ----------------------------------------------------------
% ELEMENTOS PRÉ-TEXTUAIS
% ----------------------------------------------------------
% página de titulo
\setlength{\droptitle}{-50pt}
\maketitle
%\vspace{-10mm}
% resumo em português
%\begin{resumoumacoluna}
%	\vspace{-2mm}
%	Neste artigo pretende-se introduzir uma teoria a respeito da origem de tudo. O objetivo inicial é responder se existe algo ao invés de nada. Essa pergunta vem incomodando a filosofia e a ciência até os dias de hoje. A resposta a essa pergunta está na compreensão de que a lógica em sua essência remete ao nada (NÃO É - NEGA A SI - NEGA SER). A negação de si, essência lógica, gera expansões lógicas que caracterizam os fundamentos do teorema central do limite. Os passos da expansão lógica regidos pela probabilidade descrita no teorema central do limite correspondem à consciência, a maior onda lógica de uma população e seus aspectos: infinito, ondas, tempo, espaço, forças fundamentais, matéria escura, energia escura, antimatéria e buraco negro. Em outras palavras, a infinita negação de si da lógica gera expansões lógicas que probabilisticamente irão formar ondas lógicas e suas sub-ondas, estabelecendo qual é a natureza fundamental da realidade, do conhecimento e da existência. As expansões lógicas acontecem na ausência de tempo, o que define a essência lógica como uma infinita recursão generalizada, uma constante, análogo aos infinitos números ou pontos que compõem o intervalo de uma reta qualquer.
%	% \vspace{\onelineskip} 
% 	
% 	\noindent
%	\textbf{Palavras-chaves}: lógica. nada. tudo. expansão lógica. teorema central do limite. consciência. infinito. ondas. tempo. espaço. forças fundamentais. matéria escura. energia escura. antimatéria. buraco negro. observador e a vida.
%\end{resumoumacoluna}

% resumo em inglês
\vspace{-5mm}
\renewcommand{\resumoname}{Abstract}
\begin{resumoumacoluna}
 \begin{otherlanguage*}{english}
	\vspace{-2mm}
		In this article, the intention is to introduce a theory about the origin of everything. The initial goal is to answer whether there is something rather than nothing. This question has preoccupied philosophy and science to this day. The answer to this question lies in the understanding that logic in its essence refers to nothingness (NOT IS - NEGATES ITSELF - SELF-NEGATION - NEGATES BEING). The self-denial of nothingness (the primordial logic) generates logical expansions that characterize the foundations of the central limit theorem. These logical expansions characterize the foundations of the central limit theorem.  The steps of the logical expansion governed by probability described in the central limit theorem correspond to consciousness, the largest logical wave in a population, and its aspects: infinity, waves, time, space, fundamental forces, dark matter, dark energy, antimatter, and black hole. In other words, the infinite negation of logic (self-denial of nothingness) generates logical expansions that probabilistically will form logical waves and their sub-waves, establishing what is the fundamental nature of reality, knowledge, and existence. Logical expansions occur in the absence of time, which defines the logical essence as a generalized infinite recurrence, a constant, analogous to the infinite numbers or points that make up the interval of any given line.
	\vspace{\onelineskip}  
	
	\noindent
	\textbf{Keywords}: logic. nothing. everything. logical expansion. central limit theorem. consciousness. infinite. waves. time. space. fundamental forces. dark matter. dark energy. antimatter. black hole. observer and life.
 \end{otherlanguage*}  
\end{resumoumacoluna}
