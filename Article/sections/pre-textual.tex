% ----------------------------------------------------------
% ELEMENTOS PRÉ-TEXTUAIS
% ----------------------------------------------------------
% página de titulo
\maketitle

% resumo em português
\begin{resumoumacoluna}
	Neste artigo pretende-se introduzir uma teoria a respeito da origem de tudo. O objetivo inicial é responder se existe algo ao invés de nada. Essa pergunta vem incomodando a filosofia e a ciência até os dias de hoje. A resposta a essa pergunta está na compreensão de que a lógica em sua essência remete ao nada, que NÃO É, ou seja, nega a si mesmo ou nega ser. A negação de si gera expansões binomiais, no qual cada passo dessas expansões é formado por uma negação de si mais suas subnegações que unificadas nega ser, o que configura o teorema central do limite. Em vista disso, este estudo está centrado na expansão binomial e na distribuição aleatória de amostras que combinadas em cada passo dessa expansão se aproximam da distribuição normal e se aproximam do centro dessa distribuição infinitamente. Os passos da expansão binomial regidos pelo teorema central do limite compreendem a consciência e torna visível o que é e o porquê de seus aspectos mais perceptíveis: infinito, tempo, espaço, gravidade, matéria escura, energia escura e buraco negro.
 
 \vspace{\onelineskip}
 
 \noindent
 \textbf{Palavras-chaves}: lógica. nada. tudo. binômio. expansão binomial. teorema central do limite. consciência. infinito. tempo. espaço. gravidade. matéria escura. energia escura. buraco negro.
\end{resumoumacoluna}


% resumo em inglês
\renewcommand{\resumoname}{Abstract}
\begin{resumoumacoluna}
 \begin{otherlanguage*}{english}
	This article aims to introduce the theory about the origin of everything. The initial goal is to answer if there is something instead of nothing. This question has been bothering philosophy and science to this day. The answer to this question lies in understanding that logic in its essence refers to nothing, which IS NOT, that is, denies itself or denies being. Self-denial generates binomial expansions, in which each step of these expansions is formed by self-negation plus its subnegations that unify deny being, which configures the central limit theorem. Therefore, this study focuses on binomial expansion and random distribution of samples that combined at each step of this expansion approach the normal distribution and approach the center of this distribution infinitely. The steps of binomial expansion governed by the central limit theorem comprise consciousness and make visible what it is and why it’s most noticeable aspects are: infinity, time, space, gravity, dark matter, dark energy, and black hole. 

	\vspace{\onelineskip}
 
	\noindent
	\textbf{Keywords}: 
logic. nothing. all. binomial. binomial expansion. central limit theorem. consciousness. infinite. time. space. gravity. dark matter. dark energy. black hole.
 \end{otherlanguage*}  
\end{resumoumacoluna}

%\begin{center}\smaller
%\textbf{Identificação e disponibilidade}: elemento opcional. Pode ser indicado 
%o endereço eletrônico, DOI, suportes e outras informações relativas ao acesso.
%\end{center}