% ----------------------------------------------------------
% Seção Lógica - Principal
% ----------------------------------------------------------
\section{Lógica}
Segundo o dicionário online de Português Dicio\cite{dicio_logica}, a palavra lógica se refere a:
\begin{enumerate}
   \item Modo de raciocinar coerente que \underline{expressa uma relação de causa e consequência};
   \item Maneira coerente através da qual os \underline{fatos ou situações se encadeiam}. 
\end{enumerate}
 
\bigbreak
A palavra lógica expressa uma relação de causa e consequência ou fatos encadeados. Pode-se distinguir como essência dessas duas definições o movimento, a mudança, a transição. A palavra lógica, em sua essência, se encaixa perfeitamente na definição do NADA − NÃO SER.  A lógica NÃO SER é consonante com o NADA, pois se a lógica NÃO É ela também É seu contrário, ou seja, ilógica e imutável. Nessa dualidade, tem-se a existência fundamentada pela lógica que "nega a si", enquanto, por outro lado É ilógica, imutável e inexistente. A expressão "negação de si" refere-se à negação do SER - NÃO SER. 

A lógica está centrada na mudança e a mudança está centrada naquilo que NÃO É, uma vez que aquilo que É não pode deixar de SER. A mudança demanda que, em algum momento, algo DEIXE DE SER o que fora a se transformar. Em \citeonline{brasilescola_parmenides}, Parmênides  o filósofo da unidade e da identidade do SER, diz que a contínua mudança é a principal característica do não ser. Para Parmênides o SER é uno, eterno, não gerado e imutável.

A lógica SER ilógica não a impede de NÃO SER. Na dualidade SER e NÃO SER, o SER limita e define, o NÃO SER \textit{ad infinitum}.
\begin{figure}[H]
\caption{Analogia da lógica primordial}
\label{fig:primordial_logic_representation}
\centering
\includegraphics[scale=1]{sections/images/primordial_logic_representation.jpg}
\floatfoot{Reta utilizada para representar e validar o conceito da lógica primordial.}%\footnotemark}
\end{figure}
%\footnotetext{Fonte: note}

Com base na Figura \ref{fig:primordial_logic_representation} pode-se extrair as seguintes observações (axiomas) em relação aos pontos \textbf{0}, \textbf{1} e o \textbf{intervalo} entre eles:
\begin{description}
   \item[Ponto 1 - {[1,1]}] É ilógico, pois é a totalidade não fracionada da reta, neste caso a premissa primordial da lógica (NÃO SER) não foi atendida..
   \item[Ponto 0 - {[0,0]}] É ilógico, pois é um ponto nulo incapaz de negar a si, dado que toda lógica ou sub-lógica (fração lógica) deve se manter negando a si, uma vez que essa é a premissa primordial da lógica. A lógica NÃO É em sua essência, primordialmente.
   \item[Intervalo - $\textbf{{[0,1[ x ]0,1]}}$] A lógica é possível apenas na representação das frações ou intervalos dos pontos \textbf{0} e \textbf{1}. Uma fração da reta nega ser a reta, pois é apenas uma parte dela. Os subintervalos, do mesmo modo, também são hábeis a negar a si infinitamente, garantindo a premissa primordial da lógica (negação de si) em todo o intervalo e seus subintervalos. 
\end{description}

Se os três pontos acima puderem ser considerados axiomas da lógica primordial, a essência desse estudo, provavelmente essa teoria não seja recursivamente enumerável, uma vez que o ponto inicial, os intermediários e o final representados na reta são consonantes com os números reais \cite{smb_numeros_reais}.
\begin{figure}[H]
\caption{Primeiro momento lógico}
\label{fig:first_logical_moment}
\centering
\includegraphics[scale=1]{sections/images/first_logical_moment.jpg}
\floatfoot{Reta fracionada em dois intervalos representando o primeiro momento lógico.}%\footnotemark}
\end{figure}
%\footnotetext{Fonte: note}

Na Figura \ref{fig:first_logical_moment} a união do traço à reta é a representação de uma divisão lógica, pois é da negação da lógica em SER que surgi esses dois intervalos lógicos ou duas sub-lógicas. O segmento em azul representa a negação da lógica em SER o todo ilógico nesse primeiro momento. As duas frações geradas pela negação lógica negam SER a reta, pois são apenas intervalos delas e são capazes de negar a si infinitamente, garantindo a premissa primordial da lógica NÃO SER. 

% ----------------------------------------------------------
% Subseção Expansão lógica
% ----------------------------------------------------------
\subsection{Expansão lógica}
A lógica primordial (negação de si) cria expansões lógicas infinitas. Uma expansão lógica é análoga a um universo. O primeiro momento lógico é o início de uma dessas expansões, porém existem infinitas possibilidades de negação do primeiro momento lógico, o que revela a possibilidade das infinitas expansões binomiais, pois o SER é ilógico e imutável, portanto pode ser negado infinitamente. A negação do SER não transforma SER em NÃO SER, pois este é imutável (o NÃO SER é apenas o outro lado do SER). 
\begin{figure}[H]
\caption{Momentos lógicos iniciais}
\label{fig:third_logical_moment}
\centering
\includegraphics[scale=.85]{sections/images/third_logical_moment.jpg}
\floatfoot{Exemplo dos três primeiros momentos de uma expansão.}%\footnotemark}
\end{figure}
%\footnotetext{Fonte: note}

Com base na Figura \ref{fig:third_logical_moment} pode-se extrair as seguintes observações em relação ao primeiro, segundo e terceiro momentos lógicos:
\begin{description}
   \item[Primeiro momento lógico] A negação da lógica primordial a si, a subdivide em duas unidades, que somadas são o todo ilógico. Apesar dessas partes terem proporções diferentes, elas exprimem as mesmas quantidades de pontos ou possibilidades de mudança, uma vez que são representações da lógica primordial, que \textit{ad infinitum}. A parte fracionada em azul representa a proporção da negação lógica em relação à sua unidade.
   \item[Segundo momento lógico] É gerado pela negação das duas sub-lógicas primordiais, fracionadas no primeiro momento lógico. Na impossibilidade dessas frações lógicas do primeiro momento lógico continuar negando a si, faria com que elas fossem incapazes de negar suas unidades que formam o todo, ou seja, seriam incapazes de negar suas duas unidades e por consequência o todo que é formado precisamente por elas, o que faria da lógica apenas ilógica (SER), uma unicidade. As partes fracionadas em azul representam a proporção da negação lógica em relação às suas respectivas unidades.
   \item[Terceiro momento lógico] Decorre da negação do segundo momento lógico, assim como o segundo momento lógico decorre da negação do primeiro e assim por diante.
\end{description}

A cada negação ou subnegação da lógica primordial, seus novos valores são influenciados pelos valores adjacentes do momento lógico anterior. Na figura \ref{fig:imposition_of_binomial_expansion}, a lógica primordial nega a si gerando o primeiro momento lógico com o valor [0,2].  No segundo momento lógico, suas subdivisões estão contidas no limite imposto pelo valor do primeiro momento lógico. Os pontos do terceiro momento lógico, por exemplo, sofrem as imposições dos valores do segundo momento lógico que por sua vez sofrem a imposição do primeiro. Os valores de momentos lógicos descendentes sofrem imposições acumulativas dos valores dos momentos lógicos anteriores. À imposição de um valor em seus dois valores imediatamente descendentes denominou-se sincronismo lógico. Isso é o que pode ser visto no triângulo de pascal. Esse sincronismo irá levar à frequências de amostras cada vez maiores em intervalos cada vez menores, que serão vistos na próxima seção do Teorema central do limite.

\begin{figure}[H]
\caption{Imposição da expansão lógica}
\label{fig:imposition_of_binomial_expansion}
\centering
\includegraphics[scale=.85]{sections/images/imposition_of_binomial_expansion.jpg}
\floatfoot{Imposição acumulativa aos momentos lógicos descendentes.}%\footnotemark}
\end{figure}
%\footnotetext{Fonte: note}

No triângulo de pascal, Figura \ref{fig:pascal_triangle}, cada número é os dois números acima mais próximos somados. Esse número representa quantos diferentes possíveis caminhos levam até ele. Por exemplo, o número [4], na Figura \ref{fig:pascal_triangle}, representa os quatro diferentes caminhos que levam até ele. Os coeficientes binômias encontrados no triangulo de Pascal representam apenas as quantidades de imposições sofridas por cada valor de um momento lógico. Um outro aspecto interessante do triângulo de pascal é a sequência de Fibonacci, Figura \ref{fig:pascal_triangle_fibonacci} \cite{mathisfun_pascal_triangle}.  

\begin{figure}[H]
\centering
	\begin{subfigure}[H]{0.47\linewidth}
	\centering
	\includegraphics[width=.55\linewidth]{sections/images/pascal_triangle.jpg}
	\caption{}
	\label{fig:pascal_triangle}
	\end{subfigure}
\hfill
	\begin{subfigure}[H]{0.47\linewidth}
	\centering
	\includegraphics[width=.9\linewidth]{sections/images/pascal_triangle_fibonacci.jpg}
	\caption{}
	\label{fig:pascal_triangle_fibonacci}
	\end{subfigure}%
\caption{Características do triângulo de Pascal}

\floatfoot{Fonte: MathsIsFun, 2019.\protect\footnotemark}
\end{figure}
\footnotetext{\url{www.mathsisfun.com/pascals-triangle.html}}

O NÃO SER da lógica primordial é análogo a uma constante abstrata, ou seja, suas infinitas negações e subnegações transcendem o tempo. Todas essas infinitas negações acontecem no tempo zero. A incapacidade da lógica negar a si por um intervalo entre suas negações faria a lógica SER ilógica nesse intervalo, por menor que este seja, o que quebraria a premissa primordial da lógica, NÃO SER. Em outras palavras, é como se fosse “todos vezes todos”, ou seja, não é preciso esperar o negação do primeiro momento lógico, pois todos as subnegações do segundo momento lógico são possíveis para todos as negações do primeiro momento lógico e assim por diante. A lógica é como um algoritmo composto de apenas uma constante auto executada, uma sequência simultânea. É a consciência que conduz a experiência do tempo, não pela criação da sequência de mudanças que é simultânea, mas sim pela ordem dessa sequência, que nada mais é que do que a observação da ordem das mudanças de cada momento lógico.

Algumas respostas podem ajudar a esclarecer o que é essa sequência simultânea:
\begin{description}
   \item[Todas as negações acontecem simultaneamente?] Sim, infinitas negações na ausência de tempo, ou tempo zero.
   \item[Como ou porque essa simultaneidade acontece?] Acontecem em uma sequência de negações da lógica a si mesma, no tempo zero, onde em nenhum momento a lógica converte-se em SER, garantindo assim a premissa primordial da constante lógica, NÃO SER.
   \item[O que é uma sequência simultânea?] É a negação da lógica a si (uma sequência ou ordem) no tempo zero, ou seja, em nenhum momento a lógica passa a SER durante as infinitas negações (simultaneidade). Sequência simultânea é o sinônimo da constante lógica NÃO SER.
\end{description}

% ----------------------------------------------------------
% Central limit theorem subsection
% ----------------------------------------------------------
\subsection{Central limit theorem}
Based on the axioms seen in Figure \ref{fig:primordial_logic_representation}, the following theorem is discriminated: If the parts of the subintervals are subparts of the entire interval, then these subparts summed are part of the entire interval.

Thus, in Figure \ref{fig:second_logical_moment}, the negation of the first logical moment negates [being], while the subnegations of the other logical moments are subparts that subnegate [being], so these subparts only negate [being] when added together or unified according to the first logical moment.
	\begin{figure}[H]
	\caption{Subdivided logical moments}
	\label{fig:second_logical_moment}
	\centering
	\includegraphics[scale=.8]{sections/images/second_logical_moment.jpg}
	\floatfoot{Example of the first two moments of an expansion.}%\footnotemark}
	\end{figure}
	%\footnotetext{Fonte: note}

In Figure \ref{fig:logical_units} the representation of the first and second logical moments from Figure \ref{fig:second_logical_moment} can be seen as logical units.
	\begin{figure}[H]
	\caption{Unified logical moments}
	\label{fig:logical_units}
	\centering
	\includegraphics[scale=.8]{sections/images/logical_units.jpg}
	\floatfoot{Example of the first two unified moments of an expansion.}%\footnotemark}
	\end{figure}
	%\footnotetext{Fonte: note}

The dynamics of the theorem described above and its essential axioms of logic are cognitively observable through the mathematical construction of the natural numbers, readjusting the scale of the symbols representing each logical moment as needed by the logical expansion. Mathematics supports the addition operation, necessary in the representation of the above theorem, with Presburguer's arithmetic, which is consistent, complete, and decidable \cite{wiki_arithmetic_presburger}.

The theorem and its essential axioms of logic can also be cognitively observable by the mathematical construction of positive real numbers (represented without operations such as fractions, roots and others - the finite decimals), which is supported by the mathematical theory of the ordered field - a subset of the real numbers greater than or equal to zero and closed for the sum and product operations. The product operation and its properties are not necessary for the dynamics of the theorem and its essential axioms of logic \cite{wiki_ordered_field}. The ordered field mathematical theory is a first-order mathematical theory, with all its axioms described by first-order logic, making it complete and decidable \cite{wiki_RealClosedField}.

It is important to note that logic in its essence is not subject to mathematics, but all mathematics is restricted to logic, and therefore some of its simplest constructions may come closer to essential logic than others.

The unity present in the negation (first logical moment) and in the logical subnegations (other logical moments) is the characteristic that corresponds to the central axis of the central limit theorem. This theorem states that the sample distribution of a population approaches a normal distribution as the sample sizes increase, regardless of the shape of the population distribution. This is especially true for sample sizes greater than 30. A simple test that demonstrates this fact is the rolling of unbiased dice. The higher the dice roll number, the more likely the graph will look like the normal distribution graph \cite{statisticshowto_central_limit_theorem}. The appendix \ref{app:algorithms} explains the Distribution\_PROB algorithm in order to clarify the probabilistic essence of the central limit theorem.

It is important to note, as shown in Figure \ref{fig:trend_chart_of_normal_distribution}, that the probabilistic balance or synchronism to the right and left of the median, caused by the distribution of unified logical moments, can illustrate the doctrine of opposites of Heraclitus of Ephesus \cite{heraclitus}.
	\begin{figure}[H]
	\caption{Probabilistic synchronism of the opposite samples with respect to the median}
	\label{fig:trend_chart_of_normal_distribution}
	\centering
	\includegraphics[scale=1]{sections/images/trend_chart_of_normal_distribution.jpg}
	\floatfoot{Example of a distribution that approximates the normal distribution.}%\footnotemark}
	\end{figure}
	%\footnotetext{Fonte: note}

In the table \ref{tab:10000_all} is the probability of the binomial distribution between 100 and 10000 samples, in line with the unified samples, Figure \ref{fig:logical_units}, or sample averages treated in the central limit theorem.

The binomial distribution behaves like the tossing of coins (heads or tails), in the case of the first row of the table, distribution of 100 samples, there are 101 possibilities, from 0 to 100, as if 100 coins were tossed adding their sides up, which can be 0 for heads and 1 for tails, for example. So if all 100 coins tossed come out heads, the sum is 0, and if they all come out tails, the sum is 100. This sum is a combination of possibilities, not a permutation, that is, in permutation [0, 1] is a possibility other than [1, 0], in combination it is 1 possibility, but with 2 probabilities of occurrence. Therefore, the sum corresponding to 100\% of the heads or 100\% of the tails corresponds to 1 possibility each, while the other sums have a higher possibility of occurrence. For this first row of the table, 100 coins, 99.994\% of all possibilities sum between 31 and 70. 

The construction of this table was performed with the general binomial probability formula (which represents a uniform distribution) using the algorithm \tiny BinomialDistribuion\_PROB \normalsize explained in Appendix \ref{app:algorithms} \cite{mathisfun_binomial_distribution}.
	\begin{align*}
	f(k;n,p) &= \binom{n}{k} p^k(1 - p)^{n-k}
	\end{align*}
The binomial distribution was used in this section of the study, but other discrete distributions could be used, such as the unbiased dice roll, and the observations in this study would remain the same because the central limit theorem is independent of the shape of the population distribution \cite{statisticsbyjim_central_limit_theorem_explainded}.
\begin{table}[H]
\caption{Probabilidade da distribuição binomial}
\floatfoot{Tabela gerada pelo algoritmo BinomialDistribuion\_PROB com a distribuição binomial de 100 a 10000. \footnotemark}
\label{tab:10000_all}
\resizebox{\textwidth}{!}{%
\begin{tabular}{cccc
>{\columncolor[HTML]{8D3CE1}}c 
>{\columncolor[HTML]{5754D6}}c 
>{\columncolor[HTML]{8FFFFB}}c c}
\hline
\cellcolor[HTML]{95ABEC} & \cellcolor[HTML]{95ABEC} & \multicolumn{2}{c}{\cellcolor[HTML]{95ABEC}} & \cellcolor[HTML]{95ABEC} & \cellcolor[HTML]{95ABEC} & \cellcolor[HTML]{95ABEC} & \cellcolor[HTML]{95ABEC} \\
\cellcolor[HTML]{95ABEC} & \cellcolor[HTML]{95ABEC} & \multicolumn{2}{c}{\cellcolor[HTML]{95ABEC}} & \cellcolor[HTML]{95ABEC} & \cellcolor[HTML]{95ABEC} & \cellcolor[HTML]{95ABEC} & \cellcolor[HTML]{95ABEC} \\
\multirow{-3}{*}{\cellcolor[HTML]{95ABEC}\textbf{Meta}} & \multirow{-3}{*}{\cellcolor[HTML]{95ABEC}\textbf{\begin{tabular}[c]{@{}c@{}}Soma do \\ Range\end{tabular}}} & \multicolumn{2}{c}{\multirow{-3}{*}{\cellcolor[HTML]{95ABEC}\textbf{Range}}} & \multirow{-3}{*}{\cellcolor[HTML]{95ABEC}\textbf{\begin{tabular}[c]{@{}c@{}}Total de \\ Amostras\end{tabular}}} & \multirow{-3}{*}{\cellcolor[HTML]{95ABEC}\textbf{\begin{tabular}[c]{@{}c@{}}Amostras \\ do Range\end{tabular}}} & \multirow{-3}{*}{\cellcolor[HTML]{95ABEC}\textbf{\begin{tabular}[c]{@{}c@{}}\% das \\ Amostras \\ do Range\end{tabular}}} & \multirow{-3}{*}{\cellcolor[HTML]{95ABEC}\textbf{\begin{tabular}[c]{@{}c@{}}Range de +/- \\ 14\% (28\%) da \\ Faixa Central\end{tabular}}} \\ \hline
99,99\% & 99,994\% & 31 & 70 & \textbf{101} & \textbf{39} & \textbf{38\%} & 72,87\% \\ \hline
\cellcolor[HTML]{C0C0C0}99,99\% & \cellcolor[HTML]{C0C0C0}99,992\% & \cellcolor[HTML]{C0C0C0}73 & \cellcolor[HTML]{C0C0C0}128 & \textbf{201} & \textbf{55} & \textbf{27\%} & \cellcolor[HTML]{C0C0C0}{\color[HTML]{000000} 71,11\%} \\ \hline
99,99\% & 99,991\% & 117 & 184 & \textbf{301} & \textbf{67} & \textbf{22\%} & 72,73\% \\ \hline
\cellcolor[HTML]{C0C0C0}99,99\% & \cellcolor[HTML]{C0C0C0}99,990\% & \cellcolor[HTML]{C0C0C0}162 & \cellcolor[HTML]{C0C0C0}239 & \textbf{401} & \textbf{77} & \textbf{19\%} & \cellcolor[HTML]{C0C0C0}70,62\% \\ \hline
99,99\% & 99,991\% & 207 & 294 & \textbf{501} & \textbf{87} & \textbf{17\%} & 73,64\% \\ \hline
\cellcolor[HTML]{C0C0C0}99,99\% & \cellcolor[HTML]{C0C0C0}99,991\% & \cellcolor[HTML]{C0C0C0}253 & \cellcolor[HTML]{C0C0C0}348 & \textbf{601} & \textbf{95} & \textbf{15\%} & \cellcolor[HTML]{C0C0C0}72,96\% \\ \hline
99,99\% & 99,991\% & 299 & 402 & \textbf{701} & \textbf{103} & \textbf{14\%} & 72,69\% \\ \hline
\cellcolor[HTML]{C0C0C0}99,99\% & \cellcolor[HTML]{C0C0C0}99,990\% & \cellcolor[HTML]{C0C0C0}346 & \cellcolor[HTML]{C0C0C0}455 & \textbf{801} & \textbf{109} & \textbf{13\%} & \cellcolor[HTML]{C0C0C0}72,69\% \\ \hline
99,99\% & 99,991\% & 392 & 509 & \textbf{901} & \textbf{117} & \textbf{12\%} & 72,86\% \\ \hline
\cellcolor[HTML]{C0C0C0}99,99\% & \cellcolor[HTML]{C0C0C0}99,991\% & \cellcolor[HTML]{C0C0C0}439 & \cellcolor[HTML]{C0C0C0}562 & \textbf{1001} & \textbf{123} & \textbf{12\%} & \cellcolor[HTML]{C0C0C0}73,16\% \\ \hline
99,99\% & 99,991\% & 486 & 615 & \textbf{1101} & \textbf{129} & \textbf{11\%} & 73,54\% \\ \hline
\cellcolor[HTML]{C0C0C0}99,99\% & \cellcolor[HTML]{C0C0C0}99,991\% & \cellcolor[HTML]{C0C0C0}533 & \cellcolor[HTML]{C0C0C0}668 & \textbf{1201} & \textbf{135} & \textbf{11\%} & \cellcolor[HTML]{C0C0C0}71,45\% \\ \hline
99,99\% & 99,991\% & 580 & 721 & \textbf{1301} & \textbf{141} & \textbf{10\%} & 72,06\% \\ \hline
\cellcolor[HTML]{C0C0C0}99,99\% & \cellcolor[HTML]{C0C0C0}99,990\% & \cellcolor[HTML]{C0C0C0}628 & \cellcolor[HTML]{C0C0C0}773 & \textbf{1401} & \textbf{145} & \textbf{10\%} & \cellcolor[HTML]{C0C0C0}72,68\% \\ \hline
99,99\% & 99,991\% & 675 & 826 & \textbf{1501} & \textbf{151} & \textbf{10\%} & 73,31\% \\ \hline
\cellcolor[HTML]{C0C0C0}99,99\% & \cellcolor[HTML]{C0C0C0}99,990\% & \cellcolor[HTML]{C0C0C0}723 & \cellcolor[HTML]{C0C0C0}878 & \textbf{1601} & \textbf{155} & \textbf{9\%} & \cellcolor[HTML]{C0C0C0}71,76\% \\ \hline
99,99\% & 99,991\% & 770 & 931 & \textbf{1701} & \textbf{161} & \textbf{9\%} & 72,49\% \\ \hline
\cellcolor[HTML]{C0C0C0}99,99\% & \cellcolor[HTML]{C0C0C0}99,990\% & \cellcolor[HTML]{C0C0C0}818 & \cellcolor[HTML]{C0C0C0}983 & \textbf{1801} & \textbf{165} & \textbf{9\%} & \cellcolor[HTML]{C0C0C0}73,20\% \\ \hline
99,99\% & 99,990\% & 866 & 1035 & \textbf{1901} & \textbf{169} & \textbf{8\%} & 71,90\% \\ \hline
\cellcolor[HTML]{C0C0C0}99,99\% & \cellcolor[HTML]{C0C0C0}99,990\% & \cellcolor[HTML]{C0C0C0}914 & \cellcolor[HTML]{C0C0C0}1087 & \textbf{2001} & \textbf{173} & \textbf{8\%} & \cellcolor[HTML]{C0C0C0}72,67\% \\ \hline
99,99\% & 99,990\% & 1394 & 1607 & \textbf{3001} & \textbf{213} & \textbf{7\%} & 71,86\% \\ \hline
\cellcolor[HTML]{C0C0C0}99,99\% & \cellcolor[HTML]{C0C0C0}99,991\% & \cellcolor[HTML]{C0C0C0}1877 & \cellcolor[HTML]{C0C0C0}2124 & \textbf{4001} & \textbf{247} & \textbf{6\%} & \cellcolor[HTML]{C0C0C0}72,47\% \\ \hline
99,99\% & 99,990\% & 2363 & 2638 & \textbf{5001} & \textbf{275} & \textbf{5\%} & 72,38\% \\ \hline
\cellcolor[HTML]{C0C0C0}99,99\% & \cellcolor[HTML]{C0C0C0}99,990\% & \cellcolor[HTML]{C0C0C0}2850 & \cellcolor[HTML]{C0C0C0}3151 & \textbf{6001} & \textbf{301} & \textbf{5\%} & \cellcolor[HTML]{C0C0C0}72,75\% \\ \hline
99,99\% & 99,990\% & 3338 & 3663 & \textbf{7001} & \textbf{325} & \textbf{4\%} & 72,32\% \\ \hline
\cellcolor[HTML]{C0C0C0}99,99\% & \cellcolor[HTML]{C0C0C0}99,990\% & \cellcolor[HTML]{C0C0C0}3827 & \cellcolor[HTML]{C0C0C0}4174 & \textbf{8001} & \textbf{347} & \textbf{4\%} & \cellcolor[HTML]{C0C0C0}72,18\% \\ \hline
99,99\% & 99,990\% & 4316 & 4685 & \textbf{9001} & \textbf{369} & \textbf{4\%} & 72,23\% \\ \hline
\cellcolor[HTML]{C0C0C0}99,99\% & \cellcolor[HTML]{C0C0C0}99,990\% & \cellcolor[HTML]{C0C0C0}4806 & \cellcolor[HTML]{C0C0C0}5195 & \textbf{10001} & \textbf{389} & \textbf{3\%} & \cellcolor[HTML]{C0C0C0}72,42\% \\ \hline
\end{tabular}%
}
\end{table}
\footnotetext{O Apêndice \ref{app:algoritmos} é dedicado a clarificar o algoritmo BinomialDistribuion\_PROB e validar o fórmula da probabilidade binomial geral usada por ele.}
\vspace{-8mm}
\begin{description}
   \item[Meta] Porcentagem das amostras observadas;
   \item[Soma do Range] Porcentagem que o \textbf{"Range"} atingiu a \textbf{"Meta"}, da mediana para as bordas, descentralizado;Porcentagem que o Range atingiu a Meta, da mediana para as bordas, descentralizado;
   \item[Range] Range de amostras onde a \textbf{"Meta"} foi atingida do \textbf{"Total de Amostras"};
   \item[Total de Amostras] Exibe o range total avaliado, no caso da primeira linha da tabela o valor 101 corresponde às possibilidades de 0 a 100, como se fossem lançadas 100 moedas (distribuição binomial) e somassem suas faces voltadas para cima, podendo ser 0 para as caras e 1 para as coroas. Essa soma é uma combinação de possibilidades não uma permutação, ou seja, na permutação [0 1] é uma possibilidade diferente de [1 0], na combinação essa é uma possiblidade, porém com duas probabilidades de ocorrência;
   \item[Amostras do Range] Quantidade de amostras do \textbf{"Range"} do \textbf{"Total de Amostras"};
   \item[Porcentagem das Amostras do Range] Porcentagem que o \textbf{"Range} representa do \textbf{"Total de Amostras"};
   \item[Range de +/- 14\% (28\%) da Mediana] Esse range é  subconjunto do \textbf{"Range"}, formado a partir da mediana somando 14\% a direita e a esquerda, totalizando 28\%. Esses 28\% correspondem a aproximadamente 72\% das amostras da população do Range, que correspondem a 99,99\% da população total. O restante, que representam 72\% \textbf{"Range"}, correspondem a aproximadamente 28\% das amostras. Isso condiz com o Princípio de Pareto também conhecido como a regra do 80/20, que também pode ser 70/30 ou 90/10, por exemplo \cite{administradores_principio_pareto}.
\end{description}
\bigbreak


It can be seen in Table \ref{tab:10000_all} that as the samples increase, the percentage occupied by 99.99\% of the samples \textbf{"\% of Samples in the Range"} tends to decrease more and more slowly, although the amount of samples representing this percentage tends to increase \textbf{"Samples in the Range"}.

The column of \textbf{"Range Samples"} from Table \ref{tab:10000_all}, blue arrows in the graph of Figure \ref{fig:total_comparison_chart_with_99_range}, will be getting closer and closer to the center of the graph proportionally. Although the amount of \textbf{"Range Samples"} increases, the proportion they take in \textbf{"Total Samples"} decreases. The purple arrows in the graph represent the column \textbf{"Total Samples"} of Table \ref{tab:10000_all}. 
	\begin{figure}[H]
	\caption{Comparison of total samples with a range of 99.99\% }
	\label{fig:total_comparison_chart_with_99_range}
	\centering
	\includegraphics[scale=.9]{sections/images/total_comparison_chart_with_99_range.jpg}
	\floatfoot{The purple arrows represent the "Total Samples" column and the blue arrows the "Range Samples" column of Table \ref{tab:10000_all}. \footnotemark}
	\end{figure}
	\footnotetext{The graph in Figure \ref{fig:total_comparison_chart_with_99_range} represents the first 20 rows of Table \ref{tab:10000_all}, as they suffer equal increments of 100 samples in each row. Rows 21 onwards are incremented by 1000 samples on each row.}

At \url{https://www.mathsisfun.com/data/quincunx.html} there is a tool called Quincunx or Galton Board that dynamically exemplifies what the above pictures show. An explanation of how this tool works can be found at \url{https://www.mathsisfun.com/data/quincunx-explained.html}. 

% ----------------------------------------------------------
% Subseção Consciência
% ----------------------------------------------------------
\subsection{Consciência}
Um momento lógico pode ser formado por uma divisão (primeiro momento) ou por subdivisões lógicas (demais momentos).
	\begin{figure}[H]
	\caption{Intervalo lógico}
	\label{fig:consciousness_logical_moments}
	\centering
	\includegraphics[scale=.7]{sections/images/consciousness_logical_moments.jpg}
	\floatfoot{Exemplo de um intervalo lógico com dez momentos lógicos.}%\footnotemark}
	\end{figure}
	%\footnotetext{Fonte: note}

A consciência são os momentos lógicos de uma expansão representados em suas unidades.
	\begin{figure}[H]
	\caption{Intervalo lógico consciente}
	\label{fig:consciousness}
	\centering
	\includegraphics[scale=.7]{sections/images/consciousness.jpg}
	\floatfoot{Exemplo de um intervalo lógico consciente com dez unidades de momentos lógicos.}%\footnotemark}
	\end{figure}
	%\footnotetext{Fonte: note}

Pode ser observado na Tabela \ref{tab:10000_all} que a probabilidade de 99,99\% das amostras (Amostras do Range), que aumentam em quantidade a medida que crescem os momentos lógicos, tendem a estar cada vez mais ao centro do intervalo lógico, sendo que essa centralização tende ao infinito.
	\begin{figure}[H]
	\caption{Centralização de 99,99\% das amostras}
	\label{fig:centering_of_99_range}
	\centering
	\includegraphics[scale=1]{sections/images/centering_of_99_range.jpg}
	\floatfoot{Tendência de centralização do range de 99,99\% das amostras.}%\footnotemark}
	\end{figure}
	%\footnotetext{Fonte: note}

A consciência tende à representação de um histograma da distribuição normal. Todos os aspectos listados abaixo são inerentes a abstração lógica chamada consciência.

\subsubsection{Infinito}
Um dos aspectos mais importantes que a negação do nada traz (negação de si), é o infinito, ou seja, em qualquer intervalo lógico cabe o infinito novamente. A lógica primordial que iniciou todo o intervalo lógico é a mesma encontrada em seus intervalos subsequentes. Isso fundamenta como uma lógica de alto nível como a subconsciência humana explica a lógica primordial, uma vez que não é preciso voltar ao primeiro momento lógico do intervalo para deduzi-lo, pois esse fenômeno é onipresente em todo o intervalo.

\subsubsection{Ondas}
Probabilisticamente a distribuição de novas amostras de uma população tendem a concentrar mais amostras sentido a mediana da população com frequências de amostras cada vez maiores neste sentido. Porém, a distribuição dessas amostras com frequências de crescimento uniformes é infinitesimal se comparado às possibilidades randômicas desse crescimento. Assim, a tendência de crescimento dessas frequências sentido a mediana somadas a baixíssima probabilidade (infinitesimal) desse crescimento ser uniforme, conduz a frequências no padrão de ondas.
	\begin{figure}[H]
	\caption{Padrão de onda}
	\label{fig:consciousness_waves}
	\centering
	\includegraphics[scale=1]{sections/images/consciousness_waves.jpg}
	\floatfoot{Padrão de onda inferido pela tendência dessa distribuição com frequências maiores sentido a mediana da população e a baixíssima probabilidade de crescimento uniforme dessas frequências.}%\footnotemark}
	\end{figure}
	%\footnotetext{Fonte: note}

A junção de duas ondas além de eliminar suas discrepâncias, faz com que a primeira onda da união fique maior e a segunda onda acabe por deixar de existir a se tornar parte da primeira, que tem seu pico mais próximo da mediana. Probabilisticamente uma onda não morre, apenas une-se com outras ondas mais centrais a ela.
	\begin{figure}[H]
	\caption{Unificação de ondas}
	\label{fig:consciousness_uniform_wave}
	\centering
	\includegraphics[scale=1]{sections/images/consciousness_uniform_wave.jpg}
	\floatfoot{Ondas sendo unificadas para exemplificar o crescimento amostral uniforme.}%\footnotemark}
	\end{figure}
	%\footnotetext{Fonte: note}

\subsubsubsection{Entrelaçamento e subconsciente}
As amostras que mais se parecem em termos de frequências e distribuição são as amostras que fazem parte da mesma onda. Elas são frequências opostas não sobrepostas que se completam.

Probabilisticamente as duas partes complementares de uma onda estarão a uma distância aproximadamente iguais, equidistante da mediana, porém essa não é uma regra e as partes complementares de uma onda podem estar em distâncias diferentes da mediana. O fenômeno da paridade das partes de uma onda tem o nome de entrelaçamento de ondas.

Essas ondas formam subconsciências de uma consciência maior. A consciência é única para todo o intervalo, é a lógica do intervalo, enquanto formam subconsciências ou sub-lógicas, como pequenas ondas de uma onda maior. Assim, uma mudança na onda maior (consciência) também é uma mudança na onda menor (subconsciência), mudança essa que é induzida pelas subconsciências indiretamente, análogo ao comprimir gás em um cilindro, onde ao adicionar uma nova molécula de gás no cilindro parcialmente cheio, mais próximas ou apertas as moléculas dentro dele estarão. O contrário também é verdadeiro, uma nova amostra em uma subconsciência que por esta é observada diretamente é também uma mudança da consciência e vai ser induzida por outras subconsciências indiretamente.
	\begin{figure}[H]
	\caption{Subconsciência}
	\label{fig:consciousness_subconscious}
	\centering
	\includegraphics[scale=1]{sections/images/consciousness_subconscious.jpg}
	\floatfoot{O padrão de ondas forma subconsciências semelhantes ao padrão criado pela consciência (histograma de distribuição normal) como visto na Figura \ref{fig:statisticsbyjim_central_limit_theorem} ou na Figura \ref{fig:trend_chart_of_normal_distribution}.}%\footnotemark}
	\end{figure}
	%\footnotetext{Fonte: note}

\subsubsubsection{Salto}
O salto é uma reordenação feita pelo entrelaçamento de ondas a medida que as amostras do entrelaçamento deixam de ser equivalentes com a adição de novas amostras em seus lados.

Na Figura \ref{fig:consciousness_space_subconscious_observation_jump} é observado os entrelaçamento de ondas (representadas por colunas do histograma na vertical). A reordenação feita pelo entrelaçamento provoca um salto nas coordenadas (X, Y e Z) conforme subseção do Espaço.
	\begin{figure}[H]
	\caption{Reordenação subconsciente - Salto}
	\label{fig:consciousness_space_subconscious_observation_jump}
	\centering
	\includegraphics[scale=.6]{sections/images/consciousness_space_subconscious_observation_jump.jpg}
	\floatfoot{Salto provocado pela não equivalência do entrelaçamento com a adição de novas amostras.}%\footnotemark}
	\end{figure}
	%\footnotetext{Fonte: note}

A tendência probabilística é que, por exemplo, o elétron que saltou de sua orbita de origem retorne à esta conforme mais amostras são adicionadas ao entrelaçamento desse átomo, estabelecendo a normalidade probabilística.

\subsubsection{Tempo}
O tempo é a adição de novos momento lógicos entre momentos existentes à medida que prossegue a negação de si da lógica. Essas mudanças são acumulativas e a medida que aumentam o número desses momentos lógicos, menos relevante cada novo momento será dentro do intervalo consciente. Um em cem é mais relevante do que um em mil. 
	\begin{figure}[H]
	\caption{Tempo}
	\label{fig:consciousness_time}
	\centering
	\includegraphics[scale=.8]{sections/images/consciousness_time.jpg}
	\floatfoot{Progressão do tempo conforme os momentos lógicos avançam.}%\footnotemark}
	\end{figure}
	%\footnotetext{Fonte: note}

Outro fator importante a observar do tempo é que, probabilisticamente, subconsciências mais próximas da mediana da população terão uma adição maior de novas amostras em seus intervalos, o que são observados diretamente por essas subconsciências. Por outro lado, subconsciências distantes da mediana da população terão uma adição menor de amostras em seus intervalos e sujeitam-se a um número maior de mudança induzidas indiretamente. Esse fenômeno de observação temporal proporcionado pela consciência e subconsciências evita o paradoxo dos gêmeos \cite{brasilescola_paradoxo_gemeos}.

Na seção Expansão lógica foi apresentado que a lógica é uma sequência de negações de si no tempo zero, ou seja, em nenhum momento entre suas negações a lógica passa a SER, garantindo a premissa primordial da constante lógica, NÃO SER. Assim, a lógica é uma sequência infinita e simultânea, uma constante. Logo, o tempo é apenas uma grandeza da consciência oriunda da ordenação dessa sequência lógica, não da sequência propriamente. A simultaneidade dessa sequência torna a lógica uma constante com todas as suas infinitas possibilidades, sendo esse universo uma delas. 

Cada universo tem uma ordem diferente em sua sequência e é essa ordem que dá origem à grandeza que chamamos de tempo. É essa ordem do universo ou consciência que vai dar a noção do que acontece antes ou depois, ou seja, o passado, o presente e o futuro. 

Na experiência do tempo conduzida pela consciência a ordenação da sequência é a essência dessa grandeza e, portanto, mais relevante do que sua origem que é de natureza simultânea.

\subsubsection{Espaço}
As ondas da consciência exibidas em forma de histograma, onde as partes das ondas que se completam são colocados lado a lodo é exibida na Figura \ref{fig:consciousness_space_waves}. A formação desse histograma é proveniente do entrelaçamento de ondas.
	\begin{figure}[H]
	\caption{Histograma proveniente do entrelaçamento de ondas}
	\label{fig:consciousness_space_waves}
	\centering
	\includegraphics[scale=.7]{sections/images/consciousness_space_waves.jpg}
	\floatfoot{Exemplo do padrão de ondas obtido pelo algoritmo Logic\_WavePattern. \footnotemark}
	\end{figure}
	\footnotetext{O algoritmo Logic\_WavePattern pode ser visto no Apêndice \ref{app:algoritmos}.}

Ao representar as grandezas espaciais do gráfico da Figura \ref{fig:consciousness_space_waves} em um gráfico de distribuição 3D e distribuir seus pontos de extremidade (desprezando seus volumes e possíveis pontos internos), obtém-se algo parecido com uma espiral (como redemoinhos no ar ou na água) mesmo em volumes muito pequenos de dados (poucos momentos lógicos), conforme Figuras \ref{fig:consciousness_space_3DScatter15000-10} e \ref{fig:consciousness_space_3DScatter_200000-2}. Os pontos se movem em formato de espiral, aproximadamente, uma vez que as coordenadas X, Y e Z aumentam à medida que novas amostras são adicionadas na população.
	\begin{figure}[H]
	\centering
		\begin{subfigure}[H]{0.47\linewidth}
		\centering
		\includegraphics[width=.96\linewidth]{sections/images/consciousness_space_3DScatter15000-10.jpg}
		\caption{15.000 amostras ou momentos}
		\label{fig:consciousness_space_3DScatter15000-10}
		\end{subfigure}
	\hfill
		\begin{subfigure}[H]{0.47\linewidth}
		\centering
		\includegraphics[width=.9\linewidth]{sections/images/consciousness_space_3DScatter_200000-2.jpg}
		\caption{200.000 amostras ou momentos}
		\label{fig:consciousness_space_3DScatter_200000-2}
		\end{subfigure}%
	\caption{Gráfico de dispersão 3D gerado com os pontos da Figura \ref{fig:consciousness_space_waves}}
	\floatfoot{O histograma no padrão de ondas e os dados para gerar o gráfico de dispersão 3D podem ser obtidos com a execução do algoritimo Logic\_WavePattern. \protect\footnotemark}
	\end{figure}
	\footnotetext{O algoritmo Logic\_WavePattern pode ser visto no Apêndice \ref{app:algoritmos} e os gráficos de dispersão 3D podem ser acessados em: \url{https://chart-studio.plot.ly/create/?fid=ren.stuchi:5&fid=ren.stuchi:4} e \url{https://chart-studio.plot.ly/create/?fid=ren.stuchi:7&fid=ren.stuchi:6}}

\subsubsubsection{Comprimento de onda - Intervalo}
A observação de outras subconsciências (sub-lógicas) depende do range de ondas (comprimento de ondas) que uma subconsciência é capaz de observar e esse range, que por sua vez depende do comprimento de ondas que a própria subconsciência é constituída. Todos os possíveis intervalos que se correspondem em X e Y encontram-se simultaneamente formando ondas em diferentes níveis. Dentre todas as possibilidades de intervalos ou comprimento de ondas permitidos por uma população, o observador está um deles ou em um range deles. Alguns exemplos de comprimentos de ondas podem ser observados na Figura \ref{fig:consciousness_space_subconsciousness_examples}.
	\begin{figure}[H]
	\caption{Diferentes comprimentos de ondas - intervalos}
	\label{fig:consciousness_space_subconsciousness_examples}
	\centering
	\includegraphics[scale=.5]{sections/images/consciousness_space_subconsciousness_examples.jpg}
	\floatfoot{Exemplo de diferentes comprimentos de ondas.}%\footnotemark}
	\end{figure}
	%\footnotetext{Fonte: note}
	
Em ranges de muitos momentos lógicos pode-se ver o agrupamento de grandes objetos (subconsciências), sendo o maior deles representado pela cor azul claro e os menores e mais distantes pela cor azul escuro ou roxo, conforme Figura \ref{fig:consciousness_space_subconsciousness}. Esse agrupamento pode representar, por exemplo, o centro do universo, então o centro de uma galáxia, estrelas, planetas e objetos menores e mais distantes.
	\begin{figure}[H]
	\caption{Abstração espacial das subconsciências - grandes agrupamentos}
	\label{fig:consciousness_space_subconsciousness}
	\centering
	\includegraphics[scale=.45]{sections/images/consciousness_space_subconsciousness.jpg}
	\floatfoot{Caracteristicas da ondas formadoras da subconsciência de grandes objetos.}%\footnotemark}
	\end{figure}
	%\footnotetext{Fonte: note}

Em ranges com uma quantidade menor de momentos lógicos pode-se ver o agrupamento de pequenos objetos (subconsciências). Quanto menores os agrupamentos menos divisões esses agrupamentos têm (cores) e mais estreitos e compridos eles são, conforme Figura \ref{fig:consciousness_space_subconsciousness_min}. Esse agrupamento pode representar, por exemplo, o átomo que são muito pequenos, se apresentam em enormes quantidades e as partículas que orbitam seu núcleo (elétrons) ficam bem mais distantes dele.
	\begin{figure}[H]
	\caption{Abstração espacial das subconsciências - pequenos agrupamentos}
	\label{fig:consciousness_space_subconsciousness_min}
	\centering
	\includegraphics[scale=.7]{sections/images/consciousness_space_subconsciousness_min.jpg}
	\floatfoot{Caracteristicas da ondas formadoras da subconsciência de pequenas partículas.}%\footnotemark}
	\end{figure}
	%\footnotetext{Fonte: note}}
	
As cores dos agrupamentos indicam a relação entre conjuntos e subconjuntos. Subconjuntos nascem de conjuntos ou outros subconjuntos e essa relação paterna filial é permanente. Conjuntos e subconjuntos também podem se dividir no mesmo nível, a depender do entrelaçamento das amostras.

\subsubsubsection{Amplitude de onda - Volume}
Da parte inicial da Figura \ref{fig:consciousness_space_volume} originam-se duas observações. A primeira que a altura das colunas do histograma (amplitude) é diretamente relacionada a quantidade de amostras ou momentos lógicos do intervalo (comprimento). A segunda observação é que a densidade de distribuição das amostras de uma coluna do histograma é inalterada. Graficamente é o mesmo que colocar o intervalo (comprimento de onda) na vertical.

O volume dobra a cada um terço de crescimento das amostras ou momentos lógicos de um agrupamento, aproximadamente. Ao imaginar uma esfera com o diâmetro equivalente à amplitude dessa onda unificada, ou seja, com o comprimento dessa onda referente à todo o objeto, a parte mais facilmente observável é onde está a maior concentração de amostras desse intervalo, que é sentido à mediana da população, probabilisticamente, conforme visto na Figura \ref{fig:consciousness_space_volume}.
\begin{figure}[H]
	\caption{Amostras vs volume}
	\label{fig:consciousness_space_volume}
	\centering
	\includegraphics[scale=1]{sections/images/consciousness_space_volume.jpg}
	\floatfoot{O volume em três dimensões dobra a cada um terço de crescimento das amostras, aproximadamente.}%\footnotemark}
	\end{figure}
	%\footnotetext{Fonte: note}}

\subsubsubsection{Espiral}
O padrão de espiral observado na Figura \ref{fig:consciousness_space_spiral} é fundamentado pelo entrelaçamento de ondas, a base para formação do espaço, e o padrão probabilístico descrito pelo teorema central do limite, onde as amostras de uma população tendem sentido à mediana. O padrão de espiral observado não invalida outros possíveis movimentos no espaço. Muitas vezes não é possível observar o padrão de espiral imediatamente nos movimentos de subconsciências, porém parece que este padrão está por traz de muitos destes movimentos, pois ao pegar os movimentos humanos como exemplo tem-se os ciclos predominantes de ir e voltar para casa, ir e voltar ao trabalho, acordar e dormir, ou seja, os hábitos parecem movimentos em círculos, movimentos espirais.
	\begin{figure}[H]
	\caption{Padrão do movimento em espiral}
	\label{fig:consciousness_space_spiral}
	\centering
	\includegraphics[scale=.65]{sections/images/consciousness_space_spiral.jpg}
	\floatfoot{Detalhes do movimento em espiral dos subconjuntos de amostras.}%\footnotemark}
	\end{figure}
	%\footnotetext{Fonte: note}}

Como as coordenadas X, Y e Z de cada subconjunto tendem a aumentar, a disposição dessas em um sistema tridimensional de coordenadas vai seguir uma referência diagonal entre esses três eixos, conforme Figura \ref{fig:consciousness_space_spiral_reference_line}.

Na Figura \ref{fig:consciousness_space_spiral_reference_line} pode ser observado também os pontos X1 e X2. Esses pontos foram espelhados nas coordenadas X e Z para facilitar a observação de que as amostras de um intervalo tende a aumentar em todas as coordenadas. Assim, por mais que X2 esteja representando a média mínima probabilística do objeto A para esse determinado ponto Z, ela é ainda maior que X1, a média máxima probabilística do objeto A para um ponto Z anterior.
	\begin{figure}[H]
	\caption{Sistema tridimensional de coordenadas}
	\label{fig:consciousness_space_spiral_reference_line}
	\centering
	\includegraphics[scale=.7]{sections/images/consciousness_space_spiral_reference_line.jpg}
	\floatfoot{Linha de referência para distribuição de uma população em um plano tridimensional.}%\footnotemark}
	\end{figure}
	%\footnotetext{Fonte: note}}

Os subconjuntos capazes de sobreviver por ciclos espirais (os subconjuntos mais próximos da base, como o subconjunto 1, são mais resilientes) estarão com suas coordenadas X e Y entres os pontos médios máximos e médios mínimos probabilísticos para determinado range em Z, conforme Figura \ref{fig:consciousness_space_spiral_undulation}. Assim, se o objeto B está em seu ponto médio máximo probabilístico, onde subconjunto 1 do bloco 1 estejam como a maior parte das amostras dentro do range de alta densidade, por exemplo. Ao seguir o ciclo esse objeto tenderá a ir para o ponto médio mínimo probabilístico, onde subconjunto 1 do bloco 1 estejam como a maior parte das amostras dentro do range de baixa densidade.
	\begin{figure}[H]
	\caption{Média mínima e média máxima probabilísticas}
	\label{fig:consciousness_space_spiral_undulation}
	\centering
	\includegraphics[scale=.6]{sections/images/consciousness_space_spiral_undulation.jpg}
	\floatfoot{Média mínima e média máxima probabilísticas das coordenadas em relação à linha de referência.}%\footnotemark}
	\end{figure}
	%\footnotetext{Fonte: note}}

Na Figura \ref{fig:consciousness_space_spiral_direction} é exibida a orientação da parte facilmente visível de um objeto juntamente com o espaço que completa a formação deste objeto. A parte facilmente observável probabilisticamente encabeça o movimento, conforme visto na Figura \ref{fig:consciousness_space_spiral_undulation} (na barra azul mais à direita do histograma no bloco 1 subconjunto 1), uma vez que as maiores densidades de amostras estão nas barras do histograma sentido à mediana, podendo as altas densidades estarem mais na parte superior ou inferior de um subconjunto, como o subconjunto 1.
	\begin{figure}[H]
	\caption{Orientação do movimento em espiral}
	\label{fig:consciousness_space_spiral_direction}
	\centering
	\includegraphics[scale=1]{sections/images/consciousness_space_spiral_direction.jpg}
	\floatfoot{Orientação do movimento em espiral da parte visível de um objeto em relação ao espaço que o completa.}%\footnotemark}
	\end{figure}
	%\footnotetext{Fonte: note}}

A adição de novas amostras à uma população cria novos subconjuntos e isso acontece em frequências muito altas em agrupamentos menores como os átomos ou ainda menores. Isso faz com que novos subconjuntos sejam criados antes (menor frequência) e depois (maior frequência sentido à mediana) de um subconjunto especifico. É o mesmo que dizer que átomos serão formados antes e depois de um átomo especifico. E é isso que faz com que um objeto se movimente à frente enquanto fica cada vez mais distante da mediana da população.

Cada agrupamento tem sua própria linha de referência. Assim como dentro de um metro existem os centímetros, milímetros etc., dentro de um agrupamento existem outros agrupamentos.
	\begin{figure}[H]
	\caption{Intervalos e linhas de referências}
	\label{fig:consciousness_space_spiral_underlines}
	\centering
	\includegraphics[scale=.5]{sections/images/consciousness_space_spiral_underlines.jpg}
	\floatfoot{Espirais em diferentes intervalos e suas linhas de referências.}%\footnotemark}
	\end{figure}
	%\footnotetext{Fonte: note}}

O acumulo de amostras em um intervalo de uma coordenada provoca uma elevação dessa coordenada. Um exemplo desses picos são os satélites, onde esses picos os colocam em orbita afastada da terra. A manutenção dessa orbita é sustentada pela manutenção constante desse pico de intervalo, o que promove uma aceleração no tempo. Como visto na subseção de Tempo, as amostras mais próximas da mediana, onde são mais frequentes esses intervalos com um número maior de amostras (pico), sofrem uma adição maior de amostras diretamente em relação aos intervalos longe da mediana que acabam induzindo essas mudanças indiretamente. 

\subsubsection{Forças fundamentais}
A força gravitacional, a força eletromagnética e a força nuclear correspondem às forças fundamentais da natureza e essas forças também são provenientes do entrelaçamento de ondas, como o espaço. As forças fundamentais não são forças propriamente, mas sim aspectos probabilísticos (distribuição normal) e do entrelaçamento de ondas principalmente.

\subsubsubsection{Força gravitacional}
O entrelaçamento ondas é o aspecto que coordena as mudanças nas coordenadas espaciais junto com a adição de novos momentos lógicos sentido a mediana da população. As mudanças dessas coordenadas provocam iterações que podem ser vistas nas Figuras \ref{fig:consciousness_space_3DScatter15000-10} e \ref{fig:consciousness_space_3DScatter_200000-2} da subseção de Espaço e na Figura \ref{fig:consciousness_dark_matter_dark_energy_wave} que mostra probabilisticamente onde está a maior concentração de momentos lógicos de um intervalo consciente ou subconsciente, devido a estes momentos serem mais intensos sentido a mediana. Estes aspectos são chamadas de gravidade.

\subsubsubsection{Força eletromagnética}
A força eletromagnética é uma especificação do aspecto gravitacional que depende da aproximação espacial (redução de diferenças nos eixos X, Y e Z) e do entrelaçamento de ondas.

Quando um objeto se aproxima de outro, seus pares de ondas provenientes do entrelaçamento de ondas ficam cada vez mais parecidos, eixos X e Y. Essa proximidade faz com que as partes das ondas de um objeto se pareça muito com as partes das ondas do outro objeto, o que pode fazer com que o entrelaçamento de ondas encontre pares mais ideais nesse outro objeto e vice-versa.  

As linhas azuis da Figura \ref{fig:consciousness_electromaagnetic_force} mostra onde é mais frequente a troca dos pares de ondas pelo entrelaçamento de ondas, ou seja, onde se tem a maior probabilidade das ondas serem parecidas. Por isso os imãs tentam se virar para se conectar quando estão face a face com o mesmo polo. As linhas cinza mostram as conexões que ocorrem em número bem menor. 
	\begin{figure}[H]
	\caption{Força eletromagnética}
	\label{fig:consciousness_electromaagnetic_force}
	\centering
	\includegraphics[scale=.6]{sections/images/consciousness_electromaagnetic_force.jpg}
	\floatfoot{Aumento das possibilidades de entrelaçamento de ondas devida a aproximação e o menor número de momentos lógicos das menores partículas. }%\footnotemark}
	\end{figure}
	%\footnotetext{Fonte: note}

Com a troca de significativos pares de ondas entre os objetos faz-se a mixagem do posicionamento dos eixos X, Y e Z entre esses objetos ocorrendo a aproximação deles no espaço. 

Quanto menor a partícula (elétron ou partículas menores), conforme Figura \ref{fig:consciousness_space_subconsciousness_min}, mais fácil o entrelaçamento ocorre. Provavelmente muitos objetos não tenham alta capacidade de entrelaçamento devido aos seus elétrons ou partículas menores serem formadas por muitos momentos lógicos (barras do histograma mais largas ou mais compridas), ou seja, quanto maior a quantidade de momentos dessas partículas menores as chances de entrelaçamento.

Probabilisticamente as partículas mais parecidas estão nas regiões mais próximas (linhas azuis do Figura \ref{fig:consciousness_electromaagnetic_force}) devido ao crescimento do número de amostras sentido a mediana da população, porém isso não é uma regra e os polos podem se inverter, ou seja, ter mais ligações com a região de menor probabilidade (isso não quer dizer que houve formação de antimatéria nessa região, as partículas ainda tendem a concentrar mais momentos lógicos sentido à mediana da população). No entanto, a probabilidade tende a corrigir esses polos conforme novos momentos vão sendo adicionadas nesse intervalo.

\subsubsubsection{força nuclear}
As forças nucleares forte e fraca representam as maiores concentrações de momentos lógicos por intervalo populacional. Esses picos podem ser vistos na Figura \ref{fig:consciousness_space_subconsciousness_min} e eles não param de crescer à medida que novos momentos lógicos são adicionados nestes intervalos. Estes momentos ou amostras tendem a estarem cada vez mais juntos dentro do intervalo formando picos cada vez mais altos.

\subsubsection{Matéria escura e energia escura}
Quanto maior o número de amostras e mais próximas elas estão da mediana, mais elas farão parte dos 99,99\% e ainda mais amostras também estarão nos 0,01\%, conforme a Tabela \ref{tab:10000_all}. Logo, a energia escura não é uma energia propriamente, mas sim um aspecto probabilístico. 
	\begin{figure}[H]
	\caption{Aspecto probabilístico da energia escura}
	\label{fig:consciousness_dark_matter_dark_energy}
	\centering
	\includegraphics[scale=.9]{sections/images/consciousness_dark_matter_dark_energy.jpg}
	\floatfoot{A energia escura não é uma energia propriamente, mas sim um aspecto probabilístico.}%\footnotemark}
	\end{figure}
	%\footnotetext{Fonte: note}

Já a matéria escura, como pode ser visto na Figura \ref{fig:consciousness_dark_matter_dark_energy_wave} mostra probabilisticamente onde está a maior concentração das amostras de um intervalo, tornando mais fácil a visualização por outras subconsciências, uma vez também que o volume dobra a cada um terço do crescimento das colunas do histograma, aproximadamente, conforme dito na seção do Espaço. Assim, uma grande área do intervalo de um agrupamento pode conter amostras dispersas que se tornam mais difíceis de observar. O aspecto descrito acima e demostrado pela Figura \ref{fig:consciousness_dark_matter_dark_energy_wave} é aplicável a qualquer intervalo de um agrupamento (Figuras \ref{fig:consciousness_space_subconsciousness} e \ref{fig:consciousness_space_subconsciousness_min}).
	\begin{figure}[H]
	\caption{Analogia da matéria escura}
	\label{fig:consciousness_dark_matter_dark_energy_wave}
	\centering
	\includegraphics[scale=.75]{sections/images/consciousness_dark_matter_dark_energy_wave.jpg}
	\floatfoot{Parte do volume é facilmente observado por outras subconsciências.}%\footnotemark}
	\end{figure}
	%\footnotetext{Fonte: note}

\subsubsection{Antimatéria}
Independente do intervalo observado, sua maior concentração de amostras tende a estar sentido da mediana, o que é o sentido provável conforme teorema central do limite. Essas amostras também podem estar com sua concentração no sentido oposto à mediana, porém com uma ocorrência probabilística cada vez menos conforme as amostras aumentam. Na Figura \ref{fig:consciousness_concentration_of_opposite_samples} é exibido dois intervalos idênticos com suas amostras com concentrações opostas.
	\begin{figure}[H]
	\caption{Parte de um intervalo idêntico com suas concentrações de amostras opostas}
	\label{fig:consciousness_concentration_of_opposite_samples}
	\centering
	\includegraphics[scale=1]{sections/images/consciousness_concentration_of_opposite_samples.jpg}
	\floatfoot{Parte de um intervalo idêntico distribuídos de formas opostas.}%\footnotemark}
	\end{figure}
	%\footnotetext{Fonte: note}

O merge ou soma dos intervalos opostos da Figura \ref{fig:consciousness_concentration_of_opposite_samples} os tornaria um intervalo simétrico, ou seja, não estaria em nenhum dos sentidos.
Na Figura \ref{fig:consciousness_concentration_of_opposite_samples_within_range} é exibido um intervalo consciente completo com suas concentrações de amostras sentido à mediana e outro idêntico, mas com suas concentrações sentido às bordas do intervalo.
	\begin{figure}[H]
	\caption{Intervalos conscientes com suas concentrações de amostras opostas}
	\label{fig:consciousness_concentration_of_opposite_samples_within_range}
	\centering
	\includegraphics[scale=.8]{sections/images/consciousness_concentration_of_opposite_samples_within_range.jpg}
	\floatfoot{Intervalos conscientes completos e idênticos distribuídos de formas opostas.}%\footnotemark}
	\end{figure}
	%\footnotetext{Fonte: note}


\subsubsection{Buraco negro}
O buraco negro é uma concentração muito alta de amostras, formada por grandes agrupamentos subconscientes, Figura \ref{fig:consciousness_space_subconsciousness}.
Esses grandes agrupamentos ocupam grandes volumes de espaço devido a quantidade de amostras. 

Os grandes volumes são encontrados na base dos grandes agrupamentos, conforme as cores azul claro e cinza da Figura \ref{fig:consciousness_black_hole}.
	\begin{figure}[H]
	\caption{Buracos negros}
	\label{fig:consciousness_black_hole}
	\centering
	\includegraphics[scale=.6]{sections/images/consciousness_black_hole.jpg}
	\floatfoot{Grandes volumes são encontrados na base dos grandes agrupamentos.}%\footnotemark}
	\end{figure}
	%\footnotetext{Fonte: note}

% ----------------------------------------------------------
% Observations subsection
% ----------------------------------------------------------
\subsection{Observations}
	\begin{description}
	   \item[Core] The negation of logic to itself (nothingness) gave rise to three axioms that are the basis of the core theorem of this theory and the basis for existence. This theorem gives rise to waves and their main attribute, wave entanglement.
	   \item[Logical rigidity] If physical rigidity and its laws seem insurmountable, below it is logic, even more rigid and insurmountable, because outside logic is the non-existent, the illogical. Existence is contained in the possibilities of what is logical. 
	   \item[Mathematics] Logic in its essence is not subject to mathematics, but all mathematics is restricted to logic, and therefore some of its simplest constructions may come closer to essential logic than others.
	   \item[Good and evil] Good and evil depend on the observer and are only valid possibilities among infinite others. Perhaps the greatest justice of the universe or logic is the non-exclusion of any path or possibilities. That is, if it is light, negation tends to darken, if it is hot to cool, etc. It is the struggle of opposites of Heraclitus of Ephesus. 
	   \item[Perfection] The primordial logic is the simplest logic, it is the essence of existence. A logic as simple as it is efficient, as efficient as it is perfect:
	   \begin{description}
		   \item[Omnipotent] The essence of all logical possibilities, that is, the essence of existence, because outside of logical possibilities is the illogical, the non-existent;
		   \item[Omniscient] Flow of all logical abstractions from consciousness to the subconsciousnesses; 
		   \item[Omnipresent] Its fractions (negations) are in all existence.
	   \end{description}
	These remarks refer to God, the consciousness of subconsciousnesses. Ultimately, God is Logic, from its infinitesimal and fundamental negation of itself to its infinite greatness. God is love, and the essence of love is attraction, which is also present in the fundamental "forces".
	   \item[Reality] As a logical possibility, the dream is as real as "reality". Perhaps the study of logical possibilities leads to paths where dreams can be as real as reality, since both are just logic, like lucid dreams, for example \cite{lucid_dreams}.  This may explain why other possible forms of "intelligent" life, when evolved, stop looking for this kind of life in a possible vast universe and look within themselves, where something much larger than the universe can be found, infinite.
	   \item[Convergence] Quantum leap and entanglement are some of the behaviors that already challenge the physical world, and may be the point of convergence with this new paradigm.
	\end{description}