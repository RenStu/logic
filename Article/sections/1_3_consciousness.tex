% ----------------------------------------------------------
% Subseção Consciência
% ----------------------------------------------------------
\subsection{Consciência}
Como visto na seção do teorema central do limite, um momento lógico é formado por divisão e subdivisões lógicas, como são as subunidades de espaço ou tempo. Um momento lógico pode ser representado por suas subunidades ou por sua unidade.

\begin{figure}[H]
\caption{Intervalos lógicos}
\label{fig:2_consciousnesses_in_all_unconscious}
\centering
\includegraphics[scale=1]{sections/images/2_consciousnesses_in_all_unconscious.jpg}
\floatfoot{Exemplo de abrangência de dois intervalos lógicos.}%\footnotemark}
\end{figure}
%\footnotetext{Fonte: note}

A consciência são os momentos lógicos de um intervalo representados em suas unidades.

\begin{figure}[H]
\centering
	\begin{subfigure}[H]{.8\linewidth}
	\centering
	\includegraphics[width=.6\linewidth]{sections/images/first_consciousness.jpg}
	\caption{}
	\label{fig:first_consciousness}
	\end{subfigure}
%\hfill
	\begin{subfigure}[H]{.8\linewidth}
	\centering
	\includegraphics[width=1\linewidth]{sections/images/second_consciousness.jpg}
	\caption{}
	\label{fig:second_consciousness}
	\end{subfigure}%
\caption{Intervalos conscientes}

\floatfoot{Exemplo de dois intervalos conscientes, momentos lógicos como unidades de negação.} %\protect\footnotemark}
\end{figure}
%\footnotetext{Note}

Pode ser observado na Tabela \ref{tab:10000_all} que a probabilidade de 99,99\% das amostras, que aumentam em quantidade a medida que crescem os momentos lógicos, tendem a estar cada vez mais ao centro do intervalo lógico, sendo que essa centralização tende ao infinito.

\begin{figure}[H]
\caption{Centralização de 99,99\% das amostras}
\label{fig:centering_of_99_range}
\centering
\includegraphics[scale=1]{sections/images/centering_of_99_range.jpg}
\floatfoot{Tendência de centralização do range de 99,99\% das amostras.}%\footnotemark}
\end{figure}
%\footnotetext{Fonte: note}

A Figura \ref{fig:unconsciousness_consciousness_consciousness_nested} também exemplifica bem essa centralização de 99.99\% das amostras na parte da figura nomeada \textbf{Consciência}. Nela é possível ver que as extremidades que em dado momento estiveram dentro desse range de 99.99\% passam a ter uma relevância lógica cada vez mais próxima de zero à medida que crescem os momentos lógicos. Porém, o que não é tão relevante na parte da Figura nomeada \textbf{Consciência} (uma consciência maior e mais abrangente), continua sendo extremamente relevante à \textbf{Consciência aninhada} (consciências menores). É análogo ao que acontece no corpo humano, não é observado pela consciência humana às mudanças de todas as células do corpo ou ainda de muitos órgãos, porém esses outros níveis de abstração sofrem a mesma evolução da negação de si. A contínua expansão centralizada da \textbf{Consciência} e da \textbf{Consciência aninhada} sugerem a formação dos chamados buracos negros, detalhados mais a frente. Essas características também sugerem que buracos negros podem conter outros buracos negros.

\begin{figure}[H]
\caption{Consciência e Consciência aninhada}
\label{fig:unconsciousness_consciousness_consciousness_nested}
\centering
\includegraphics[scale=1]{sections/images/unconsciousness_consciousness_consciousness_nested.jpg}
\floatfoot{Esboços de histogramas que exemplificam a consciência e consciência aninhada.}%\footnotemark}
\end{figure}
%\footnotetext{Fonte: note}

A consciência é o conjunto dos momentos lógicos de um intervalo. É o aspecto da lógica que unifica as amostras desses momentos, ou seja, é a lógica que abstrai muitos em um, muitas subunidades em uma unidade por momento lógico, podendo essa unidade ser uma subunidade de uma unidade superior. Todos os aspectos listados abaixo são inerentes a abstração da lógica chamada consciência.

\subsubsection{Infinito}
Um dos aspectos mais importantes que a negação do nada traz (negação de si), é o infinito. E um dos aspectos mais importantes do infinito é que as possibilidades lógicas encontradas em um intervalo lógico superior podem também ser encontradas em intervalos lógicos inferiores. A chance de ciclos de possibilidades idênticos é uma das infinitas possibilidades do infinito. Ou seja, todo intervalo lógico é um começo, assim a criatura pode ser o criador daquele que o criou em outr fluxo lógico. Não há fim, não há meio, apenas infinitos começos. Isso fundamenta como uma lógica complexa como a consciência explica a lógica primordial, uma vez que não é preciso voltar ao primeiro momento lógico de todo o intervalo para observá-lo, toda negação de um intervalo  ou subintervalo lógico é seu primeiro momento lógico.

\subsubsection{Tempo}
O tempo é a adição de novos momento lógicos à medida que prossegue a negação desses momentos.  Essas mudanças são acumulativas e o momento lógico futuro é gerado pela negação do momento presente e somado a este tornando a consciência diferente.

\begin{figure}[H]
\caption{Tempo}
\label{fig:consciousness_time}
\centering
\includegraphics[scale=1]{sections/images/consciousness_time.jpg}
\floatfoot{Progressão do tempo conforme os momentos lógicos avançam.}%\footnotemark}
\end{figure}
%\footnotetext{Fonte: note}

\subsubsection{Espaço}
O espaço é a relação da proporção dos intervalos dos momentos lógicos. A proporção da fração lógica (intervalo azul) com a unidade lógica (intervalo cinza), da unidade com a fração lógica e da diferença de entre as frações lógicas.

\begin{figure}[H]
\caption{Espaço}
\label{fig:consciousness_space}
\centering
\includegraphics[scale=1]{sections/images/consciousness_space.jpg}
\floatfoot{Relação da proporção dos intervalos dos momentos lógicos.}%\footnotemark}
\end{figure}
%\footnotetext{Fonte: note}

\subsubsection{Gravidade}
A gravidade é um aspecto probabilístico da distribuição amostral de uma população, como previsto pelo teorema central do limite. Esse teorema afirma que a distribuição amostral de uma população se aproxima de uma distribuição normal à medida que o tamanho das amostras aumenta, o que tende probabilisticamente à centralização infinita das amostras conforme os momentos lógicos progridem. A atração do amor, a gravidade que atraem os objetos à terra e a terra ao sol são sinônimos deste mesmo aspecto.

\begin{figure}[H]
\caption{Gravidade}
\label{fig:consciousness_gravity}
\centering
\includegraphics[scale=1]{sections/images/consciousness_gravity.jpg}
\floatfoot{Centralização infinita das amostras conforme os momentos lógicos progridem.}%\footnotemark}
\end{figure}
%\footnotetext{Fonte: note}

\subsubsection{Matéria escura e energia escura}
Na Figura \ref{fig:consciousness_dark_matter_dark_energy}, os três intervalos do "momento x" e os dez intervalos do "momento xy" sofrem "observação consciente". Cada intervalo do "momento xy" probabilisticamente concentram 99,99\% de suas amostras em uma proporção cada vez menor próximo ao centro do intervalo, fazendo com que os 0,01\% restantes ocupem a maior área ao redor desse range no intervalo, se tornando irrelevante à "observação consciente". Quanto maior for o número de amostras dentro do range de 0,01\% mais distantes ficaram os ranges de 99,99\% uns dos outros. O mesmo acontece com os intervalos do "momento x" e do intervalo máximo da "lógica primordial", o que faz com que seus intervalos aninhados continuem a sofrer as imposições de suas "observações conscientes" que tendem a centralizar mesmo que seus intervalos aninhados tendam a aumentar a área ocupada por suas amostras, que se tornam mais e mais irrelevantes à consciência (0,01\%) à medida que aumentam os momentos lógicos e as amostras se encontrem mais próximas à borda do intervalo. Em suma, explorando a Figura \ref{fig:consciousness_dark_matter_dark_energy}, é o mesmo que observar a parte “observação consciente” dentro de cada intervalo dos “momentos x e xy”. 

\begin{figure}[H]
\caption{Analogia da matéria escura e energia escura}
\label{fig:consciousness_dark_matter_dark_energy}
\centering
\includegraphics[scale=.8]{sections/images/consciousness_dark_matter_dark_energy.jpg}
\floatfoot{Fenômenos antevistos ou conjecturados pela consciência.}%\footnotemark}
\end{figure}
%\footnotetext{Fonte: note}

\subsubsection{Buraco negro}
Assim como a gravidade o buraco negro é um aspecto probabilístico da distribuição amostral de uma população, como previsto pelo Teorema Central do Limite. Quanto mais momentos lógicos, mais amostras, o que tende a centralizar cada vez mais amostras da consciência em uma proporção cada vez menor do intervalo lógico. Essa proporção do intervalo lógico cada vez menor tende ao infinito assim como a quantidade de amostras crescentes que ela envolve, ou seja, um alto volume de amostras em uma proporção inobservável a certas abstrações de consciência. Com essa observação é possível observar que as consciências tendem a se concentrar em intervalos infinitamente menores à medida que crescem, portanto a morte lógica ou consciente é apenas a incapacidade de observação de proporções infinitamente pequenas.

\begin{figure}[H]
\caption{Buraco negro}
\label{fig:consciousness_black_hole}
\centering
\includegraphics[scale=1]{sections/images/consciousness_black_hole.jpg}
\floatfoot{Centralização infinita das amostras em uma proporção centralizada cada vez menor.}%\footnotemark}
\end{figure}
%\footnotetext{Fonte: note}
