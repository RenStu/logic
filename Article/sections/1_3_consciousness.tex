% ----------------------------------------------------------
% Subseção Consciência
% ----------------------------------------------------------
\subsection{Consciência}
Como visto na seção do teorema central do limite, um momento lógico é formado por divisão e subdivisões lógicas, como são as subunidades de espaço ou tempo. Um momento lógico pode ser representado por suas subunidades ou por sua unidade.

\begin{figure}[H]
\caption{Intervalos lógicos}
\label{fig:2_consciousnesses_in_all_unconscious}
\centering
\includegraphics[scale=1]{sections/images/2_consciousnesses_in_all_unconscious.jpg}
\floatfoot{Exemplo de abrangência de dois intervalos lógicos.}%\footnotemark}
\end{figure}
%\footnotetext{Fonte: note}

A consciência são os momentos lógicos de um intervalo representados em suas unidades.

\begin{figure}[H]
\centering
	\begin{subfigure}[H]{.8\linewidth}
	\centering
	\includegraphics[width=.6\linewidth]{sections/images/first_consciousness.jpg}
	\caption{}
	\label{fig:first_consciousness}
	\end{subfigure}
%\hfill
	\begin{subfigure}[H]{.8\linewidth}
	\centering
	\includegraphics[width=1\linewidth]{sections/images/second_consciousness.jpg}
	\caption{}
	\label{fig:second_consciousness}
	\end{subfigure}%
\caption{Intervalos conscientes}

\floatfoot{Exemplo de dois intervalos conscientes, momentos lógicos como unidades de negação.} %\protect\footnotemark}
\end{figure}
%\footnotetext{Note}

Pode ser observado na Tabela \ref{tab:10000_all} que a probabilidade de 99,99\% das amostras, que aumentam em quantidade a medida que crescem os momentos lógicos, tendem a estar cada vez mais ao centro do intervalo lógico, sendo que essa centralização tende ao infinito.

\begin{figure}[H]
\caption{Centralização de 99,99\% das amostras}
\label{fig:centering_of_99_range}
\centering
\includegraphics[scale=1]{sections/images/centering_of_99_range.jpg}
\floatfoot{Tendência de centralização do range de 99,99\% das amostras.}%\footnotemark}
\end{figure}
%\footnotetext{Fonte: note}

A Figura \ref{fig:unconsciousness_consciousness_consciousness_nested} também exemplifica bem essa centralização de 99.99\% das amostras na parte da figura nomeada \textbf{Consciência}. Nela é possível ver que as extremidades que em dado momento estiveram dentro desse range de 99.99\% passam a ter uma relevância lógica cada vez mais próxima de zero à medida que crescem os momentos lógicos. Porém, o que não é tão relevante na parte da Figura nomeada \textbf{Consciência} (uma consciência maior e mais abrangente), continua sendo extremamente relevante à \textbf{Consciência aninhada} (consciências menores). É análogo ao que acontece no corpo humano, não é observado pela consciência humana às mudanças de todas as células do corpo ou ainda de muitos órgãos, porém esses outros níveis de abstração sofrem a mesma evolução da negação de si. A contínua expansão centralizada da \textbf{Consciência} e da \textbf{Consciência aninhada} sugerem a formação dos chamados buracos negros, detalhados mais a frente. Essas características também sugerem que buracos negros podem conter outros buracos negros.

\begin{figure}[H]
\caption{Consciência e Consciência aninhada}
\label{fig:unconsciousness_consciousness_consciousness_nested}
\centering
\includegraphics[scale=1]{sections/images/unconsciousness_consciousness_consciousness_nested.jpg}
\floatfoot{Esboços de histogramas que exemplificam a consciência e consciência aninhada.}%\footnotemark}
\end{figure}
%\footnotetext{Fonte: note}

A consciência é o conjunto dos momentos lógicos de um intervalo. É o aspecto da lógica que unifica as amostras desses momentos, ou seja, é a lógica que abstrai muitos em um, muitas subunidades em uma unidade por momento lógico, podendo essa unidade ser uma subunidade de uma unidade superior. Todos os aspectos listados abaixo são inerentes a abstração da lógica chamada consciência.

\subsubsection{Infinito}
Um dos aspectos mais importantes que a negação do nada traz (negação de si), é o infinito. E um dos aspectos mais importantes do infinito é que as possibilidades lógicas encontradas em um intervalo lógico superior podem também ser encontradas em intervalos lógicos inferiores. A chance de ciclos de possibilidades idênticos é uma das infinitas possibilidades do infinito. Ou seja, todo intervalo lógico é um começo, assim a criatura pode ser o criador daquele que o criou em outr fluxo lógico. Não há fim, não há meio, apenas infinitos começos. Isso fundamenta como uma lógica complexa como a consciência explica a lógica primordial, uma vez que não é preciso voltar ao primeiro momento lógico de todo o intervalo para observá-lo, toda negação de um intervalo  ou subintervalo lógico é seu primeiro momento lógico.

\subsubsection{Tempo}
O tempo é a adição de novos momento lógicos à medida que prossegue a negação desses momentos.  Essas mudanças são acumulativas e o momento lógico futuro é gerado pela negação do momento presente e somado a este tornando a consciência diferente. A medida que aumentam em número de amostras, menos relevante cada nova amostra será dentro do intervalo consciente. Um em cem é mais relevante do que um em mil. 

\begin{figure}[H]
\caption{Tempo}
\label{fig:consciousness_time}
\centering
\includegraphics[scale=1]{sections/images/consciousness_time.jpg}
\floatfoot{Progressão do tempo conforme os momentos lógicos avançam.}%\footnotemark}
\end{figure}
%\footnotetext{Fonte: note}

\subsubsection{Espaço}
O espaço é a relação da proporção dos intervalos dos momentos lógicos. A proporção da fração lógica (intervalo azul) com a unidade lógica (intervalo cinza), da unidade com a fração lógica e da diferença de entre as frações lógicas.

\begin{figure}[H]
\caption{Espaço}
\label{fig:consciousness_space}
\centering
\includegraphics[scale=1]{sections/images/consciousness_space.jpg}
\floatfoot{Relação da proporção dos intervalos dos momentos lógicos.}%\footnotemark}
\end{figure}
%\footnotetext{Fonte: note}

\subsubsection{Gravidade}
A gravidade é um aspecto probabilístico da distribuição amostral de uma população, como previsto pelo teorema central do limite. Esse teorema afirma que a distribuição amostral de uma população se aproxima de uma distribuição normal à medida que o tamanho das amostras aumenta, o que tende probabilisticamente à centralização infinita das amostras conforme os momentos lógicos progridem. A atração do amor, a gravidade que atraem os objetos à terra e a terra ao sol são sinônimos deste mesmo aspecto. Em outras palavras, a gravidade é semelhante uma conexão que cresce à medida que as frequências amostrais aumentam sentido à mediana.

\begin{figure}[H]
\caption{Gravidade}
\label{fig:consciousness_gravity}
\centering
\includegraphics[scale=1]{sections/images/consciousness_gravity.jpg}
\floatfoot{Centralização infinita das amostras conforme os momentos lógicos progridem.}%\footnotemark}
\end{figure}
%\footnotetext{Fonte: note}

\subsubsection{Matéria escura e energia escura}
Quanto maior o número de amostras e mais próximas elas estão da mediana, mais elas farão parte dos 99,99\% e ainda mais amostras também estarão nos 0,01\%, conforme a Tabela \ref{tab:10000_all}. Assim, a observação desses 0,01\% passa a ser cada vez mais difícil, pois sua relevância consciente passa a ser cada vez mais próxima de zero. É importante notar também que a medida que os 99,99\% aumentam em número de amostras, menos relevante cada nova amostra será dentro desse conjunto (um em cem é mais relevante do que um em mil) e uma porcentagem menor será ocupada pelo range dos 99,99\% das amostras, conforme a Tabela \ref{tab:10000_all}.

\begin{figure}[H]
\caption{Analogia da matéria escura e energia escura}
\label{fig:consciousness_dark_matter_dark_energy}
\centering
\includegraphics[scale=1]{sections/images/consciousness_dark_matter_dark_energy.jpg}
\floatfoot{Fenômenos antevistos ou conjecturados pela consciência.}%\footnotemark}
\end{figure}
%\footnotetext{Fonte: note}

\subsubsection{Antimatéria}
Independente do intervalo observado (análogo à bytes, kilobytes, prótons, elétrons etc.), que são contextos lógicos de observação e/ou utilização consciente, este pode estar com sua maior concentração de amostras no sentido da mediana, o que é o sentido provável conforme os números de amostras crescem em um intervalo, conforme teorema central do limite. Essas amostras também podem estar com sua concentração no sentido oposto a mediana, porém com uma ocorrência probabilística cada vez menos conforme as amostras aumentam. Na Figura \ref{fig:consciousness_concentration_of_opposite_samples} é exibido dois intervalos idênticos com suas amostras com concentrações opostas.

\begin{figure}[H]
\caption{Parte de um intervalo idêntico com suas concentrações de amostras opostas}
\label{fig:consciousness_concentration_of_opposite_samples}
\centering
\includegraphics[scale=.8]{sections/images/consciousness_concentration_of_opposite_samples.jpg}
\floatfoot{Parte de um intervalo idêntico distribuídos de formas opostas.}%\footnotemark}
\end{figure}
%\footnotetext{Fonte: note}

O merge ou soma dos intervalos opostos da Figura \ref{fig:consciousness_concentration_of_opposite_samples} os tornaria um intervalo simétrico, ou seja, não estaria em nenhum dos sentidos.
Na Figura \ref{fig:consciousness_concentration_of_opposite_samples_within_range} é exibido um intervalo consciente completo com suas concentrações de amostras sentido à mediana e outro idêntico, mas com suas concentrações sentido às bordas do intervalo.

\begin{figure}[H]
\caption{Intervalos conscientes com suas concentrações de amostras opostas}
\label{fig:consciousness_concentration_of_opposite_samples_within_range}
\centering
\includegraphics[scale=.8]{sections/images/consciousness_concentration_of_opposite_samples_within_range.jpg}
\floatfoot{Intervalos conscientes completos e idênticos distribuídos de formas opostas.}%\footnotemark}
\end{figure}
%\footnotetext{Fonte: note}

\subsubsection{Ondas}
Probabilisticamente a distribuição de novas amostras de uma população tendem a concentrar mais amostras sentido a mediana da população com frequências de amostras cada vez maiores neste sentido. Porém, a distribuição dessas amostras com frequências de crescimento uniformes é infinitesimal se comparado às possibilidades randômicas desse crescimento. Assim, a tendência de crescimento dessas frequências sentido a mediana somadas a baixíssima probabilidade (infinitesimal) desse crescimento ser uniforme, conduz a frequências no padrão de ondas. Conscientemente, grandes intervalos com baixas frequências de amostras e grandes intervalos com frequências uniformes de amostras são mais difíceis de observar devido à ausência de discrepâncias.

\begin{figure}[H]
\caption{Padrão de onda}
\label{fig:consciousness_waves}
\centering
\includegraphics[scale=1]{sections/images/consciousness_waves.jpg}
\floatfoot{Padrão de onda inferido pela tendência dessa distribuição com frequências maiores sentido a mediana da população e a baixíssima probabilidade de crescimento uniforme dessas frequências.}%\footnotemark}
\end{figure}
%\footnotetext{Fonte: note}

\subsubsection{Coexistência}
A tendência probabilística descrita pelo teorema central do limite faz com que a população concentre uma frequência menor de amostras em suas extremidades que vai aumentando gradualmente conforme se aproxima da mediana, conforme visto na seção anterior do padrão de ondas. A frequência de amostras dispostas desta forma torna as interações conscientes mais semelhantes em conjuntos de amostras adjacentes. Assim, ao comparar a frequência de um conjunto de amostras próximo à mediana com um conjunto de amostras mais próximo da extremidade de uma grande população, probabilisticamente haverá uma grande discrepância em suas frequências, ou seja, em um intervalo com tamanho similares haverá mais amostras no conjunto próximo à mediana. Essa observação mostra que conjuntos ou intervalos de frequências adjacentes são mais parecidos do que conjuntos distantes. Provavelmente esses conjuntos de ondas com frequências semelhantes e adjacentes são tratados pela consciência e os demais pelo subconsciente ou inconsciente. 

\begin{figure}[H]
\caption{Conjuntos de frequências de um intervalo}
\label{fig:consciousness_coexistence}
\centering
\includegraphics[scale=1]{sections/images/consciousness_coexistence.jpg}
\floatfoot{Diferentes frequências de um intervalo e suas paridades.}%\footnotemark}
\end{figure}
%\footnotetext{Fonte: note}

Probabilisticamente as amostras que mais se parecem em termos de frequências e distribuição são as amostras que fazem parte da mesma onda, que em momentos passados estiveram mais próximas. Essas não são frequências opostas que se sobrepõem e se anulam como na antimatéria, mas sim frequências opostas não sobrepostas que se completam.

\begin{figure}[H]
\caption{Intervalos de uma mesma onda}
\label{fig:consciousness_coexistence_waves}
\centering
\includegraphics[scale=1]{sections/images/consciousness_coexistence_waves.jpg}
\floatfoot{Intervalos opostos não sobrepostos de uma mesma onda que se completam.}%\footnotemark}
\end{figure}
%\footnotetext{Fonte: note}

\subsubsection{Buraco negro}
Assim como a gravidade o buraco negro é um aspecto probabilístico da distribuição amostral de uma população, como previsto pelo Teorema Central do Limite. 

Quanto mais momentos lógicos, mais amostras, o que tende a centralizar cada vez mais amostras da consciência em uma proporção cada vez menor do intervalo lógico. Essa proporção do intervalo lógico cada vez menor tende ao infinito assim como a quantidade de amostras crescentes que ela envolve, ou seja, um alto volume de amostras em uma proporção inobservável a certas abstrações de consciência. O buraco negro é uma concentração muito alta de amostras, uma altíssima frequência de amostras em um intervalo lógico extremamente pequeno, que muito provavelmente, em um passado distante dessa consciência esteve muito próximo ao centro dessa consciência, onde mesmo com a evolução da consciência em alguns níveis capaz de tornar o intervalo do buraco negro distante de seu centro ainda é incapaz de observar uma frequência tão alta em um intervalo lógico tão pequeno. Outro fator que contribui para essa dificuldade na observação desse pequeno intervalo lógico e que a medida que ele se afasta da mediana menos relevante à consciência ele se torna. Em suma, são picos de amostras em intervalos extremamente pequenos, que se afastam do centro da consciência conforme esta evolui e que ainda não são observados por ela.

\begin{figure}[H]
\caption{Buraco negro}
\label{fig:consciousness_black_hole}
\centering
\includegraphics[scale=1]{sections/images/consciousness_black_hole.jpg}
\floatfoot{Centralização infinita das amostras em uma proporção centralizada cada vez menor.}%\footnotemark}
\end{figure}
%\footnotetext{Fonte: note}