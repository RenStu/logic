% ----------------------------------------------------------
% Subseção Consciência
% ----------------------------------------------------------
\subsection{Consciência}
Um momento lógico pode ser formado por uma divisão (primeiro momento) ou por subdivisões lógicas (demais momentos).
	\begin{figure}[H]
	\caption{Intervalo lógico}
	\label{fig:consciousness_logical_moments}
	\centering
	\includegraphics[scale=.7]{sections/images/consciousness_logical_moments.jpg}
	\floatfoot{Exemplo de um intervalo lógico com dez momentos lógicos.}%\footnotemark}
	\end{figure}
	%\footnotetext{Fonte: note}

A consciência são os momentos lógicos de uma expansão representados em suas unidades.
	\begin{figure}[H]
	\caption{Intervalo lógico consciente}
	\label{fig:consciousness}
	\centering
	\includegraphics[scale=.7]{sections/images/consciousness.jpg}
	\floatfoot{Exemplo de um intervalo lógico consciente com dez unidades de momentos lógicos.}%\footnotemark}
	\end{figure}
	%\footnotetext{Fonte: note}

Pode ser observado na Tabela \ref{tab:10000_all} que a probabilidade de 99,99\% das amostras de uma população (Amostras do Range), que aumentam em quantidade à medida que crescem os momentos lógicos, tendem a estar cada vez mais ao centro do intervalo lógico, sendo que essa centralização tende ao infinito.
	\begin{figure}[H]
	\caption{Centralização de 99,99\% das amostras}
	\label{fig:centering_of_99_range}
	\centering
	\includegraphics[scale=1]{sections/images/centering_of_99_range.jpg}
	\floatfoot{Tendência de centralização do range de 99,99\% das amostras.}%\footnotemark}
	\end{figure}
	%\footnotetext{Fonte: note}

A consciência tende à representação de uma onda lógica, a maior onda lógica de uma população, um histograma da distribuição normal, conforme Figura \ref{fig:trend_chart_of_normal_distribution}. Todos os aspectos listados abaixo são inerentes a abstração lógica chamada consciência.

\subsubsection{Infinito}
Um dos aspectos mais importantes que a negação do nada traz (negação de si), é o infinito, ou seja, em qualquer intervalo lógico cabe o infinito novamente. A lógica primordial que iniciou todo o intervalo lógico é a mesma encontrada em seus intervalos subsequentes. Isso fundamenta como uma lógica de alto nível como a subconsciência humana explica a lógica primordial, uma vez que não é preciso voltar ao primeiro momento lógico do intervalo para deduzi-lo, pois esse fenômeno é onipresente em todo o intervalo.

\subsubsection{Ondas}
Probabilisticamente a distribuição de novas amostras de uma população tendem a concentrar mais amostras sentido a mediana da população com frequências de amostras cada vez maiores neste sentido. Porém, a distribuição dessas amostras com frequências de crescimento uniformes é infinitesimal se comparado às possibilidades randômicas desse crescimento. Assim, a tendência de crescimento dessas frequências sentido a mediana somadas a baixíssima probabilidade (infinitesimal) desse crescimento ser uniforme, conduz a frequências no padrão de ondas. A relação de densidade ou amplitude de uma onda com seu comprimento é detalhada subseção posterior.
	\begin{figure}[H]
	\caption{Padrão de onda}
	\label{fig:consciousness_waves}
	\centering
	\includegraphics[scale=.8]{sections/images/consciousness_waves.jpg}
	\floatfoot{Padrão de onda inferido pela tendência dessa distribuição com frequências maiores sentido a mediana da população e a baixíssima probabilidade de crescimento uniforme dessas frequências.}%\footnotemark}
	\end{figure}
	%\footnotetext{Fonte: note}

A junção de uma onda a outra elimina sua discrepância e faz com que essa onda deixe de existir a se tornar parte da primeira, que tem seu pico mais próximo da mediana. Uma onda não morre, apenas une-se com outra onda mais ao centro da população.
	\begin{figure}[H]
	\caption{Unificação de ondas}
	\label{fig:consciousness_uniform_wave}
	\centering
	\includegraphics[scale=1]{sections/images/consciousness_uniform_wave.jpg}
	\floatfoot{Ondas sendo unificadas para exemplificar o crescimento amostral uniforme.}%\footnotemark}
	\end{figure}
	%\footnotetext{Fonte: note}

\subsubsubsection{Comprimento e amplitude}
O histograma é utilizado nas figuras dessa subseção e posteriormente para facilitar a visualização e entendimento, pois representa muito bem a curva de densidade de uma população, conforme as diferentes visualizações da Figura \ref{fig:consciousness_wave_histogram} representando apenas um intervalo ou um comprimento de onda pareado pela mediana da população.  
	\begin{figure}[H]
	\caption{Histograma em diferentes visualizações }
	\label{fig:consciousness_wave_histogram}
	\centering
	\includegraphics[scale=.7]{sections/images/consciousness_wave_histogram.jpg}
	\floatfoot{Diferentes maneiras da representação populacional em histograma.}%\footnotemark}
	\end{figure}
	%\footnotetext{Fonte: note}}
O comprimento e amplitude de ondas estabelecem uma relação de quantidade por intervalo ou unidade. Essas unidades são estabelecidas pelo entrelaçamento de ondas, conforme subseção posterior. Assim, a amplitude é a densidade de um comprimento de onda, a densidade de um intervalo qualquer.  

Ao adicionar uma nova amostra na população todo o intervalo se distribui proporcionalmente para acoplar essa amostra, conforme Figura \ref{fig:consciousness_space_volume_amplitude}.
	\begin{figure}[H]
	\caption{Expansão do intervalo}
	\label{fig:consciousness_space_volume_amplitude}
	\centering
	\includegraphics[scale=.5]{sections/images/consciousness_space_volume_amplitude.jpg}
	\floatfoot{Expansão do intervalo com a adição de novas amostras.}%\footnotemark}
	\end{figure}
	%\footnotetext{Fonte: note}}

Outro fator importante é que as novas amostras tendem a serem mais distribuídas no pico do intervalo, provavelmente o local mais denso da onda. Na Figura \ref{fig:consciousness_space_amplitude_growth} o pico é representado na parte superior do subintervalo que compõe o pico da onda (porque é o intervalo mais denso que compõe o pico e porque a parte superior do intervalo está mais próxima de a mediana da população). No entanto, o pico pode estar em qualquer outro ponto nos subintervalos que compõem o pico de uma onda.
	\begin{figure}[H]
	\caption{Amplitude de onda - pico}
	\label{fig:consciousness_space_amplitude_growth}
	\centering
	\includegraphics[scale=.6]{sections/images/consciousness_space_amplitude_growth.jpg}
	\floatfoot{Tendência da maior concentração de amostras nos subintervalos de uma onda maior.}%\footnotemark}
	\end{figure}
	%\footnotetext{Fonte: note}}

Em grandes intervalos com muitos momentos lógicos é observado uma discrepância menor das amplitudes das ondas. Nesses intervalos podem ser observados grandes sistemas de objetos. Quanto maiores os intervalos mais equilibrados eles estarão crescendo sentido a mediana da população, probabilisticamente, conforme Figura \ref{fig:consciousness_space_subconsciousness}. A onda mais inferior, azul escuro, é a onda base do sistema, ou seja, a onda que formou as outras ondas. Os sistemas de ondas podem ser complexos, tendo várias ondas aninhadas. Intervalos mais complexos e com essa característica podem representar, por exemplo, o centro do universo, então o centro de uma galáxia, estrelas, planetas etc.
	\begin{figure}[H]
	\caption{Amplitude de ondas em grandes intervalos ou comprimentos}
	\label{fig:consciousness_space_subconsciousness}
	\centering
	\includegraphics[scale=.45]{sections/images/consciousness_space_subconsciousness.jpg}
	\floatfoot{Menor discrepância das ondas em grandes intervalos.}%\footnotemark}
	\end{figure}
	%\footnotetext{Fonte: note}

Em intervalos menores e com muitos momentos lógicos é observado uma discrepância maior das amplitudes das ondas. Nesses intervalos podem ser observados sistemas menores de objetos. Quanto menores os intervalos mais desequilibrados eles estarão crescendo sentido a mediana da população, probabilisticamente, conforme Figura \ref{fig:consciousness_space_subconsciousness_min}. A onda mais inferior, azul escuro, é a onda base do sistema, ou seja, a onda formadora de outras ondas. Os sistemas de ondas mais complexos e com essa característica podem representar, por exemplo, o átomo que são muito pequenos, se apresentam em enormes quantidades e as partículas que orbitam seu núcleo (elétrons) ficam bem mais distantes dele.
	\begin{figure}[H]
	\caption{Amplitude de ondas em pequenos intervalos ou comprimentos}
	\label{fig:consciousness_space_subconsciousness_min}
	\centering
	\includegraphics[scale=.45]{sections/images/consciousness_space_subconsciousness_min.jpg}
	\floatfoot{Alta discrepância das ondas em pequenos intervalos.}%\footnotemark}
	\end{figure}
	%\footnotetext{Fonte: note}}

\subsubsubsection{Entrelaçamento}
As amostras que mais se parecem em termos de frequências e distribuição são as amostras que fazem parte da mesma onda. Elas são frequências opostas não sobrepostas que se completam.

Probabilisticamente, as duas partes complementares de uma onda tendem a estar a uma distância aproximadamente iguais, equidistante da mediana, porém essa não é uma regra e as partes complementares de uma onda podem estar em distâncias diferentes em relação à mediana. O fenômeno da paridade das partes de uma onda tem o nome de entrelaçamento de ondas.

Esses pares tendem a serem formados pela probabilidade, onde comprimentos de ondas iguais detém a mesma probabilidade de distribuição de amostras em dois ou mais pontos diferentes da população. 

Intervalos com frequências temporais e distribuições espaciais parecidas são intervalos formados pela mesma unidade probabilística, ou seja, intervalos que têm o mesmo cenário ou contexto probabilístico em dado momento lógico. Por estarem no mesmo cenário probabilístico (unidades probabilísticas) esses intervalos têm suas amostras no mesmo cenário espaço-temporal, que é chamado de malha espaço-tempo e é formado pela maior unidade probabilística da população (todas as amostras da população intermediadas pela mediana). 

Esses entrelaçamentos formam ondas menores (subconsciências), semelhantes a maior onda do intervalo, comumente entrelaçada pela mediana da população, a consciência. A consciência é a lógica do intervalo, enquanto formam subconsciências ou sub-lógicas, como pequenas ondas de uma onda maior, sendo essas pequenas ondas semelhantes ao padrão da onda maior. Assim, uma mudança na onda maior (consciência) também é uma mudança na onda menor (subconsciência), mudança essa que é induzida pelas subconsciências indiretamente, análogo ao comprimir gás em um cilindro, onde ao adicionar uma nova molécula de gás no cilindro parcialmente cheio mais próximas ou apertas as moléculas dentro dele estarão. O contrário também é verdadeiro, uma nova amostra em uma subconsciência que por esta é observada diretamente é também uma mudança da consciência e vai ser induzida por outras subconsciências indiretamente, conforme Figura \ref{fig:consciousness_space_plan}.
	\begin{figure}[H]
	\caption{Subconsciência}
	\label{fig:consciousness_subconscious}
	\centering
	\includegraphics[scale=.8]{sections/images/consciousness_subconscious.jpg}
	\floatfoot{O padrão de ondas forma subconsciências semelhantes ao padrão criado pela consciência, como visto na Figura \ref{fig:trend_chart_of_normal_distribution}.}%\footnotemark}
	\end{figure}
	%\footnotetext{Fonte: note}
	
O entrelaçamento de ondas pode ocorrer em diferentes níveis ou intervalos, conforme visto na Figura \ref{fig:consciousness_subconscious_entanglement}, o que forma sistemas. As chavetas sem bordas (direita) identificam os intervalos os quais uma nova amostra despertou o salto, conforme visto na próxima subseção. Os arcos numerados indicam a ordem dos entrelaçamentos. Um entrelaçamento pode ocorrer de maneira equidistante da mediana não havendo o salto, como o primeiro entrelaçamento (violeta).

O maior entrelaçamento é mostrado nos exemplo da Figura \ref{fig:consciousness_subconscious_entanglement} como o primeiro entrelaçamento (violeta), ocorrido quando esse intervalo era o menor, provavelmente. Os grandes intervalos tendem a ser mantido ordenados pelas reordenações de seus subintervalos subsequentemente. A maior onda é comumente entrelaçada pela mediana da população. 

Os intervalos menores se entrelaçam primeiro, e estas reordenações provocadas por eles permitem a união de intervalos maiores. O encontro de dois intervalos já entrelaçados não implica um novo entrelaçamento, apenas a soma dessas ondas, pois já estão entrelaçadas.
	\begin{figure}[H]
	\caption{Níveis do entrelaçamento de ondas - comprimentos de ondas}
	\label{fig:consciousness_subconscious_entanglement}
	\centering
	\includegraphics[scale=.8]{sections/images/consciousness_subconscious_entanglement.jpg}
	\floatfoot{Exemplos dos níveis do entrelaçamento de ondas ou níveis dos comprimentos de ondas.}%\footnotemark}
	\end{figure}
	%\footnotetext{Fonte: note}

Os possíveis comprimentos de ondas de uma população são definidos por esses níveis de entrelaçamentos de ondas. Assim, independente da ordem dos saltos, níveis maiores de entrelaçamento são os comprimentos de ondas maiores e níveis menores os comprimentos menores, o que permite que ondas maiores tenham sub-ondas menores. 

Todo entrelaçamento é uma onda e o encontro de dois entrelaçamentos não acarreta um novo entrelaçamento, apenas a soma dessas ondas, pois estas já estão entrelaçadas.  

O entrelaçamento ocorre em intervalos bem pequenos. Uma vez entrelaçados, cada nova amostra pode causar movimento, a depender do ambiente mais ou menos rarefeito. Os intervalos maiores são formados por meio da soma de intervalos menores já entrelaçados através do movimento e pela adição de novas amostras.

O vigor de um entrelaçamento está na equivalência dos pares de uma onda ou na quantidade de subintervalos que uma grande onda está constantemente trocando com outra grande onda, seja com suas amostras já entrelaçadas ou principalmente com suas amostras não entrelaçadas (nuvem virtual – amostras não entrelaçadas), o qual permiti a reorganização dessas ondas de forma contínua, semelhante ao que acontece fortemente com os imãs, devido à baixa entropia, conforme subseção da força eletromagnética. Como exemplo, os entrelaçamentos das ondas maiores de um carro (peças do carro) podem ocorrer com vigor maior ao menor, como a fusão das peças ou solda, colas entre outros. Assim, o motor e as rodas de um carro ao acelerar e frear respectivamente, igualmente aceleram e desaceleram suas partes mais fortemente entrelaçadas, já a partes fracamente entrelaçadas como objetos e ocupantes soltos em seu interior sofrem com a inércia. Com os imãs algo semelhante ocorre, uma vez que ao mover um imã também pode-se mover as partes mais fortemente entrelaçadas a ele. O entrelaçamento também ocorre em ondas afastadas, como os núcleos atômicos e seus elétrons, os núcleos dos sistemas estelares e seus planetas, os núcleos galácticos e suas estrelas entre outros. Os núcleos desses sistemas são as partes mais densas de suas ondas e mesmo distantes causam influências probabilísticas nos pequenos intervalos das ondas menores em seu interior, o que proporciona os entrelaçamentos. 

\subsubsubsection{Salto}
O salto é uma reordenação feita pelo entrelaçamento de ondas à medida em que as amostras dos pares entrelaçados deixam de ser equivalentes com a adição de novas amostras em um dos lados do par. O salto ocorre em uma das partes do par de uma onda e é uma reordenação, ou seja, tanto a parte do intervalo que acabou de receber a nova amostra deve melhor se adequar ao intervalo pretendido ao salto quanto o contrário.

Na Figura \ref{fig:consciousness_space_subconscious_observation_jump} é observado os entrelaçamento de ondas (representadas por colunas de um histograma para facilitar a visualização do intervalo). A reordenação feita pelo entrelaçamento provoca um salto nas coordenadas (X, Y e Z) conforme subseção do Espaço.
	\begin{figure}[H]
	\caption{Reordenação - Salto}
	\label{fig:consciousness_space_subconscious_observation_jump}
	\centering
	\includegraphics[scale=.53]{sections/images/consciousness_space_subconscious_observation_jump.jpg}
	\floatfoot{Salto provocado pela não equivalência do par entrelaçado com a adição de novas em um de seus lados.}%\footnotemark}
	\end{figure}
	%\footnotetext{Fonte: note}

Como exemplo, um fóton ao entrar no intervalo do elétron pode desequilibrar um dos lados do par entrelaçado do elétron que o faz saltar, porém como o intervalo do elétron é pequeno e o fóton é rápido (por ser ainda menor) ele sai rapidamente do intervalo do elétron que fica desequilibrado novamente e retorna para o nível de energia equivalente ao anterior ao salto.

\subsubsection{Tempo}
O tempo é a adição de novos momento lógicos entre momentos existentes à medida que prossegue a negação de si da lógica. Essas mudanças são acumulativas e a medida que aumentam o número desses momentos lógicos, menos relevante cada novo momento será dentro do intervalo consciente. Um em cem é mais relevante do que um em mil. 
	\begin{figure}[H]
	\caption{Tempo}
	\label{fig:consciousness_time}
	\centering
	\includegraphics[scale=.8]{sections/images/consciousness_time.jpg}
	\floatfoot{Progressão do tempo conforme os momentos lógicos avançam.}%\footnotemark}
	\end{figure}
	%\footnotetext{Fonte: note}

O tempo também passa em uma onda quando ela recebe amostras por meio do movimento de entrada de ondas menores, de acordo com a próxima subsecção do Espaço.

Na introdução desse artigo foi apresentado que a lógica é uma sequência de negações de si no tempo zero, ou seja, em nenhum momento entre suas negações a lógica passa a \underline{SER}, garantindo a premissa primordial da constante lógica, \underline{NÃO SER}. Assim, a lógica é uma sequência infinita, simultânea e generalizada, uma constante. Na experiência do tempo conduzida pelo observador a ordenação da sequência é a essência dessa grandeza e, portanto, mais relevante do que sua origem que é de natureza simultânea, o qual transcende o tempo.

Cada população tem uma ordem diferente em sua sequência e é essa ordem que dá origem à grandeza que chamamos de tempo. É essa ordem do universo ou da consciência que vai dar a noção do que acontece antes ou depois, ou seja, o passado, o presente e as prospecções futuras.

Na expansão generalizada de todos os infinitos universos paralelos é possível observar que o NADA é uma Lógica constante (como visto na seção inicial - Lógica, a seção genitora de todas as demais nesse artigo) e a negação de si de lógica remete ao tempo somente quando observado os fluxos universais separadamente (expansão não generalizada de apenas um universo). Fluxos temporais de uma população podem ter o mesmo início e até a mesma expansão temporal, infinitamente, a partir da primeira negação da expansão. Desta forma, existem infinitos fluxos temporais na mesma expansão populacional em diferentes pontos dela, conforme expansão em destaque na Figura \ref{fig:consciousness_constant_time}. Logo, não faz sentido associar a quantidade de amostras ou idade do fluxo temporal de um universo particular com a idade da constante lógica \underline{NÃO SER}, a qual precede o tempo. O movimento, mudança ou tempo são versões menores daquilo que é constante e total.
	\begin{figure}[H]
	\caption{Constante lógica - fluxos de tempo da população}
	\label{fig:consciousness_constant_time}
	\centering
	\includegraphics[scale=.6]{sections/images/consciousness_constant_time.jpg}
	\floatfoot{Fonte: Twitter, 2022. \protect\footnotemark}
	\end{figure}
	\footnotetext{\url{https://twitter.com/akiyoshikitaoka/status/851375952051421184}}

Outro fator importante ao observar o tempo (o observador é mais detalhado na subseção da consciência – Observador e a vida) é que, probabilisticamente, subconsciências ou intervalos mais próximos da mediana da população terão uma adição maior de novas amostras em seus intervalos, o que são observados diretamente por essas subconsciências. Por outro lado, subconsciências distantes da mediana da população terão uma adição menor de amostras em seus intervalos e sujeitam-se a um número maior de mudança induzidas indiretamente, conforme Figura \ref{fig:consciousness_subconscious}. Esse fenômeno de observação temporal proporcionado pela probabilidade de distribuição da população evita o paradoxo dos gêmeos \cite{brasilescola_paradoxo_gemeos}.

As prospecções de futuro do observador fundamentam-se na probabilidade de distribuição da população e, portanto, da distribuição probabilística de cada subintervalo dela. Logo, o universo tende a ser probabilístico ainda que aleatório em níveis de detalhes, o que faz os eventos serem inusitados ainda que preditos em algum nível, conforme as Figuras \ref{fig:consciousness_logical_moments} e \ref{fig:consciousness}. 

\subsubsection{Espaço}
Na Figura \ref{fig:consciousness_space_waves}, é exibida a densidade de amostras de uma população, onde os pares que tendem a mesma distribuição probabilística são colocados lado a lodo e representados em forma de histograma. A formação desses pares é proveniente do entrelaçamento de ondas.
	\begin{figure}[H]
	\caption{Pares entrelaçados representados em três dimensões espaciais}
	\label{fig:consciousness_space_waves}
	\centering
	\includegraphics[scale=.7]{sections/images/consciousness_space_waves.jpg}
	\floatfoot{Exemplo de ondas entrelaçadas, representadas em forma de histograma e obtidas pelo algoritmo Logic\_WavePattern. \footnotemark}
	\end{figure}
	\footnotetext{O algoritmo Logic\_WavePattern pode ser visto no Apêndice \ref{app:algoritmos}.}

A área cresce de forma quadrática ao crescimento da amplitude de uma onda (colunas do histograma), uma vez que o salto provocado pelo entrelaçamento de ondas e a própria distribuição probabilística das amostras do intervalo tendem a manter um crescimento equivalente nos pares que formam uma onda. E esse aspecto configura a lei do inverso do quadrado, que será mais aprofundada na subseção da Força gravitacional.

Ao representar as grandezas espaciais do gráfico da Figura \ref{fig:consciousness_space_waves} em um gráfico de distribuição 3D e distribuir seus pontos de extremidade (desprezando seus volumes e possíveis pontos internos), obtém-se algo parecido com uma espiral (como redemoinhos no ar ou na água) mesmo em volumes muito pequenos de dados (poucos momentos lógicos), conforme Figuras \ref{fig:consciousness_space_3DScatter15000-10} e \ref{fig:consciousness_space_3DScatter_200000-2}. Os pontos tendem a se moverem em forma de espiral, aproximadamente, conforme mostra a subseção posterior. 
	\begin{figure}[H]
	\centering
		\begin{subfigure}[H]{0.47\linewidth}
		\centering
		\includegraphics[width=1\linewidth]{sections/images/consciousness_space_3DScatter15000-10.jpg}
		\caption{15.000 amostras ou momentos}
		\label{fig:consciousness_space_3DScatter15000-10}
		\end{subfigure}
	
		\begin{subfigure}[H]{0.47\linewidth}
		\centering
		\includegraphics[width=1\linewidth]{sections/images/consciousness_space_3DScatter_200000-2.jpg}
		\caption{200.000 amostras ou momentos}
		\label{fig:consciousness_space_3DScatter_200000-2}
		\end{subfigure}%
	\caption{Gráficos de dispersão 3D gerados com pontos semelhantes aos da Figura \ref{fig:consciousness_space_waves}}
	\floatfoot{O histograma no padrão de ondas e os dados para gerar os gráficos de dispersão 3D podem ser obtidos com a execução do algoritimo Logic\_WavePattern. \protect\footnotemark}
	\end{figure}
	\footnotetext{O algoritmo Logic\_WavePattern pode ser visto no Apêndice \ref{app:algoritmos} e os gráficos de dispersão 3D podem ser acessados em: \url{https://chart-studio.plot.ly/create/?fid=ren.stuchi:5&fid=ren.stuchi:4} e \url{https://chart-studio.plot.ly/create/?fid=ren.stuchi:7&fid=ren.stuchi:6}}

Probabilisticamente, a grande concentração das amostras de uma população está em seu pico, sentido a mediana da população. Assim, devido à altas concentrações probabilísticas de amostras em intervalos cada vez menores de uma onda, o pico irá ocupar um subintervalo proporcional cada vez menor dentro da população, conforme observado na Figura \ref{fig:consciousness_flat_universe}. A figura \ref{fig:total_comparison_chart_with_99_range} é baseada na Tabela \ref{tab:10000_all} e também demonstra esta característica que dentro do pico da população pode demonstrar um universo aproximadamente homogéneo e plano na sua distribuição, mesmo que as suas amostras tendam para a linha de referência.   
	\begin{figure}[H]
	\caption{Universo plano}
	\label{fig:consciousness_flat_universe}
	\centering
	\includegraphics[scale=.6]{sections/images/consciousness_flat_universe.jpg}
	\floatfoot{Concentração de 99\% das amostras.}%\footnotemark}
	\end{figure}
	%\footnotetext{Fonte: note}

\textbf{Obviamente a representação e os movimentos do intervalo ou seus subintervalos entrelaçados não podem ser representados fielmente em 1 ou 2 dimensões, pois o entrelaçamento é essência inerente (inseparável) das 3 dimensões.} 

Cada nova amostra é tempo e também espaço (movimento ou mudança). Cada nova amostra adicionada dentro de um subintervalo fará este se movimentar conforme a distribuição de suas novas amostras. O intervalo que contém subintervalos pode se movimentar em qualquer direção, porém da mesma forma que as amostras em uma distribuição em um 1D (eixo Z da Figura \ref{fig:consciousness_space_plan}) são concentras na parte de maior valor do plano, em 2D ou 3D é análogo e ocorre o mesmo, conforme exibido no subintervalo da Figura \ref{fig:consciousness_space_plan}. Tanto o intervalo quanto os subintervalos tendem a ter suas maiores concentrações de amostras sentido a sua linha de referência interna e das linhas de referências dos seus intervalos superiores. Isso faz com que algo aproximado com a representação dos histogramas vá se formando naturalmente. O intervalo e seus subintervalos têm seus tamanhos em X, Y e Z proporcionais a seus tamanhos dentro da população em representação 1D (eixo Z da Figura \ref{fig:consciousness_space_plan}), logo suas escalas internas estão relacionadas com a quantidade de amostras que eles têm.
	\begin{figure}[H]
	\caption{Distribuição espacial e movimento}
	\label{fig:consciousness_space_plan}
	\centering
	\includegraphics[scale=.6]{sections/images/consciousness_space_plan.jpg}
	\floatfoot{Distribuição espacial e movimento dos subintervalos e do intervalo populacional.}%\footnotemark}
	\end{figure}
	%\footnotetext{Fonte: note}
	
Uma nova amostra em um subintervalo movimenta este e seus intervalos superiores, pois uma mudança em um subintervalo também é uma mudança em seus intervalos superiores, conforme o vigor de seus entrelaçamentos, como visto na subseção do entrelaçamento. O movimento é continuo e suas escalas de tempo e espaço são referentes a unidade básica da população (1 amostra [espaço] a cada nova amostra da população [tempo]). Por exemplo, um subintervalo pode estar se movimentando em uma direção a 2 amostra (espaço) a cada nova amostra da população (tempo) e continuará nesse movimento até que receba duas amostras internas contrárias ou trombe com alguma amostra em seu movimento, em ambientes mais ou menos rarefeitos, que diminua sua velocidade ou o faça parar (leis da dinâmica). 

O entrelaçamento ocorre em intervalos bem pequenos e eles se formam na base de seus intervalos superiores, de acordo com o eixo Z. Uma vez entrelaçados, cada nova amostra pode causar movimento, a depender do ambiente mais ou menos rarefeito. Os intervalos maiores são formados por meio da soma de intervalos menores já entrelaçados através do movimento e pela adição de novas amostras, conforme subseção do Entrelaçamento. Assim os movimentos dos elétrons dentro átomo não é totalmente continuo, pois, os saltos causam novos entrelaçamentos que redefine suas posições, configurando camadas.

Um subintervalo pode sair naturalmente da gravitação de seu intervalo superior. Isso ocorre mais facilmente com intervalos bem pequenos e rápidos (muitas amostras concentradas em um pequeno intervalo - pico) favorecido por seu movimento num ambiente rarefeito devido ao tamanho. 

É muito difícil saber a posição do intervalo ou de um subintervalo olhando para uma representação em 1D, pois cada nova amostra movimenta o subintervalo e o intervalo e essa interação requer um cálculo para representação que levaria até uma representação naturalmente em 3D novamente.

Todo subintervalo ao ser entrelaçado surge na base de seu intervalo superior. Logo, a adição de novas amostras nesse subintervalo que acabou de surgir o fará subir no seu intervalo superior e esse é o sentido de todos os entrelaçamentos que não possui subintervalos, subir à medida que somam amostras e velocidade sentido a linha de referência. Porém o ambiente pode não ser rarefeito, o que dificulta esse movimento (podendo o manter parado a depender do quão denso é esse ambiente). Assim, na Figura \ref{fig:consciousness_space_plan_nosubinterval} é exibido um intervalo que ainda não contém subintervalos, portanto, ele sobe e ganha velocidade a cada nova amostra que ele recebe e sua subida vai ser centralizada se suas amostras são distribuídas uniformemente ou com um pico centralizado, ou será inclinada para a direita ou esquerda à medida que a concentração maior de amostras estiver mais de um lado do que do outro (é mais comum que a concentração de amostras esteja sentido a mediana da população).
	\begin{figure}[H]
	\caption{Intervalo sem subintervalos}
	\label{fig:consciousness_space_plan_nosubinterval}
	\centering
	\includegraphics[scale=.7]{sections/images/consciousness_space_plan_nosubinterval.jpg}
	\floatfoot{Movimentação de um intervalo sem subintervalos.}%\footnotemark}
	\end{figure}
	%\footnotetext{Fonte: note}
	
Os intervalos que possuem subintervalos podem se mover em qualquer sentido, uma vez que, seus subintervalos podem receber saltos e por meio desses novos entrelaçamentos a posição desses subintervalos são redefinidas para a base de seu intervalo superior e assim esse intervalo pode ter uma destruição de subintervalos em qualquer sentido, sejam eles mais densos ou menos densos, conforme o subintervalo da Figura \ref{fig:consciousness_space_plan}.

Na Figura \ref{fig:consciousness_space_matter_antimatter}, nos exemplos A, B e C é exemplificado o movimento das amostras comuns, matéria, e no exemplo D a antimatéria. Da mesma forma que a matéria comum, os subintervalos de um intervalo de antimatéria também deve ser composto em sua maioria por antimatéria. No exemplo A o pico do intervalo está a sua direita, o que provoca uma velocidade maior sentido a mediana da população. No exemplo B o pico está centralizado no intervalo, contudo suas amostras estão mais concentradas levemente sentido a mediana da população, o que provoca uma velocidade menor neste sentido e a depender do ponto de vista do observador e dos intervalos ao seu redor esse intervalo pode parecer estar indo para trás. No exemplo C pico do intervalo está a sua esquerda, o que provoca uma velocidade maior ao contrário da mediana da população e é mais comum em pequenos intervalos. O exemplo D é parecido com o C, porém crescimento dos subintervalos deste exemplo estão claramente no sentido contrário da mediana da população representando a antimatéria. 
	\begin{figure}[H]
	\caption{Movimentação em 1 dimensão - matéria e antimatéria}
	\label{fig:consciousness_space_matter_antimatter}
	\centering
	\includegraphics[scale=.7]{sections/images/consciousness_space_matter_antimatter.jpg}
	\floatfoot{Exemplos de movimentações em 1 dimensão.}%\footnotemark}
	\end{figure}
	%\footnotetext{Fonte: note}

\subsubsection{Forças fundamentais}
A força gravitacional, a força eletromagnética e a força nuclear correspondem às chamadas forças fundamentais da natureza. Essas forças fundamentais não são forças propriamente, mas sim aspectos probabilísticos de distribuição da população e do entrelaçamento de ondas.

\subsubsubsection{Força gravitacional}
A força gravitacional não é uma força propriamente e sim um aspecto da probabilidade de distribuição de novas amostras sentido a mediana da população, conforme teorema central do limite. Esse sentido probabilístico faz com que as ondas tenham um caminho provável a seguir dentro da população, ou seja, o pico de amostras da população ou o pico da maior onda da população. Da mesma maneira, fazem também com que as amostras dentro de um intervalo tenham um caminho provável a seguir, o pico de amostras do intervalo ou o pico da onda. Estes picos de amostras costumam ser a parte mais facilmente observáveis no intervalo de amostras desde ocupem uma área não tão pequena.

Na Figura \ref{fig:consciousness_gravitational_force} pode ser visto que a parte mais facilmente observável está levemente a direita no pico da onda. Essa onda tende a caminhar para cima e para a direita, em uma diagonal sentido ao pico do seu intervalo superior. 
	\begin{figure}[H]
	\caption{Força gravitacional}
	\label{fig:consciousness_gravitational_force}
	\centering
	\includegraphics[scale=.7]{sections/images/consciousness_gravitational_force.jpg}
	\floatfoot{Aspecto gravitacional, o sentido probabilístico da distribuição de novas amostras dentro de um intervalo.}%\footnotemark}
	\end{figure}
	%\footnotetext{Fonte: note}

Conforme visto na subseção de Amplitude de ondas, a área de um intervalo cresce de forma quadrática, uma vez que o salto provocado pelo entrelaçamento de ondas e a própria distribuição probabilística das amostras tendem a manter um crescimento equivalente nos pares que formam a onda. Esse aspecto configura a lei do inverso do quadrado, onde, no caso da gravidade, quando mais perto os objetos, maiores serão as chances probabilísticas das novas amostras do objeto menor ir em direção ao objeto maior (o pico da onda), que por estar dentro de uma área quadrada menor e por consequência de menor possibilidades de posicionamento das amostras, as chances desses objetos se aproximarem com uma quantidade bem menor de momentos lógicos aumenta muito. Assim, quanto mais longe os objetos, maior a área, maior as possibilidades de posicionamento e mais momentos lógicos são precisos para a aproximação, caracterizando assim uma atração menor. A probabilidade também pode afastar objetos mais rarefeitos que devem estar mais afastados da parte mais facilmente observável e densa de amostras, como no caso do gás hélio, por exemplo. A distribuição de novas amostras nos intervalos rarefeito são mais lentas (caso contrário não seriam rarefeito) do que nas partículas mais densas que tomam a frete dessas partículas menos densas afastando-as do pico da onda. 

\subsubsubsection{Força eletromagnética}
A força eletromagnética não é uma força propriamente e sim um aspecto do entrelaçamento de ondas que se intensifica em intervalos ou comprimentos de ondas com baixa entropia e com a aproximação espacial (redução de diferenças nos eixos X, Y e Z) desses intervalos.

O eletromagnetismo está relacionado à intervalos semelhantes ao lado da onda mais uniforme encontrada exemplo da Figura \ref{fig:consciousness_gravitational_force} (direito), porém com baixa entropia, ou seja, a mesma estrutura que facilita o movimento dos objetos somado a baixa entropia, a qual facilita os saltos. Quando os intervalos têm baixa entropia a aproximação desses, seja naturalmente pela estrutura que facilita o movimento ou pela distribuição de novas amostras capaz de criar essa estrutura como a eletrificação, faz com que os pares de ondas de um intervalo se pareça muito com os pares de ondas do outro intervalo, o que torna muito desses pares viáveis para que o entrelaçamento de ondas encontre pares mais ideais no outro intervalo e vice-versa. Desta forma, ocorre uma reordenação entre os intervalos por meio do entrelaçamento de ondas e essa reordenação torna esses intervalos mais equalizado (baixa entropia).

As linhas azuis da Figura \ref{fig:consciousness_electromaagnetic_force} mostra onde é mais frequente a troca dos pares de ondas pelo entrelaçamento de ondas, ou seja, onde se tem a maior probabilidade das ondas serem parecidas. Por isso os imãs tentam se virar para se conectar quando estão face a face com o mesmo polo. A linha cinza mostra as conexões que ocorrem em número bem menor.
	\begin{figure}[H]
	\caption{Força eletromagnética}
	\label{fig:consciousness_electromaagnetic_force}
	\centering
	\includegraphics[scale=.7]{sections/images/consciousness_electromaagnetic_force.jpg}
	\floatfoot{Aumento das possibilidades de entrelaçamento de ondas devida a equalização probabilística em objetos próximos e de baixa entropia.}%\footnotemark}
	\end{figure}
	%\footnotetext{Fonte: note}

A Figura \ref{fig:consciousness_electromaagnetic_force_entropy} mostra um exemplo de baixa entropia. 
	\begin{figure}[H]
	\caption{Força eletromagnética - entropia}
	\label{fig:consciousness_electromaagnetic_force_entropy}
	\centering
	\includegraphics[scale=.9]{sections/images/consciousness_electromaagnetic_force_entropy.jpg}
	\floatfoot{Aumento das possibilidades de entrelaçamento de ondas devido à baixa entropia.}%\footnotemark}
	\end{figure}
	%\footnotetext{Fonte: note}

O aspecto eletromagnético está intimamente relacionado com a baixa entropia de um intervalo e a possibilidade de entrelaçamento de seus pares com os pares ao redor. A baixa entropia de um intervalo indica que suas amostras estão em uma ordem qualquer em seu interior.

Probabilisticamente, os pares de ondas mais parecidos estão nas regiões mais próximas (linhas azuis do Figura \ref{fig:consciousness_electromaagnetic_force}). Isso ocorre devido ao crescimento do número de amostras sentido a mediana da população, porém não é regra e os polos podem se inverter, ou seja, ter mais ligações com a região de menor probabilidade, ainda que a maior parte dos pares que compõem essa região estejam de forma crescente sentido a mediana.

\subsubsubsection{força nuclear}
Os mesmos aspectos probabilísticos que regem a gravidade e que podem ser vistos na Figura \ref{fig:consciousness_gravitational_force} também regem as chamadas forças nucleares. A diferença é que nas forças nucleares os intervalos são menores possibilitando uma quantidade muito maior de saltos e suas ondas são mais discrepantes, conforme mostra a Figura \ref{fig:consciousness_space_subconsciousness_min}.

As forças nucleares forte e fraca representam grandes concentrações de momentos lógicos por intervalo populacional, uma alta densidade em um pequeno intervalo. A grande concentração dessas amostras está no pico do intervalo, que ocupa um subintervalo cada vez menor dentro da onda (proporcionalmente), devido à alta concentração de amostras em intervalos cada vez menores, conforme Figura \ref{fig:total_comparison_chart_with_99_range}.

A penetração desses intervalos pequenos e densos por uma quantidade excessiva de momentos lógicos (outro intervalo semelhante), em um curto período, faz com que os inúmeros pares de seus subintervalos oscilem potencializando os saltos. Dessa forma os subintervalos saltam de forma continua, progressiva e rapidamente até que a probabilidade de distribuição da população normalize todo o intervalo posteriormente. Junto dos saltos que irão provocar movimentos em grande número de partículas ou intervalos ao redor, há um enorme número de atritos em velocidades tremendas provocados pelas colisões dessas pequenas partículas que provocam grandes ondas de choque. 

Em intervalos menores de amostras, como os subintervalos do núcleo atômico, pode ser mais comum que a energia se comporte como ondas ou nuvens mais ou menos densas, sem um pico definido e visível, devido a alto discrepância de seus subintervalos aninhados.

\subsubsection{Espiral e órbita}
Como as coordenadas X, Y e Z dos pares emaranhados de uma população tendem a aumentar, a disposição dessas em um sistema tridimensional de coordenadas vai seguir uma referência diagonal entre esses três eixos, conforme Figura \ref{fig:consciousness_space_spiral_reference_line}. O padrão de espiral observado não invalida outros possíveis movimentos no espaço. Muitas vezes não é possível observar o padrão de espiral imediatamente nos movimentos de um intervalo (subintervalo), porém esse padrão está por traz de muitos destes movimentos. Ao pegar os movimentos humanos, como exemplo, tem-se os ciclos predominantes de ir e voltar para casa, ir e voltar ao trabalho, acordar e dormir, ou seja, os hábitos se assemelham a movimentos em ciclos, movimentos espirais.
	\begin{figure}[H]
	\caption{Sistema tridimensional de coordenadas}
	\label{fig:consciousness_space_spiral_reference_line}
	\centering
	\includegraphics[scale=.7]{sections/images/consciousness_space_spiral_reference_line.jpg}
	\floatfoot{Linha de referência probabilística para distribuição de uma população em um plano tridimensional.}%\footnotemark}
	\end{figure}
	%\footnotetext{Fonte: note}}

Na Figura \ref{fig:consciousness_space_spiral_reference_line} também podem ser observado os pontos X1 e X2. Esses pontos foram espelhados nas coordenadas X e Y para facilitar a observação de que mesmo na parte inferior da espiral o intervalo continua a somar amostras, ainda que em menor quantidade do que quanto subindo para a parte superior da espiral. A linhas tracejadas mostram os caminhos mais prováveis para os intervalos A e B. Dessa forma, quando uma parte do intervalo está em seu ponto médio máximo (eixos X e/ou Y) a tendência probabilística é que ele receba menos amostras do que a parte do intervalo que está em seu ponto médio mínimo. Esse efeito espiral é mais notável quanto maior for um intervalo e sua quantidade de amostras, pois mais prováveis e estáveis serão esses caminhos.

O movimento contínuo num ambiente rarefeito também ajuda na formação e manutenção das espirais. À medida que as amostras são adicionadas aos subintervalos, as suas velocidades tendem a aumentam em direção à linha de referência, e como esta adição não é uniforme (variando entre picos e vales) e o movimento é aproximadamente contínuo, os subintervalos podem derivar ou deslizar de um lado da linha de referência para o outro.

Cada intervalo ou subintervalo (comprimento de ondas) tem sua própria linha de referência. Assim como dentro de um metro existem os centímetros, milímetros etc., dentro de um intervalo e subintervalos podem existir inúmeros outros, conforme exibido abaixo.
	\begin{figure}[H]
	\caption{Intervalos e linhas de referências}
	\label{fig:consciousness_space_spiral_underlines}
	\centering
	\includegraphics[scale=.5]{sections/images/consciousness_space_spiral_underlines.jpg}
	\floatfoot{Espirais em diferentes intervalos e suas linhas de referências.}%\footnotemark}
	\end{figure}
	%\footnotetext{Fonte: note}}

\subsubsubsection{Órbitas}
Órbita pode ser definida nesse estudo como o conceito das espirais somada a orientação de um pico probabilístico (gravidade) ao invés da linha de referência das espirais, apenas.

Os sistemas que orbitam como descrito anteriormente (em espiral - orientada pela linha de referência) são sistemas ou intervalos em que seu pico está subdividido em subintervalos e não formam um centro de gravidade, ou seja, todos os subintervalos do pico não estão concentrados em um ponto do sistema, orbitando dessa forma a linha de referência do intervalo. Provavelmente os aglomerados de galáxia, e os superaglomerados sejam exemplos dessa orbita. A orbita espiral (orientada pela linha de referência) não está restrita a sistemas grandes, essa orbita é uma característica que pode acontecer em qualquer tamanho de intervalo.

Outro tipo de órbita é definido quando os subintervalos que orbitam o pico da onda (que representa aproximadamente 99,9\% das amostras do intervalo) diminuem sua velocidade de orbita à medida que se afastam do pico. Na Figura \ref{fig:consciousness_elliptical_orbit_system} as colunas do histograma em azul representam o pico da onda. Essa diminuição de velocidade ocorre gradualmente a medida que esses intervalos em cinza se afastam do pico da onda, recebendo assim uma quantidade menor de amostras, diminuindo sua aceleração. O sistema solar é possivelmente um exemplo desse tipo de órbita. As órbitas atômicas também podem se assemelhar a esse tipo de órbita devido as diferenças de energias entre as camadas estruturadas pelos saltos.  
	\begin{figure}[H]
	\caption{Órbitas dos subintervalos fora do pico da onda}
	\label{fig:consciousness_elliptical_orbit_system}
	\centering
	\includegraphics[scale=.7]{sections/images/consciousness_elliptical_orbit_system.jpg}
	\floatfoot{Os subintervalos diminuem de velocidade conforme se afastam do pico da onda.}%\footnotemark}
	\end{figure}
	%\footnotetext{Fonte: note}

Ainda outro tipo de órbita é definido quando os subintervalos que orbitam o pico da onda mantêm uma velocidade média constante, independente da distância do pico. Isso ocorre porque esses subintervalos também fazem parte do pico da onda em azul, conforme Figura \ref{fig:consciousness_circular_orbit_system}. Assim, por estes subintervalos permanecerem em órbita dentro do subintervalo de 99,9\% da onda, suas velocidades não diminuem. Esses 99,9\% da onda é parte mais facilmente visível, portanto, a parte observada das galáxias, provavelmente. Talvez essa característica também seja responsável pelos anéis dos planetas. 
	\begin{figure}[H]
	\caption{Órbitas dos subintervalos dentro do pico da onda}
	\label{fig:consciousness_circular_orbit_system}
	\centering
	\includegraphics[scale=.9]{sections/images/consciousness_circular_orbit_system.jpg}
	\floatfoot{Os subintervalos mantêm a velocidade conforme se afastam do pico da onda.}%\footnotemark}
	\end{figure}
	%\footnotetext{Fonte: note}

Para a distância dos subintervalos em órbita a gravidade exerce mais uma orientação do que uma atração. Porém como o movimento é praticamente continuo em ambientes rarefeitos e essa orientação é permanente, as órbitas vão se formando e sendo mantidas pela velocidade crescente desses subintervalos, que tende a afastá-los. 

Quanto mais distantes as órbitas estiverem do pico, mais uniformemente suas amostras serão distribuídas internamente. Quando a atração gravitacional é mais forte as amostras são mais intensamente distribuídas nesta direção. Essa distribuição mais uniforme pode influenciar os relógios atômicos, o corpo humano, etc., pois cada amostra influencia o tempo e o espaço.

Conforme afirmado na subseção do espaço, uma nova amostra em um subintervalo o move e seus intervalos superiores. Também novas amostras no intervalo superior movem os intervalos inferiores, pois são frações da mesma onda. Mesmo em ambiente denso os intervalos podem se mover normalmente por meio do movimento de seus intervalos superiores (que são compostos pelo movimento de seus subintervalos – posição que ocupam dentro de seu intervalo superior) em ambientes rarefeito.  Desta forma, uma onda inferior permanecerá na onda superior quando sua velocidade for menor que a velocidade de pico da onda superior e sua trajetória não for oposta à trajetória da onda superior. O movimento da onda superior arrasta seus subintervalos, que com velocidades probabilísticas mais baixas continuarão a se mover dentro de sua onda superior, pois mesmo com velocidades mais baixas são arrastados pelo movimento da onda superior, que também é composta pelo movimento de seus subintervalos.

Dobrar a quantidade de amostras de um intervalo não irá dobrar definitivamente a velocidade de um intervalo, pois se tem o dobro de amostras para mover tornando inalterada a velocidade. Porém, probabilisticamente, quanto maior a quantidade de amostras de uma onda, mais essas amostras estarão sentido à mediana da população e menos relevantes se tornam as amostras contrárias, o que torna a velocidade probabilística dos picos de ondas superiores à velocidade de seus subintervalos. Isso não é uma regra e um subintervalo pode ter velocidade superior ao pico de sua onda superior ou seu movimento pode ser contrário ao movimento do pico da onda superior, fazendo com que ele saia naturalmente da gravitação de sua onda superior (e isso ocorre mais facilmente com intervalos bem pequenos e rápidos favorecido por seu movimento num ambiente rarefeito devido ao tamanho). 

\subsubsubsection{Matéria escura e energia escura}
Os intervalos naturalmente se afastam com velocidades crescentes por receberem mais amostras em direção à linha de referência ou pico da onda de seus intervalos superiores e por se moverem em um ambiente rarefeito.

As espirais podem ocorrer sem a necessidade de um pico de onda concentrado, mas quando há tal concentração no pico da onda, as órbitas podem ocorrer dentro ou fora dos 99,9\% das amostras do pico, resultando em órbitas que mantêm suas velocidades médias constantes quando longe do pico (dentro dos 99,9\%) ou órbitas que diminuem suas velocidades quando estão longe do pico (fora dos 99,9\%).

Portanto, matéria escura e energia escura não são matéria nem energia, mas aspectos probabilísticos da distribuição de amostras em uma população que se assemelha à distribuição normal.

\subsubsection{Antimatéria}
Quando um intervalo tende a concentrar suas amostras sentido da mediana, o que é o sentido provável conforme teorema central do limite, dá-se o nome de matéria. A antimatéria é o contrário, quando um intervalo tende a concentrar suas amostras no sentido oposto à mediana. 

A maneira mais simples de visualizar o sentido probabilístico das amostras de qualquer comprimento de onda é observar a \textbf{linha de referência probabilística}, conforme exibido na Figura \ref{fig:consciousness_space_spiral_reference_line}. Quanto maior a quantidade de amostra de um intervalo maior será sua tendência probabilística sentido a mediana da população.

Na Figura \ref{fig:consciousness_concentration_of_opposite_samples} é exibido dois intervalos idênticos com suas amostras em concentrações opostas.
	\begin{figure}[H]
	\caption{Parte de um intervalo idêntico com suas concentrações de amostras opostas}
	\label{fig:consciousness_concentration_of_opposite_samples}
	\centering
	\includegraphics[scale=1.2]{sections/images/consciousness_concentration_of_opposite_samples.jpg}
	\floatfoot{Parte de um intervalo idêntico distribuídos de formas opostas.}%\footnotemark}
	\end{figure}
	%\footnotetext{Fonte: note}

O merge ou soma dos intervalos opostos da Figura \ref{fig:consciousness_concentration_of_opposite_samples} os tornaria um intervalo simétrico, ou seja, não estaria em nenhum dos sentidos.
Na Figura \ref{fig:consciousness_concentration_of_opposite_samples_within_range} é exibido uma população com suas concentrações de amostras sentido à mediana e outra com suas concentrações sentido às bordas do intervalo.
	\begin{figure}[H]
	\caption{Populações com suas concentrações de amostras opostas}
	\label{fig:consciousness_concentration_of_opposite_samples_within_range}
	\centering
	\includegraphics[scale=.7]{sections/images/consciousness_concentration_of_opposite_samples_within_range.jpg}
	\floatfoot{Populações distribuídas em sentidos contrários.}%\footnotemark}
	\end{figure}
	%\footnotetext{Fonte: note}

\subsubsection{Buraco negro}
A área de um intervalo cresce à medida que novas amostras são adicionadas na população, conforme Figura \ref{fig:consciousness_space_volume_amplitude}, e com a soma de outros intervalos já entrelaçados, como mostra o subintervalo superior direito na Figura \ref{fig:consciousness_interval_contraction}. Contudo, os intervalos que recebem uma grande quantidade de amostras dentro da população tendem a formar cada vez mais subintervalos a diferentes níveis, que contraem a área ocupada destes intervalos (devido ao entrelaçamento de ondas do subintervalos menores sustentar essas áreas aproximadamente quadráticas), como mostra os subintervalos roxos e azuis na Figura \ref{fig:consciousness_interval_contraction}. Assim, essa relação do tamanho do intervalo com sua quantidade de amostras e os seus subintervalos é o aspecto que pode ajudar a descrever os chamados buracos negros, onde o pico da onda irá ocupar um subintervalo proporcional cada vez menor dentro do intervalo da onda, mesmo com uma concentração de amostras crescentes. Esses picos são frequentemente encontrados do meio para frente de um intervalo ou sistema (o núcleo ou pico do sistema).
	\begin{figure}[H]
	\caption{Contração do intervalo}
	\label{fig:consciousness_interval_contraction}
	\centering
	\includegraphics[scale=.5]{sections/images/consciousness_interval_contraction.jpg}
	\floatfoot{Relação do tamanho do intervalo e sua possível contração à medida que surgem novos subintervalos.}%\footnotemark}
	\end{figure}
	%\footnotetext{Fonte: note}
	
\subsubsection{Observador e a vida}
Os intervalos de ondas (comprimentos de ondas) que uma subconsciência (sub-lógica) é capaz de observar depende do comprimento de ondas que a própria subconsciência é constituída. Dentre todas as possibilidades de intervalos ou comprimento de ondas permitidos por uma população, o observador está em um deles.

A capacidade de comparar ou distinguir a ordem das mudanças de uma sequência amostral é a capacidade lógica de um observador, o observador do tempo (passado e presente). A velocidade dessa observação é dada pelo range que o observador é capaz de comparar, ou seja, o qual rápido ele for capaz de distinguir pequenas mudanças (poucas amostras) o fará perceber que mudanças maiores levam mais tempo (muitas amostras). 

A capacidade lógica de fazer prospecções probabilísticas, dentro das limitações lógicas do observador e com base na probabilidade da distribuição do intervalo ou subintervalo observado é a essência do pensamento e, portanto, da vida. Essas prospecções estão fundamentadas na probabilidade de distribuição de cada intervalo (no sentido do intervalo) e, portanto, estão relacionadas com a detecção de padrão e com possibilidades probabilísticas futuras.

A capacidade de comparar ou distinguir ondas lógicas, subconjuntos ou subconsciências, é a capacidade que define o sujeito (eu). A razoabilidade dessa definição depende da proporcionalidade dessa capacidade de comparação.

A vida \underline{NÃO É}, como qualquer outra lógica. Comumente, as formas mais notáveis de vida se multiplica por estarem na média probabilística do intervalo entre seus picos e vales, por mais diferente que sejam. Porém, algo muito discrepante ou diferente do padrão médio do intervalo tende a não multiplicar e permanecer.

\subsubsubsection{Sentidos}
A parte cognitiva de uma onda não observa a si mesmo diretamente e sim o exterior (a consciência – o todo) ou mais comumente uma parte dela (a subconsciência). Essa observação pode incluir o restante da onda a qual a parte cognitiva faz parte, que também é exterior da parte cognitiva e, portanto, uma subconsciência - parte da consciência. A parte cognitiva da subconsciência humana é, provavelmente, onde se tem o maior pico de ondas do subconjunto humano. Esse é o local onde é observado a maior intensidade de mudanças. Essas mudanças são caracterizadas pelo pensamento (observação e prospecção probabilística de um intervalo) que tende ao infinito (respeitando as limitações lógicas do observador), assim como a essência da lógica, o \underline{NÃO SER}. Ou seja, a parte cognitiva é a parte que está mais próxima da observação do todo, da lógica em sua essência e totalidade, da consciência.

A obtenção de amostras pelos sentidos dos seres humanos os modifica e essas ondulações funcionam como ajustes ou configurações. Cada sentido observa a população amostral de forma independente, como canais de frequências distintos. Assim a visão pode estar vendo objetos muito distantes e os ouvidos escutando sons bem próximos. Os sentidos são limitados pelas ondas que constituem o observador e sua capacidade máxima de observação está limitada na profundidade máxima de intervalos aninhados observados.

Uma característica importante do processo de observação de pequenos intervalos é que eles podem ser observados com partículas ou ondas. Na observação como partícula o observador acompanha um intervalo representado por um par entrelaçado, observando sua forma e movimento consistentes no espaço. No efeito partícula, a consistência da forma e seus movimentos são estabelecidas pelo par entrelaçado, visto que o salto ocorre em um lado do par de cada vez, garantido estabilidade nas mudanças. No intervalo observado como onda o observador acompanha uma das partes que compõe o par entrelaçado observando seus movimentos e saltos, uma vez que os saltos são frequentes em pequenos intervalos.

Talvez não seja possível observar o efeito onda sem entrelaçar seu par. A alta frequência desse intervalo faz com que ele ocupe ou transite rapidamente em uma área ao seu redor, o que pode facilitar o colapso da onda em um ponto especifico e então observar o seu efeito partícula (semelhante ao olho humano) ou em um local mais amplo e observar seu efeito onda com o colapso de muitas amostragens.
