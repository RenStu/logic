% ----------------------------------------------------------
% Subseção Consciência
% ----------------------------------------------------------
\subsection{Consciência}
Um momento lógico pode ser formado por uma divisão (primeiro momento) ou por subdivisões lógicas (demais momentos).
	\begin{figure}[H]
	\caption{Intervalo lógico}
	\label{fig:consciousness_logical_moments}
	\centering
	\includegraphics[scale=.7]{sections/images/consciousness_logical_moments.jpg}
	\floatfoot{Exemplo de um intervalo lógico com dez momentos lógicos.}%\footnotemark}
	\end{figure}
	%\footnotetext{Fonte: note}

A consciência são os momentos lógicos de uma expansão representados em suas unidades.
	\begin{figure}[H]
	\caption{Intervalo lógico consciente}
	\label{fig:consciousness}
	\centering
	\includegraphics[scale=.7]{sections/images/consciousness.jpg}
	\floatfoot{Exemplo de um intervalo lógico consciente com dez unidades de momentos lógicos.}%\footnotemark}
	\end{figure}
	%\footnotetext{Fonte: note}

Pode ser observado na Tabela \ref{tab:10000_all} que a probabilidade de 99,99\% das amostras (Amostras do Range), que aumentam em quantidade a medida que crescem os momentos lógicos, tendem a estar cada vez mais ao centro do intervalo lógico, sendo que essa centralização tende ao infinito.
	\begin{figure}[H]
	\caption{Centralização de 99,99\% das amostras}
	\label{fig:centering_of_99_range}
	\centering
	\includegraphics[scale=1]{sections/images/centering_of_99_range.jpg}
	\floatfoot{Tendência de centralização do range de 99,99\% das amostras.}%\footnotemark}
	\end{figure}
	%\footnotetext{Fonte: note}

A consciência tende à representação de um histograma da distribuição normal. Todos os aspectos listados abaixo são inerentes a abstração lógica chamada consciência.

\subsubsection{Infinito}
Um dos aspectos mais importantes que a negação do nada traz (negação de si), é o infinito, ou seja, em qualquer intervalo lógico cabe o infinito novamente. A lógica primordial que iniciou todo o intervalo lógico é a mesma encontrada em seus intervalos subsequentes. Isso fundamenta como uma lógica de alto nível como a subconsciência humana explica a lógica primordial, uma vez que não é preciso voltar ao primeiro momento lógico do intervalo para deduzi-lo, pois esse fenômeno é onipresente em todo o intervalo.

\subsubsection{Ondas}
Probabilisticamente a distribuição de novas amostras de uma população tendem a concentrar mais amostras sentido a mediana da população com frequências de amostras cada vez maiores neste sentido. Porém, a distribuição dessas amostras com frequências de crescimento uniformes é infinitesimal se comparado às possibilidades randômicas desse crescimento. Assim, a tendência de crescimento dessas frequências sentido a mediana somadas a baixíssima probabilidade (infinitesimal) desse crescimento ser uniforme, conduz a frequências no padrão de ondas. A relação de densidade ou amplitude de uma onda com seu comprimento é detalhada nas subseções do Espaço, Comprimento de onda e Amplitude de onda.
	\begin{figure}[H]
	\caption{Padrão de onda}
	\label{fig:consciousness_waves}
	\centering
	\includegraphics[scale=1]{sections/images/consciousness_waves.jpg}
	\floatfoot{Padrão de onda inferido pela tendência dessa distribuição com frequências maiores sentido a mediana da população e a baixíssima probabilidade de crescimento uniforme dessas frequências.}%\footnotemark}
	\end{figure}
	%\footnotetext{Fonte: note}

A junção de duas ondas além de eliminar suas discrepâncias, faz com que a primeira onda da união fique maior e a segunda onda acabe por deixar de existir a se tornar parte da primeira, que tem seu pico mais próximo da mediana. Probabilisticamente uma onda não morre, apenas une-se com outras ondas mais centrais a ela.
	\begin{figure}[H]
	\caption{Unificação de ondas}
	\label{fig:consciousness_uniform_wave}
	\centering
	\includegraphics[scale=1]{sections/images/consciousness_uniform_wave.jpg}
	\floatfoot{Ondas sendo unificadas para exemplificar o crescimento amostral uniforme.}%\footnotemark}
	\end{figure}
	%\footnotetext{Fonte: note}

\subsubsubsection{Entrelaçamento e subconsciente}
As amostras que mais se parecem em termos de frequências e distribuição são as amostras que fazem parte da mesma onda. Elas são frequências opostas não sobrepostas que se completam.

Probabilisticamente as duas partes complementares de uma onda estarão a uma distância aproximadamente iguais, equidistante da mediana, porém essa não é uma regra e as partes complementares de uma onda podem estar em distâncias diferentes da mediana. O fenômeno da paridade das partes de uma onda tem o nome de entrelaçamento de ondas.

Essas ondas formam subconsciências de uma consciência maior. A consciência é única para todo o intervalo, é a lógica do intervalo, enquanto formam subconsciências ou sub-lógicas, como pequenas ondas de uma onda maior. Assim, uma mudança na onda maior (consciência) também é uma mudança na onda menor (subconsciência), mudança essa que é induzida pelas subconsciências indiretamente, análogo ao comprimir gás em um cilindro, onde ao adicionar uma nova molécula de gás no cilindro parcialmente cheio, mais próximas ou apertas as moléculas dentro dele estarão. O contrário também é verdadeiro, uma nova amostra em uma subconsciência que por esta é observada diretamente é também uma mudança da consciência e vai ser induzida por outras subconsciências indiretamente.
	\begin{figure}[H]
	\caption{Subconsciência}
	\label{fig:consciousness_subconscious}
	\centering
	\includegraphics[scale=1]{sections/images/consciousness_subconscious.jpg}
	\floatfoot{O padrão de ondas forma subconsciências semelhantes ao padrão criado pela consciência (histograma de distribuição normal) como visto na Figura \ref{fig:statisticsbyjim_central_limit_theorem} ou na Figura \ref{fig:trend_chart_of_normal_distribution}.}%\footnotemark}
	\end{figure}
	%\footnotetext{Fonte: note}

\subsubsubsection{Salto}
O salto é uma reordenação feita pelo entrelaçamento de ondas a medida que as amostras do entrelaçamento deixam de ser equivalentes com a adição de novas amostras em seus lados.

Na Figura \ref{fig:consciousness_space_subconscious_observation_jump} é observado os entrelaçamento de ondas (representadas por colunas do histograma na vertical). A reordenação feita pelo entrelaçamento provoca um salto nas coordenadas (X, Y e Z) conforme subseção do Espaço.
	\begin{figure}[H]
	\caption{Reordenação subconsciente - Salto}
	\label{fig:consciousness_space_subconscious_observation_jump}
	\centering
	\includegraphics[scale=.6]{sections/images/consciousness_space_subconscious_observation_jump.jpg}
	\floatfoot{Salto provocado pela não equivalência do entrelaçamento com a adição de novas amostras.}%\footnotemark}
	\end{figure}
	%\footnotetext{Fonte: note}

A tendência probabilística é que, por exemplo, o elétron que saltou de sua orbita de origem retorne à esta conforme mais amostras são adicionadas ao entrelaçamento desse átomo, estabelecendo a normalidade probabilística.

\subsubsection{Tempo}
O tempo é a adição de novos momento lógicos entre momentos existentes à medida que prossegue a negação de si da lógica. Essas mudanças são acumulativas e a medida que aumentam o número desses momentos lógicos, menos relevante cada novo momento será dentro do intervalo consciente. Um em cem é mais relevante do que um em mil. 
	\begin{figure}[H]
	\caption{Tempo}
	\label{fig:consciousness_time}
	\centering
	\includegraphics[scale=.8]{sections/images/consciousness_time.jpg}
	\floatfoot{Progressão do tempo conforme os momentos lógicos avançam.}%\footnotemark}
	\end{figure}
	%\footnotetext{Fonte: note}

Outro fator importante a observar do tempo é que, probabilisticamente, subconsciências mais próximas da mediana da população terão uma adição maior de novas amostras em seus intervalos, o que são observados diretamente por essas subconsciências. Por outro lado, subconsciências distantes da mediana da população terão uma adição menor de amostras em seus intervalos e sujeitam-se a um número maior de mudança induzidas indiretamente. Esse fenômeno de observação temporal proporcionado pela consciência e subconsciências evita o paradoxo dos gêmeos \cite{brasilescola_paradoxo_gemeos}.

Na seção Expansão lógica foi apresentado que a lógica é uma sequência de negações de si no tempo zero, ou seja, em nenhum momento entre suas negações a lógica passa a SER, garantindo a premissa primordial da constante lógica, NÃO SER. Assim, a lógica é uma sequência infinita e simultânea, uma constante. Logo, o tempo é apenas uma grandeza da consciência oriunda da ordenação dessa sequência lógica, não da sequência propriamente. A simultaneidade dessa sequência torna a lógica uma constante com todas as suas infinitas possibilidades, sendo esse universo uma delas. 

Cada universo tem uma ordem diferente em sua sequência e é essa ordem que dá origem à grandeza que chamamos de tempo. É essa ordem do universo ou consciência que vai dar a noção do que acontece antes ou depois, ou seja, o passado, o presente e o futuro. 

Na experiência do tempo conduzida pela consciência a ordenação da sequência é a essência dessa grandeza e, portanto, mais relevante do que sua origem que é de natureza simultânea.

As predições conscientes do futuro fundamentam-se na probabilidade, Figura \ref{fig:consciousness}, proveniente do caos da expansão lógica, Figura \ref{fig:consciousness_logical_moments}. Logo, o universo tende a ser probabilístico ainda que aleatório em níveis de detalhes, o que faz os eventos serem inusitados ainda que preditos. Quanto maior a quantidade de amostras de um universo, mais forte será sua tendência probabilística, condizente com o teorema central do limite. As amostras distribuídas probabilisticamente fundem o passado, o presente e o futuro na consciência e subconsciências. 

\subsubsection{Espaço}
As ondas da consciência exibidas em forma de histograma, onde as partes das ondas que se completam são colocados lado a lodo é exibida na Figura \ref{fig:consciousness_space_waves}. A formação desse histograma é proveniente do entrelaçamento de ondas.
	\begin{figure}[H]
	\caption{Histograma proveniente do entrelaçamento de ondas}
	\label{fig:consciousness_space_waves}
	\centering
	\includegraphics[scale=.7]{sections/images/consciousness_space_waves.jpg}
	\floatfoot{Exemplo do padrão de ondas obtido pelo algoritmo Logic\_WavePattern. \footnotemark}
	\end{figure}
	\footnotetext{O algoritmo Logic\_WavePattern pode ser visto no Apêndice \ref{app:algoritmos}.}

Ao representar as grandezas espaciais do gráfico da Figura \ref{fig:consciousness_space_waves} em um gráfico de distribuição 3D e distribuir seus pontos de extremidade (desprezando seus volumes e possíveis pontos internos), obtém-se algo parecido com uma espiral (como redemoinhos no ar ou na água) mesmo em volumes muito pequenos de dados (poucos momentos lógicos), conforme Figuras \ref{fig:consciousness_space_3DScatter15000-10} e \ref{fig:consciousness_space_3DScatter_200000-2}. Os pontos se movem em formato de espiral, aproximadamente, uma vez que as coordenadas X, Y e Z aumentam à medida que novas amostras são adicionadas na população.
	\begin{figure}[H]
	\centering
		\begin{subfigure}[H]{0.47\linewidth}
		\centering
		\includegraphics[width=.96\linewidth]{sections/images/consciousness_space_3DScatter15000-10.jpg}
		\caption{15.000 amostras ou momentos}
		\label{fig:consciousness_space_3DScatter15000-10}
		\end{subfigure}
	\hfill
		\begin{subfigure}[H]{0.47\linewidth}
		\centering
		\includegraphics[width=.9\linewidth]{sections/images/consciousness_space_3DScatter_200000-2.jpg}
		\caption{200.000 amostras ou momentos}
		\label{fig:consciousness_space_3DScatter_200000-2}
		\end{subfigure}%
	\caption{Gráfico de dispersão 3D gerado com os pontos da Figura \ref{fig:consciousness_space_waves}}
	\floatfoot{O histograma no padrão de ondas e os dados para gerar o gráfico de dispersão 3D podem ser obtidos com a execução do algoritimo Logic\_WavePattern. \protect\footnotemark}
	\end{figure}
	\footnotetext{O algoritmo Logic\_WavePattern pode ser visto no Apêndice \ref{app:algoritmos} e os gráficos de dispersão 3D podem ser acessados em: \url{https://chart-studio.plot.ly/create/?fid=ren.stuchi:5&fid=ren.stuchi:4} e \url{https://chart-studio.plot.ly/create/?fid=ren.stuchi:7&fid=ren.stuchi:6}}

\subsubsubsection{Comprimento de onda - Intervalo}
As possiblidades de comprimentos de ondas que uma população é capaz de alcançar é fundamentada pela paridade da densidade de amostras de uma população, a qual estabelece uma relação de quantidade por unidade ou intervalo. Essa paridade da densidade, por sua vez, é fruto da destruição de amostras regida pelos fundamentos do teorema central do limite. 

As unidades são isonômicas, tendo o mesmo tamanho ou comprimento em seus respectivos níveis e ocorrem quando existe pelo menos uma amostra (momento lógico) em cada intervalo. A primeira paridade, a unidade base, se dá por meio da mediana (Figura \ref{fig:trend_chart_of_normal_distribution}) e preconiza, devido a isonomia, uma progressão geométrica de razão dois às subunidades da unidade base.

Na Figura \ref{fig:consciousness_space_subconsciousness_intervals} é possível observar que o último intervalo está vazio, portanto essa divisão de intervalos não existe, pois se trata de uma unidade sem quantidade, o que quebra a relação entre unidade e quantidade. A partir do momento em que uma amostra for adicionada nesse range essa divisão de intervalos se faz pertinente e será encontrada. Os intervalos (comprimento de ondas) são todas as possíveis formas de observação de uma população.
	\begin{figure}[H]
	\caption{Intervalo indevido}
	\label{fig:consciousness_space_subconsciousness_intervals}
	\centering
	\includegraphics[scale=.7]{sections/images/consciousness_space_subconsciousness_intervals.jpg}
	\floatfoot{Exemplo de uma divisão de intervalos inexistente.}%\footnotemark}
	\end{figure}
	%\footnotetext{Fonte: note}

Alguns exemplos de comprimentos de ondas podem ser observados na Figura \ref{fig:consciousness_space_subconsciousness_examples}.
	\begin{figure}[H]
	\caption{Diferentes comprimentos de ondas - intervalos}
	\label{fig:consciousness_space_subconsciousness_examples}
	\centering
	\includegraphics[scale=.7]{sections/images/consciousness_space_subconsciousness_examples.jpg}
	\floatfoot{Exemplo de diferentes comprimentos de ondas.}%\footnotemark}
	\end{figure}
	%\footnotetext{Fonte: note}

Em intervalos de muitos momentos lógicos é observado uma discrepância menor e nesses intervalos podem ser observados grandes objetos (subconsciências), conforme Figura \ref{fig:consciousness_space_subconsciousness}, onde o maior deles representado pela cor azul claro e os menores e mais distantes pela cor azul escuro e roxo. Esses intervalos podem representar, por exemplo, o centro do universo, então o centro de uma galáxia, estrelas, planetas e objetos menores e mais distantes.
	\begin{figure}[H]
	\caption{Abstração espacial das subconsciências - grandes agrupamentos}
	\label{fig:consciousness_space_subconsciousness}
	\centering
	\includegraphics[scale=.45]{sections/images/consciousness_space_subconsciousness.jpg}
	\floatfoot{Caracteristicas da ondas formadoras da subconsciência de grandes objetos.}%\footnotemark}
	\end{figure}
	%\footnotetext{Fonte: note}

Em intervalos com uma quantidade menor de momentos lógicos é observado uma discrepância maior e nesses intervalos podem ser observalos pequenos objetos (subconsciências). Quanto menores os agrupamentos menos divisões (cores) esses agrupamentos têm e mais estreitos e compridos eles são, conforme Figura \ref{fig:consciousness_space_subconsciousness_min}. Esses intervalos podem representar, por exemplo, o átomo que são muito pequenos, se apresentam em enormes quantidades e as partículas que orbitam seu núcleo (elétrons) ficam bem mais distantes dele.
	\begin{figure}[H]
	\caption{Abstração espacial das subconsciências - pequenos agrupamentos}
	\label{fig:consciousness_space_subconsciousness_min}
	\centering
	\includegraphics[scale=.7]{sections/images/consciousness_space_subconsciousness_min.jpg}
	\floatfoot{Caracteristicas da ondas formadoras da subconsciência de pequenas partículas.}%\footnotemark}
	\end{figure}
	%\footnotetext{Fonte: note}}
	
As cores dos agrupamentos indicam a relação entre conjuntos e subconjuntos. Subconjuntos nascem do conjunto inicial ou de outros subconjuntos e essa relação paterna filial é permanente. Conjunto e subconjuntos também podem se dividir no mesmo nível, a depender do comprimento e amplitude de ondas.

Os intervalos de ondas (comprimentos de ondas) que uma subconsciência (sub-lógica) é capaz de observar depende do comprimento de ondas que a própria subconsciência é constituída. Dentre todas as possibilidades de intervalos ou comprimento de ondas permitidos por uma população, o observador está em um deles. O universo não tem uma forma definida, é o observador presente em uma das possibilidades de comprimentos de onda que observa essas amostras de forma condizente a esses comprimentos de ondas. 

Uma característica importante do processo de observação de pequenas quantidades de amostras de um conjunto ou subconjunto é que elas podem ser observadas com partículas ou ondas, conforme Figura \ref{fig:consciousness_space_wave-particle}. Nessa Figura é contemplado um pequeno subconjunto do subconjunto 1, análogo a um fóton, como exemplo. O fóton observado como partícula (um subconjunto) se move à medida que novas amostras vão sendo adicionadas, substancialmente na parte inferior do comprimento de ondas. O fóton como onda trafega informação de um conjunto ou subconjunto maior, um objeto maior. Conforme novas amostras vão sendo adicionadas, substancialmente na parte inferior de cada comprimento de onda, esse pequeno intervalo representado pelo fóton passa a receber outras amostras, essencialmente inferiores, o que dá o efeito de onda. O fóton como onda é parte de uma informação maior e, portanto, esse intervalo se move com essa informação, tornando-se parte de outros objetos quando este subconjunto se dissolve em outros ocasionalmente.
	\begin{figure}[H]
	\caption{Observador - onda-partícula}
	\label{fig:consciousness_space_wave-particle}
	\centering
	\includegraphics[scale=.65]{sections/images/consciousness_space_wave-particle.jpg}
	\floatfoot{Características da observação de uma pequena parte de um subconjunto.}%\footnotemark}
	\end{figure}
	%\footnotetext{Fonte: note}}
	
\subsubsubsection{Amplitude de onda}
Ao adicionar uma nova amostra todo o intervalo se distribui proporcionalmente para acoplar essa amostra. Ao dividir esse intervalo em comprimentos de ondas menores suas amplitudes de ondas obedecerão a distribuição de amostras desses subintervalos proporcionalmente, conforme Figura \ref{fig:consciousness_space_volume_amplitude}.
	\begin{figure}[H]
	\caption{Comprimento vs Amplitude de onda}
	\label{fig:consciousness_space_volume_amplitude}
	\centering
	\includegraphics[scale=.4]{sections/images/consciousness_space_volume_amplitude.jpg}
	\floatfoot{Relação de comprimento e amplitude de ondas.}%\footnotemark}
	\end{figure}
	%\footnotetext{Fonte: note}}
	
A área cresce de forma quadrática ao crescimento da amplitude de uma onda (colunas do histograma), uma vez que o salto provocado pelo entrelaçamento de ondas e a própria distribuição probabilística das amostras tendem a manter um crescimento equivalente no par de colunas que formam uma onda, conforme visto nas Figuras \ref{fig:consciousness_subconscious} e \ref{fig:consciousness_space_subconscious_observation_jump}. Ao imaginar uma esfera com o diâmetro equivalente à amplitude de um intervalo, a parte mais facilmente observável é onde está a maior concentração de amostras desse intervalo, que é sentido à mediana da população, probabilisticamente, conforme visto na Figura \ref{fig:consciousness_space_volume}.
	\begin{figure}[H]
	\caption{Amostras vs volume}
	\label{fig:consciousness_space_volume}
	\centering
	\includegraphics[scale=1]{sections/images/consciousness_space_volume.jpg}
	\floatfoot{O volume cresce de forma quadrática, aproximadamente, ao crescimento da amplitude de uma onda (colunas do histograma).}%\footnotemark}
	\end{figure}
	%\footnotetext{Fonte: note}}

\subsubsubsection{Espiral}
O padrão de espiral observado na Figura \ref{fig:consciousness_space_spiral} é fundamentado pelo comprimento de ondas, amplitude de ondas e entrelaçamento de ondas, a base para formação do espaço, seguindo o padrão probabilístico descrito pelo teorema central do limite, onde as amostras da população tendem sentido à mediana. O padrão de espiral observado não invalida outros possíveis movimentos no espaço. Muitas vezes não é possível observar o padrão de espiral imediatamente nos movimentos de uma subconsciência, porém esse padrão está por traz de muitos destes movimentos. Ao pegar os movimentos humanos, como exemplo, tem-se os ciclos predominantes de ir e voltar para casa, ir e voltar ao trabalho, acordar e dormir, ou seja, os hábitos se assemelham de movimentos em ciclos, movimentos espirais.
	\begin{figure}[H]
	\caption{Padrão do movimento em espiral}
	\label{fig:consciousness_space_spiral}
	\centering
	\includegraphics[scale=.6]{sections/images/consciousness_space_spiral.jpg}
	\floatfoot{Detalhes do movimento em espiral dos subconjuntos de amostras.}%\footnotemark}
	\end{figure}
	%\footnotetext{Fonte: note}}

Como as coordenadas X, Y e Z de cada subconjunto tendem a aumentar, a disposição dessas em um sistema tridimensional de coordenadas vai seguir uma referência diagonal entre esses três eixos, conforme Figura \ref{fig:consciousness_space_spiral_reference_line}.

Na Figura \ref{fig:consciousness_space_spiral_reference_line} pode ser observado também os pontos X1 e X2. Esses pontos foram espelhados nas coordenadas X e Z para facilitar a observação de que ao elevar o eixo Z também se eleva o eixo X, independente de seus pontos probabilísticos médios mínimos e médios máximos.
	\begin{figure}[H]
	\caption{Sistema tridimensional de coordenadas}
	\label{fig:consciousness_space_spiral_reference_line}
	\centering
	\includegraphics[scale=.8]{sections/images/consciousness_space_spiral_reference_line.jpg}
	\floatfoot{Linha de referência para distribuição de uma população em um plano tridimensional.}%\footnotemark}
	\end{figure}
	%\footnotetext{Fonte: note}}

Probabilisticamente a adição dessas novas amostras tendem a estar na base dos subintervalos de um objeto ou subconjunto, o que é chamado na Figura \ref{fig:consciousness_space_spiral_undulation} de alta densidade de amostras. Assim as adições dessas amostras aumentam a amplitude desses subintervalos movimentando ou deformando suas ondas internas para cima. A passagem das cristas e vales dessas ondas dentro dos subintervalos de um objeto, como mostrado subconjunto 1 do bloco 1 da figura abaixo, fazem com que os eixos X e Y de cada respectivo subintervalo dentro desse objeto passem por ondulações espirais.

Esses movimentos espirais são observados dentro dos subintervalos de um objeto, assim como para observar o movimento ondulatório espiral do objeto em si seria necessária a observação de um comprimento de ondas maior, onde todo o conjunto da qual faz parte o objeto fosse observado como um único subintervalo, como exibido na parte inferior da figura abaixo.

	\begin{figure}[H]
	\caption{Movimentação de ondas dentro dos intervalos que compõem um objeto}
	\label{fig:consciousness_space_spiral_undulation}
	\centering
	\includegraphics[scale=.65]{sections/images/consciousness_space_spiral_undulation.jpg}
	\floatfoot{Amplitude das ondas no interior dos intervalos (comprimento de ondas) que compõem um objeto.}%\footnotemark}
	\end{figure}
	%\footnotetext{Fonte: note}}

Na Figura \ref{fig:consciousness_space_spiral_direction} é exibida a orientação da parte facilmente visível de um objeto juntamente com o espaço que completa a formação deste objeto. A parte facilmente observável probabilisticamente encabeça o movimento, uma vez que as maiores densidades de amostras estão nos intervalos sentido à mediana, podendo essas altas densidades estarem mais na parte superior ou inferior de um subconjunto (a depender das ondas internas de cada intervalo que compõem o subconjunto), como o subconjunto 1 da figura abaixo.
	\begin{figure}[H]
	\caption{Orientação do movimento devido às ondas internas de um subconjunto}
	\label{fig:consciousness_space_spiral_direction}
	\centering
	\includegraphics[scale=1]{sections/images/consciousness_space_spiral_direction.jpg}
	\floatfoot{Orientação do movimento de um objeto devido à movimentação ondas internas de cada intervalo que o compõem e a relação de sua parte visível com espaço o que o completa.}%\footnotemark}
	\end{figure}
	%\footnotetext{Fonte: note}}

Cada intervalo ou subintervalo (comprimento de ondas) tem sua própria linha de referência. Assim como dentro de um metro existem os centímetros, milímetros etc., dentro de um intervalo e subintervalos podem existir inúmeros outros.
	\begin{figure}[H]
	\caption{Intervalos e linhas de referências}
	\label{fig:consciousness_space_spiral_underlines}
	\centering
	\includegraphics[scale=.5]{sections/images/consciousness_space_spiral_underlines.jpg}
	\floatfoot{Espirais em diferentes intervalos e suas linhas de referências.}%\footnotemark}
	\end{figure}
	%\footnotetext{Fonte: note}}

\subsubsection{Forças fundamentais}
A força gravitacional, a força eletromagnética e a força nuclear correspondem às forças fundamentais da natureza e essas forças também são provenientes do entrelaçamento de ondas, como o espaço. As forças fundamentais não são forças propriamente, mas sim aspectos probabilísticos (distribuição normal) e do entrelaçamento de ondas principalmente.

\subsubsubsection{Força gravitacional}
O entrelaçamento ondas é o aspecto que coordena as mudanças nas coordenadas espaciais junto com a probabilidade de distribuição de novos momentos lógicos sentido a mediana da população. As mudanças dessas coordenadas provocam iterações que podem ser vistas nas Figuras \ref{fig:consciousness_space_3DScatter15000-10} e \ref{fig:consciousness_space_3DScatter_200000-2} da subseção de Espaço e na Figura \ref{fig:consciousness_dark_matter_dark_energy_wave} que mostra probabilisticamente onde está a maior concentração de momentos lógicos de um intervalo consciente ou subconsciente, uma vez que estes momentos são mais intensos sentido a mediana. A força gravitacional não é uma força propriamente e sim aspectos do entrelaçamento ondas, probabilidade de distribuição de novas amostras e do comprimento e amplitude de ondas.

Na Figura \ref{fig:consciousness_gravitational_force} pode ser visto os subconjunto 1 e 4. A parte mais facilmente observável de cada subconjunto caminha para cima e para direita, em uma diagonal que depende da distribuição probabilística de novas amostras nesses subconjuntos, conforme Figura \ref{fig:consciousness_space_volume}, sendo que essa parte facilmente observável pode crescer para atrás ainda que sua tendência é ir para frente diagonalmente a medida que as colunas em azul da Figura \ref{fig:consciousness_gravitational_force} tanto esquerda quanto direita, que cresce com maior velocidade, cheguem ao tamanho mínimo das colunas do subconjunto 1, por exemplo.

A coluna do histograma destacada em azul, na Figura \ref{fig:consciousness_gravitational_force}, cresce de forma quadrática, uma vez que o salto provocado pelo entrelaçamento de ondas e a própria distribuição probabilística das amostras tendem a manter um crescimento equivalente no par de colunas que formam essa parte onda. Esse aspecto configura a lei do inverso do quadrado, onde, no caso da gravidade, quando mais perto os objetos menores serão a área quadrada entre eles (menor possibilidades de posicionamento das amostras), o que torna a aproximação mais rápida (com adição de menos momentos lógicos). Assim, quanto mais longe os objetos, maior a área, maior as possibilidades de posicionamento, mais momentos lógicos caracterizando assim uma atração menor.
	\begin{figure}[H]
	\caption{Força gravitacional}
	\label{fig:consciousness_gravitational_force}
	\centering
	\includegraphics[scale=.9]{sections/images/consciousness_gravitational_force.jpg}
	\floatfoot{Aspectos gravitacionais do entrelaçamento ondas e da probabilidade de distribuição de novas amostras dentro de um intervalo.}%\footnotemark}
	\end{figure}
	%\footnotetext{Fonte: note}

Outro fator importante que pode ser observado na Figura \ref{fig:consciousness_gravitational_force}, na coluna do histograma destacada em azul é que ao adicionar uma nova amostra nessa coluna, mesmo que no início da coluna, fará com que está coluna cresça proporcionalmente à adição dessa nova amostra e somente o subconjunto 4 terá o crescimento proporcional total dessa coluna. Assim a adição de uma amostra nessa coluna no subconjunto 0, faria o subconjunto 0 alongar 1/5 de uma amostra nessa coluna, o subconjunto 1 alongar 2/5 de uma amostra até o subconjunto 4 alongar 5/5 de uma amostra nessa coluna. Esse alongamento das ondas é observado como a atração de objetos de baixo para cima da parte mais facilmente observável de cada subconjunto. Isso também pode facilitar o entendimento do adiantamento dos relógios atómicos nos satélites. Objetos em grandes distancias sofrem uma atração menor, porém os intervalos de amostras obtidos são maiores.

\subsubsubsection{Força eletromagnética}
A força eletromagnética é uma especificação do aspecto gravitacional que depende da aproximação espacial (redução de diferenças nos eixos X, Y e Z) e do entrelaçamento de ondas.

Quando um objeto se aproxima de outro, seus pares de ondas provenientes do entrelaçamento de ondas ficam cada vez mais parecidos, eixos X e Y. Essa proximidade faz com que as partes das ondas de um objeto se pareça muito com as partes das ondas do outro objeto, o que pode fazer com que o entrelaçamento de ondas encontre pares mais ideais nesse outro objeto e vice-versa.  

As linhas azuis da Figura \ref{fig:consciousness_electromaagnetic_force} mostra onde é mais frequente a troca dos pares de ondas pelo entrelaçamento de ondas, ou seja, onde se tem a maior probabilidade das ondas serem parecidas. Por isso os imãs tentam se virar para se conectar quando estão face a face com o mesmo polo. As linhas cinza mostram as conexões que ocorrem em número bem menor. 
	\begin{figure}[H]
	\caption{Força eletromagnética}
	\label{fig:consciousness_electromaagnetic_force}
	\centering
	\includegraphics[scale=.7]{sections/images/consciousness_electromaagnetic_force.jpg}
	\floatfoot{Aumento das possibilidades de entrelaçamento de ondas devida a aproximação e o menor número de momentos lógicos das menores partículas. }%\footnotemark}
	\end{figure}
	%\footnotetext{Fonte: note}

Com a troca de significativos pares de ondas entre os objetos faz-se a mixagem do posicionamento dos eixos X, Y e Z entre esses objetos ocorrendo a aproximação deles no espaço. 

Quanto menor a partícula (elétron ou partículas menores), conforme Figura \ref{fig:consciousness_space_subconsciousness_min}, mais fácil o entrelaçamento ocorre. Provavelmente muitos objetos não tenham alta capacidade de entrelaçamento devido aos seus elétrons ou partículas menores serem formadas por muitos momentos lógicos (muitas barras compridas do histograma), ou seja, quanto maior a quantidade de momentos dessas partículas menores as chances de entrelaçamento, porém a quantidade de momentos lógicos deve ser grande o suficiente para que estes tenham baixa entropia, ou seja, que seu padrão de distribuição seja parecido ao da distribuição normal para este determinado momento e posição dentro de todo o intervalo. 

Como visto na Figura \ref{fig:consciousness_electromaagnetic_force_entropy}, um objeto, com um átomo, pode ser formado por diversas colunas do histograma, como o padrão ondular (cinco, seis e sete), por exemplo, porém a posição de cada uma das cinco, seis e sete amostras em suas respectivas colunas podem assumir arranjos muito diferente, isso é a entropia que deve se aproximar ao padrão da distribuição normal (baixa entropia), pois esse é o padrão probabilístico de todo o intervalo e portanto o que tem a maior probabilidade de ocorrer, maximizando as possibilidades de saltos.
	\begin{figure}[H]
	\caption{Força eletromagnética - entropia}
	\label{fig:consciousness_electromaagnetic_force_entropy}
	\centering
	\includegraphics[scale=.9]{sections/images/consciousness_electromaagnetic_force_entropy.jpg}
	\floatfoot{Aumento das possibilidades de entrelaçamento de ondas devido à baixa entropia, favorecida pelo menor número de momentos lógicos. }%\footnotemark}
	\end{figure}
	%\footnotetext{Fonte: note}

Probabilisticamente as partículas mais parecidas estão nas regiões mais próximas (linhas azuis do Figura \ref{fig:consciousness_electromaagnetic_force}) devido ao crescimento do número de amostras sentido a mediana da população, porém isso não é uma regra e os polos podem se inverter, ou seja, ter mais ligações com a região de menor probabilidade (isso não quer dizer que houve formação de antimatéria nessa região, as partículas ainda tendem a concentrar mais momentos lógicos sentido à mediana da população). No entanto, a probabilidade tende a corrigir esses polos conforme novos momentos vão sendo adicionadas nesse intervalo.

\subsubsubsection{força nuclear}
As forças nucleares forte e fraca representam as maiores concentrações de momentos lógicos por intervalo populacional. Esses picos podem ser vistos na Figura \ref{fig:consciousness_space_subconsciousness_min} e eles não param de crescer à medida que novos momentos lógicos são adicionados nestes intervalos. Estes momentos ou amostras tendem a estarem cada vez mais juntos dentro do intervalo formando picos cada vez mais altos.

\subsubsection{Matéria escura e energia escura}
Quanto maior o número de amostras e mais próximas elas estão da mediana, mais elas farão parte dos 99,99\% e ainda mais amostras também estarão nos 0,01\%, conforme a Tabela \ref{tab:10000_all}. Ao adicionar uma nova amostra todo o intervalo se distribui proporcionalmente para acoplar essa amostra, conforme observado na Figura \ref{fig:consciousness_space_volume_amplitude}. Logo, a energia escura não é uma energia propriamente, mas sim um aspecto probabilístico.
	\begin{figure}[H]
	\caption{Aspecto probabilístico da energia escura}
	\label{fig:consciousness_dark_matter_dark_energy}
	\centering
	\includegraphics[scale=.9]{sections/images/consciousness_dark_matter_dark_energy.jpg}
	\floatfoot{A energia escura não é uma energia propriamente, mas sim um aspecto probabilístico.}%\footnotemark}
	\end{figure}
	%\footnotetext{Fonte: note}

A Figura \ref{fig:consciousness_dark_matter_dark_energy_wave} mostra probabilisticamente onde está a maior concentração das amostras de um intervalo, tornando assim mais fácil a visualização dessa maior concentração por outras subconsciências, uma vez que a adição de novas amostras nesse ponto de maior concentração fará com que todo o intervalo se distribua proporcionalmente tornando as amostras mais distantes da mediana mais dispersas. Outro fator que contribui para facilitar a observação dessa maior concentração de amostras é o fato de que a área cresce de forma quadrática ao crescimento da amplitude de uma onda (colunas do histograma), devido ao salto provocado pelo entrelaçamento de ondas e a própria distribuição probabilística das amostras que tendem a manter um crescimento equivalente no par de colunas que formam uma onda. Assim, uma grande área do intervalo de um agrupamento pode conter amostras dispersas que se tornam mais difíceis de observar. O aspecto descrito acima e demostrado pela Figura \ref{fig:consciousness_dark_matter_dark_energy_wave} é aplicável a qualquer intervalo de um agrupamento (Figuras \ref{fig:consciousness_space_subconsciousness} e \ref{fig:consciousness_space_subconsciousness_min}).
	\begin{figure}[H]
	\caption{Analogia da matéria escura}
	\label{fig:consciousness_dark_matter_dark_energy_wave}
	\centering
	\includegraphics[scale=.85]{sections/images/consciousness_dark_matter_dark_energy_wave.jpg}
	\floatfoot{Parte do volume é facilmente observado por outras subconsciências.}%\footnotemark}
	\end{figure}
	%\footnotetext{Fonte: note}

\subsubsection{Antimatéria}
Independente do intervalo observado, sua maior concentração de amostras tende a estar sentido da mediana, o que é o sentido provável conforme teorema central do limite. Essas amostras também podem estar com sua concentração no sentido oposto à mediana, porém com uma ocorrência probabilística cada vez menos conforme as amostras aumentam. Na Figura \ref{fig:consciousness_concentration_of_opposite_samples} é exibido dois intervalos idênticos com suas amostras com concentrações opostas.
	\begin{figure}[H]
	\caption{Parte de um intervalo idêntico com suas concentrações de amostras opostas}
	\label{fig:consciousness_concentration_of_opposite_samples}
	\centering
	\includegraphics[scale=1.2]{sections/images/consciousness_concentration_of_opposite_samples.jpg}
	\floatfoot{Parte de um intervalo idêntico distribuídos de formas opostas.}%\footnotemark}
	\end{figure}
	%\footnotetext{Fonte: note}

O merge ou soma dos intervalos opostos da Figura \ref{fig:consciousness_concentration_of_opposite_samples} os tornaria um intervalo simétrico, ou seja, não estaria em nenhum dos sentidos.
Na Figura \ref{fig:consciousness_concentration_of_opposite_samples_within_range} é exibido um intervalo consciente completo com suas concentrações de amostras sentido à mediana e outro idêntico, mas com suas concentrações sentido às bordas do intervalo.
	\begin{figure}[H]
	\caption{Intervalos conscientes com suas concentrações de amostras opostas}
	\label{fig:consciousness_concentration_of_opposite_samples_within_range}
	\centering
	\includegraphics[scale=.7]{sections/images/consciousness_concentration_of_opposite_samples_within_range.jpg}
	\floatfoot{Intervalos conscientes completos e idênticos distribuídos de formas opostas.}%\footnotemark}
	\end{figure}
	%\footnotetext{Fonte: note}


\subsubsection{Buraco negro}
O buraco negro é uma concentração muito alta de amostras, formada por grandes agrupamentos subconscientes, Figura \ref{fig:consciousness_space_subconsciousness}.
Esses grandes agrupamentos ocupam grandes volumes de espaço devido a quantidade de amostras. 

Os grandes volumes são encontrados na base dos grandes agrupamentos, conforme as cores azul claro e cinza da Figura \ref{fig:consciousness_black_hole}.
	\begin{figure}[H]
	\caption{Buracos negros}
	\label{fig:consciousness_black_hole}
	\centering
	\includegraphics[scale=.6]{sections/images/consciousness_black_hole.jpg}
	\floatfoot{Grandes volumes são encontrados na base dos grandes agrupamentos.}%\footnotemark}
	\end{figure}
	%\footnotetext{Fonte: note}