% ----------------------------------------------------------
% Considerações Finais
% ----------------------------------------------------------
\section*{Considerações Finais}
\addcontentsline{toc}{section}{Considerações finais}
Este é um estudo da lógica primordial que resultou em uma teoria a respeito da origem de tudo. Todas as linhas de raciocínio deste estudo podem ser aprofundadas e detalhadas. 

Eventualmente pode ser considerado um estudo filosófico e/ou científico, entretanto a base desses dois importantes ramos é a lógica, o núcleo dessa teoria. 

A resposta da pergunta central desse estudo (se existe algo ao invés de nada) vem da lógica. O estudo da lógica deu origem a uma teoria a respeito da origem de todas as coisas. Essa teoria reponde o que é a consciência, as ondas, o infinito, o tempo, o espaço, as forças fundamentais, a matéria escura, a energia escura, a antimatéria, o buraco negro e o observador e a vida.

Que o modelo desse estudo seja o início de uma nova era. Uma era onde o ser humano possa desenvolver a si e observar que é o hospedeiro do infinito. Que essa evolução possa transformar os sonhos em realidade e que seja possível observar que a realidade não é diferente de um sonho, uma vez que ambas são apenas lógicas.

Pensar que algo físico tenha surgido do nada se faz incoerente com a natureza do nada.