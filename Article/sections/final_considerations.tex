% ----------------------------------------------------------
% Considerações Finais
% ----------------------------------------------------------
\section*{Considerações Finais}
\addcontentsline{toc}{section}{Considerações finais}
Este é um estudo da lógica que resultou em uma teoria a respeito da origem de tudo. Todas as linhas de raciocínio deste estudo podem ser aprofundadas e detalhadas. 

Eventualmente pode ser considerado um estudo filosófico e/ou científico, entretanto a base desses dois importantes ramos é a lógica, o núcleo dessa teoria. 

A resposta da pergunta central desse estudo (se existe algo ao invés de nada) vem da lógica em sua dualidade, que por um lado NÃO É e por outro É ilógica, imutável e inexistente, uma vez que a existência está em tudo aquilo que NÃO É. A resposta a essa pergunta está na compreensão de que a lógica em sua essência remete ao nada, que NÃO É, ou seja, nega a si mesmo (nega ser). A negação de si gera expansões binomiais, no qual suas amostras combinadas em cada passo dessa expansão se aproximam da distribuição normal e se aproximam do centro dessa distribuição infinitamente, o que configura o teorema central do limite. Os passos da expansão binomial regidos pela probabilidade descrita no teorema central do limite compreendem a consciência e tornam visíveis o que é e o porquê de seus aspectos mais perceptíveis: infinito, tempo, espaço, gravidade, matéria escura, energia escura e buraco negro.

Pensar que algo diferente de ilógico, imutável e inexistente tenha surgido do nada se faz incoerente com sua natureza nula.