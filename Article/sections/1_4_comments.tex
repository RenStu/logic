% ----------------------------------------------------------
% Subseção Observações
% ----------------------------------------------------------
\subsection{Observações}
	\begin{description}
	   \item[Núcleo] A negação de si da lógica (ou nada) deu origem a três axiomas que são a base do teorema núcleo dessa teoria e a base para a existência. Teorema esse que dá origem as ondas e a seu principal atributo, o entrelaçamento de ondas. 
	   \item[Rigidez lógica] Se a rigidez física e suas leis parecem ser intransponíveis, abaixo dela está à lógica, ainda mais rígida e intransponível, pois fora da lógica o que se tem é o inexistente, o ilógico. A existência está contida nas possibilidades do que é lógico. 
	   \item[Matemática] A lógica em sua essência não está sujeita à matemática, mas toda a matemática está restrita à lógica e, portanto, algumas de suas construções mais simples podem se aproximar mais da lógica essencial do que outras.
	   \item[Biologia] Talvez este estudo possa aprofundar ou ajudar a esclarecer a chamada transferência horizontal de genes, uma vez que a probabilidade de distribuição amostral em subintervalos com aproximações espaciais ou temporais podem fazer com que estes compartilhem semelhanças em seus padrões de desenvolvimento. 
	   \item[Bem e mal] O bem e o mal dependem do observador e são apenas possibilidades válidas dentre infinitas outras (talvez a grande justiça do universo ou da lógica – a não exclusão de qualquer caminho). Ou seja, se está claro a negação tende a escurecer, se está calor a esfriar etc. É a briga dos contrários de Heráclito de Éfeso. 
	   \item[Perfeição] A lógica primordial é a mais simples das lógicas, é a essência da existência. Uma lógica tão simples quanto eficiente, tão eficiente quanto perfeita:
	   \begin{description}
		   \item[Onipotente] A essência de todas as possibilidades lógicas, ou seja, a essência da existência, pois fora das possibilidades lógicas está o ilógico, o inexistente;
		   \item[Onisciente] Fluxo de todas as abstrações lógicas desde a consciência às subconsciências; 
		   \item[Onipresente] Suas frações (negações) estão em toda a existência.
	   \end{description}
	Essas observações remetem a Deus, a consciência das subconsciências. Em última análise Deus é Lógica desde sua ínfima autonegação à sua infinita grandeza. Deus é amor, portanto, a ordem ou atração que está presente em todas as escalas da existência.
	   \item[Igualdade] A igualdade é infinitamente improvável no universo. A diferença é a chave da vida. Como exemplo, o racismo (predominância de um grupo social ou étnico) é uma forma de igualdade, pois não aceita os outros grupos. Tentar nivelar a sociedade em todos os seus aspectos também é uma forma de igualdade, pois não aceita as variações que ela pode ter. Até na Bíblia houve luta por igualdade, posto que Jesus sendo Deus (trindade) aceitou e disse que o pai é maior que Ele, no entanto, o anjo caído quis ser igual a Deus e foi expulso por Miguel, cujo nome quer dizer “Quem é igual a Deus?” ou “ninguém é igual a Deus”. Com essas observações fica claro que transigir com o desrespeito as diferenças é transigir contra a vida, pois respeitar a vida é amar e proteger sua diversidade. Quando mais diversidade mais rica é a vida, portanto, viver ou deixar viver em situações que contrapõem o amor, quando há condições contrárias a isso, é um desrespeito à vida, pois ameaça sua diversidade. 
	   \item[Entropia] O desequilíbrio da distribuição do universo tende a aumentar. Logo o universo vai do equilíbrio ou igualdade total ao total desequilíbrio ou desigualdade. A essência probabilística da lógica primordial \underline{NÃO SER} caracteriza um desiquilíbrio total em sua distribuição. Apesar de desordenado, o caos não é tão desequilibrado. É mais fácil equalizar ou equilibrar o caos que tende a ser menos discrepante por ser mais igualmente distribuído, do que a lógica primordial \underline{NÃO SER} em seu total desequilíbrio em forma de onda. Por ser mais equilibrado em sua distribuição, o caos está mais próximo do equilíbrio ou igualdade total inicial. Assim, o caos tem uma desordem maior (entropia maior) ainda que com um desequilíbrio menor, quando comparado com a distribuição a lógica primordial \underline{NÃO SER} que formam as ondas, as quais tem uma desordem menor (entropia menor) ainda que tenham um desequilíbrio maior e por isso estejam mais longe do equilíbrio ou igualdade total inicial.
	   \item[Realidade] Como possibilidade lógica, o sonho é tão real quando a "realidade". Talvez o estudo das possibilidades lógicas leve a caminhos onde os sonhos possam ser tão reais quanto à realidade, já que os dois não passam de lógica, como sonhos lúcidos, por exemplo \cite{ administradores_principio_pareto}. Isso talvez explique por que outras possíveis formas de vidas "inteligentes", quando evoluídas, deixam de buscar esse tipo de vida em um possível vasto universo à procurarem dentro de si, onde se pode encontrar algo bem maior que o universo, o infinito.
	   \item[Convergência] O salto e o entrelaçamento quânticos são comportamentos que desafiam o mundo físico, podendo ser o ponto convergente com o novo paradigma.
	\end{description}