% ----------------------------------------------------------
% Subseção Observações
% ----------------------------------------------------------
\subsection{Observações}

\begin{description}
   \item[Núcleo] Este estudo está centrado na expansão binomial e na distribuição aleatória de amostras que combinadas em cada passo dessa expansão se aproximam da distribuição normal e se aproximam do centro dessa distribuição infinitamente.
   \item[Rigidez lógica] Se a rigidez física e suas leis parecem ser intransponíveis, abaixo dela está à lógica, ainda mais rígida e intransponível, pois fora da lógica o que se tem é o inexistente ilógico. A existência está dentro das possibilidades lógicas. 
   \item[Matemática] A matemática é uma ótima abstração do universo, mas ela não é a linguagem do universo, pois abaixo da matemática está à lógica, a base da matemática e de toda a existência.
   \item[Bem e mal] O bem e o mal são observações da consciência. Ou seja, se está claro a negação tende a escurecer, se está calor a esfriar etc. É a briga dos contrários de Heráclito de Éfeso. Os contrários tendem a se equilibrarem.
   \item[Perfeição] A lógica primordial é a mais simples das lógicas, é a essência da existência. Uma lógica tão simples quanto eficiente, tão eficiente quanto é perfeição. A mais poderosa lógica:
   \begin{description}
	   \item[Onipotente] A essência da existência;
	   \item[Onisciente] Fluxo de todas as abstrações lógicas descendentes; 
	   \item[Onipresente] Suas frações estão em toda a existência.
   \end{description}
   \item[Realidade] Como possibilidade lógica o sonho é tão real quando a "realidade". Talvez o estudo das possibilidades lógicas leve a caminhos onde os sonhos possam ser tão reais quanto à realidade, já que os dois não passam de lógica, como sonhos lúcidos \cite{ administradores_principio_pareto}. Isso talvez explique por que outras possíveis formas de vidas "inteligentes", quando evoluídas, deixam de buscar esse tipo de vida em um possível vasto universo à procurarem dentro de si, onde se pode encontrar algo bem maior que qualquer universo, o infinito.
   \item[Foco] É provável que o foco da observação consciente esteja nas amostras ao redor e na mediana de uma população, onde está o foco das mudanças. A probabilidade dessa constante centralização, desse “puxar para dentro” tornando mais irrelevantes as amostras à margem talvez faça com que certos animais sintam a experiência do sono, pois esse foco cada vez mais ao centro pode ir desfocando os sentidos mais distantes desse foco.
\end{description}